
\section{Introduction}

This document is meant to discuss the different components of the
EBAMRTools component of the EBChombo infrastructure for embedded
boundary, block-structured adaptive mesh  applications.  The principal
operations that these tools execute are as follows:
\begin{itemize}
\item Average a level's worth of data onto the next coarser level.
\item Interpolate in a piecewise-linear fashion data from a coarser
      level to a finer level.
\item Fill ghost cells at a coarse-fine interface with a second-order
        interpolation between the coarse and fine data.
\item Fill ghost cells at a coarse-fine interface with data
        interpolated using a bilinear interpolation.
\item Preserve multi-level conservation using refluxing.
\item Redistribute mass differences between stable and conservative
schemes. 
\end{itemize}
After a discourse on the notational difficulties of embedded
boundaries, we will discuss our algorithm for each of these tasks.

\section{Notation}

All these operations take place in a very similar context to
that presented in \cite{ChomboDesign}.  For non-embedded boundary
notation, refer to that document. 
The standard $(i,j,k)$ is not sufficient here to denote a
computational cell as there can be multiple VoFs per cell.
We define $\vbold$ to be the notation for a VoF and $\fbold$
to be a face. The function $ind(\vbold)$ produces the 
cell which the VoF lives in.  We define $\vbold^+(f)$ to be the
VoF on the high side of face~$f$; $\vbold^-(f)$ is the VoF on the low side
of face~$\fbold$; $\fbold^+_d(v)$ is the set of faces on the high side of
VoF~$v$; $f^-_d(v)$ is the set of faces on the low side of VoF~$\vbold$,
where $d\in\{x,y,z\}$ is a coordinate direction (the number of
directions is $D$).  Also, we compose these operators to represent the
set of VoFs directly connected to a given VoF:  $v^+_d(v) =
\vbold^+(f^+_d(\vbold))$ and $\vbold^-_d(v) = v^-(f^-_d(\vbold))$.
The $<<$ operator shifts data in the direction of the right hand argument:
\begin{equation}
(\phi << \ebold^d)_\vbold =  \phi_{v^+_d(v)}
\end{equation}

We follow the same approach in the EB case in defining multilevel data
and operators as we did for ordinary AMR. Given an AMR mesh hierarchy
$\{ \Omega^l \}^{lmax}_{l=0}$, we define the valid VoFs on level $l$ to be 
\begin{equation}
\mathcal{V}^l_{valid} = ind^{-1}(\Omega^l_{valid})
\end{equation}
and composite cell-centered data
\begin{equation}
\varphi^{comp} = \{ \varphi^{l, valid} \}^{lmax}_{l=0}, 
\varphi^{l,valid} : \mathcal{V}^l_{valid} \rightarrow \mathbb{R}^m
\end{equation}
For face-centered data,
\begin{equation}
\begin{array}{c}
\mathcal{F}^{l, d}_{valid} = ind^{-1}(\Omega^{l, \ebold^d}_{valid}) \\
\vec{F}^{l, valid} = (F^{l, valid}_0, \ldots, F^{l, valid}_{D-1}) \\
F^{l, valid}_d : \mathcal{F}^{l, d}_{valid} \rightarrow \mathbb{R}^m
\end{array}
\end{equation}

\section{Conservative Averaging}

Assume that there are two levels of grids $\Omega^c, \Omega^f,$ with
data defined on the fine grid and on the valid region of the coarse
grid 
\begin{equation}
\varphi^f : \ind^{-1}(\Omega^f) \rightarrow \mathbb{R} \\
\varphi^{c, valid} : \ind^{-1}(\Omega^c_{valid}) \rightarrow \mathbb{R}
\end{equation}
We assume that $\mathcal{C}_r(\tilde{\Omega}^f) \cap \Gamma^c \subset
\Omega^c$.   We want to replace the coarse data which is covered
by fine data with the volume-weighted average of the fine data.
This operator is used to average
from finer levels on to coarser levels, or for constructing averaged
residuals in multigrid iteration.  We define the volume weighted average
\begin{equation}
\begin{array}{c}
\varphi^c_{\vboldc} = Av(\varphi^f, n_{ref})_{\vboldc}  \\
Av(\varphi^f)= \frac{1}{V^c} \sum\limits_{{\vboldf} \in {\cal F}} V^f
\varphi_{\vboldf} \\
{\cal F} =  \mathcal{C}^{-1}_{n_{ref}}(\vboldc)
 \end{array}
\label{eqn::coarseaverage}
\end{equation}

\section{Interpolation Operations}      

\subsection{Piecewise Linear Interpolation}
\label{sec::pwlinearinterp}
This method is primarily used to initialize fine grid data after
regridding. Given a level array $\varphi^c$ on $\Omega^c$, we want to 
compute  $I_{pwl}({\varphi})$ defined on an $\Omega^f$ properly nested
in $\Omega^c$.  
For the values on $C(\tilde{\Omega}^f)$, interpolate in a
piecewise-linear fashion in space, using the values
$\tilde{\varphi}^c$ (we assume that the coarse data already contains
the average of the fine data as discussed in the last section).
\begin{equation}
\begin{array}{c}
\varphi^f_{\vboldf} = \tilde{\varphi}^c_{\vboldc} +
\sum\limits^{D-1}_{d=0}(\frac{(\ind(\vboldf)_d + \half)}{n_{ref}} -
\ind(\vboldc) + \half))\Delta^d \cdot \varphi^c_{\vboldc} \\
\mbox{where } \vboldc \in \ind^{-1}(\tilde{\Omega}^f - \Omega^f) \\
\vboldc = \mathcal{C}_{n_{ref}}(\vboldf).
\label{eqn::uberpwlinterp}
\end{array}
\end{equation}
The slopes $\Delta^d$ are computed using minmod limiting as shown below:
\begin{equation}
\begin{array}{c}
\Delta^d W_{\vboldc} = \delta^{minmod}(W_{\vboldc})|
\delta^L(W_{\vboldc})| \delta^R(W_{\vboldc})| 0 \\
\delta^L(W_{\vboldc}) = W_{\vboldc} - (W^n_{\vbold <<-\ebold^d}) \\
\delta^R(W_{\vboldc}) =(W^n_{\vbold << \ebold^d}) - W_{\vboldc} \\
\end{array}
\end{equation}
\begin{equation}
\delta^{minmod} = \left\{ \begin{array}{ll} 
         min(|\delta_L|, |\delta_R|) \cdot sign(\delta_L +
\delta_R)  & \mbox{ if } \delta_L \cdot \delta_R > 0 \\
  ~0 & \mbox{ otherwise} 
\end{array}        \right\}
\label{eqn::deltaminmod}
\end{equation}
The shift operator (denoted by $<<$) 
is defined using a simple average of connected values.


\subsection{Piecewise-Linear Coarse-Fine Boundary Interpolation}

In the next algorithm, we use the same linear interpolant but we
also interpolate in time between levels of time.  We have
the solution on the coarser level of refinement at two time levels,
$t_{Cold}$ and $t_{Cnew}$.
We want to compute an extension $\tilde{\varphi}^f$ of $\varphi^f$ on
$\tilde{\Omega}^f = \mathcal{G}(\Omega^f, p) \cap \Gamma^f, p>0$ 
that exists at time level $t_{F}$ where $t_{Cold} < t_f < t_{Cnew}$.
We assume that $\mathcal{C}_r(\tilde{\Omega}^f) \cap \Gamma^c C
\Omega^c$. 
Extend $\varphi^{c, valid}$ to $\varphi^c$, defined on all of
$\ind^{-1}(\Omega^c)$.
\begin{equation}
\varphi^c_{\vboldc} = Av(\varphi^f, n_{ref})_{\vboldc}, 
\vboldc \in \ind^{-1} \mathcal{C}_{n_{ref}}(\Omega^f) 
\end{equation}
At both $t_{Cold}$ and $t_{Cnew}$, 
for the values on $\tilde{\Omega}^f - \Omega^f$ compute a
piecewise linear interpolant, using the values $\tilde{\varphi}^c$. 
\begin{equation}
\begin{array}{c}
\tilde{\varphi}^f_{\vboldf} = \tilde{\varphi}^f_{\vboldc} +
\sum\limits^{D-1}_{d=0}(\frac{(\ind(\vboldf)_d + \half)}{n_{ref}} -
(\ind(\vboldc) + \half))\Delta^d \cdot \varphi^c_{\vboldc} \\
\mbox{where } \vboldc \in \ind^{-1}(\tilde{\Omega}^f - \Omega^f), \\
\vboldc= \mathcal{C}_{n_{ref}}(\vboldf). 
\end{array}
\end{equation}
The slopes $\Delta^d$ are computed using minmod limiting as shown
in equation \ref{eqn::deltaminmod}.
We then interpolate  in time between the new and old interpolated
values.
\begin{equation}
{\varphi}^f_{\vboldf, t_F} =  \tilde{\varphi}^f_{\vboldf, t_{Cold}} 
+\frac{t_F - t_{Cold}}{t_{Cnew}-t_{Cold}}
(\tilde{\varphi}^f_{\vboldf, t_{Cnew}}-\tilde{\varphi}^f_{\vboldf, t_{Cold}})
\end{equation}
This process should produce an interpolated value which has
second-order error in both time and space.

\subsection{Quadratic Coarse-Fine Boundary Interpolation}

Away from locations where the embedded boundary crosses the
coarse-fine interface, we use the algorithm in  \cite{ChomboDesign}.

To proceed from here we need to define the corner ghost cells region.
Say we have two levels of refinement.  Define $\Omega^f$ as that region covered by
the finer level. Define $\Lambda^f$ to be the problem domain at the
finest refinement.  Define $G^d$ to be the grow operation that only grows a region in the
coordinate direction $d$.  Define
the  grow operator $G$ to be the operator which grows a region by one
cell in all coordinate directions.  In three dimensions, this is
$$
 G(\Omega, 1)  \equiv G^1(G^2(G^3(\Omega,1), 1), 1)
$$
The coarse-fine layer of ghost 
$\Omega^{cf}$
cells is defined to be 
$$
\Omega^{cf,f} = (G(\Omega^f, 1) - \Omega^f) \cap \Lambda^f
$$
The ghost cells which are not corners $\Omega^e$ can be
obtained by shifting $\Omega^f$ along coordinate directions:
$$
\Omega{e,f} = (\bigcup\limits^{D}_{d = 1} G^d(\Omega^f, 1) - \Omega^f)
\cap \Lambda^f
$$
The corner ghost cells $\Omega^{p}$ are defined to be 
$$
\Omega^p = \Omega^{cf} - \Omega^e
$$
Define $C$ to be the pointwise coarsening operation and $r$ to be the
refinement ratio.  We define  the coarse-fine interface set on the
coarse level $\Omega^{cf,c}$ to be the coarse cells which underly the
fine ghost cells.
$$
\Omega^{cf, c} = C(\Omega^{cf,f}, r)
$$
Because of proper nesting requirements, we claim that
$ \Omega^{e, c} = C(\Omega^{e,f}, r) $ and 
$ \Omega^{p, c} = C(\Omega^{c,f}, r) $ do not intersect and 
$\Omega^{cf, c} =\Omega^{e,c} \cap \Omega^{p,c}$.

We use the Johansen stencil for Dirichlet EB boundary conditions
away from the coarse-fine boundary.  When any VoF of the Johansen
stencil is within a ghost cell on the coarse-fine interface, we drop
to the least-squares stencil which has a smaller.   The least-squares
stencil still requires corner ghost cells be filled.
In the absence of coarse-fine interfaces intersecting embedded
boundaries,  we only need to fill ghost cells which were not corners
$\Omega^e$.    Since we are proposing to allow this intersection, we
must interpolate to all of $\Omega^{cf}$.

\subsubsection{Interpolation to non-corner ghost cells $\Omega^{e,f}$}

Away from points where the  coarse-fine interface, we interpolate
using QuadCFInterp.   It uses one-sided differences to avoid using
coarse data under finer data.  See the AMRTools section of the Chombo
design document for details.   Though that is somewhat ideologically
inconsistent with the following strategy near embedded boundaries, we
feel that, as a proven technology, QuadCFInterp should be left alone.


This section deals only with quadratic interpolation near where the
coarse-fine interface crosses the embedded boundary
design document.   The functional change from QuadCFInterp here is that we only
need to do one-sided difference when covered cells are near.   Because
we are doing higher-order averaging to fill coarse data that is under
finer data (see section \ref{sec::hoave}), 
we can allow the coarse-fine interpolation stencil to
reach under finer grids.

We present the natural extension of the regular grid description to
embedded boundaries of quadratic coarse-fine interpolation.   For each
coordinate direction $d$, we compute values of $\phi$ in the set 
$\Omega^{e,f}_d = (G^d(\Omega^f, 1) - \Omega^f) \cap \Lambda^f$.
Define the ``valid'' parts of the domain to be the parts of the domain
whose volume fractions are greater than zero.
$$
\Omega^{c,valid}_\ibold = \{ \ibold: \ibold \in \Omega^c \mbox{ and } \kappa_\ibold > 0 \}
$$

To perform this interpolation, we first observe that, given $\ibold
\in \tilde{\Omega}^f_k - \Omega^f$, there is a unique choice of $\pm$
and $d$, such that $\ibold \mp \ebold^d \in \Omega^f_k$. Having
specified that choice, the interpolant
is constructed in two steps 
\begin{trivlist}
\item 
(i) Interpolation in the direction orthogonal to $\ebold^d$. We
compute
\beqa
\xbold = \frac{\ibold + \half \ubold}{r} - (\ibold^c +
\half \ubold)
\eeqa
where $\ibold^c = {\mathcal{C}}_{r}(\ibold)$. 
The real-valued vector $\xbold$ is the displacement of the cell center
$\ibold$ on the fine grid from the cell center at $\ibold^c$ on the
coarse
grid, scaled by $h^c$.
\beqa
\hat{\varphi}_\ibold = \varphi^c_{\ibold^c} +
\sum_{d' \neq d}
\Bigl[
  \bigl(
    x_{d'}(D^{1, d'}\varphi^c)_{\ibold^c} +
    \half (x_{d'})^2(D^{2, d'}\varphi^c)_{\ibold^c}
  \bigr) + 
  \sum_{d'' \neq d, d'' \neq d'} x_{d'}x_{d''}(D^{d'd''}\varphi^c)_{\ibold^c}
\Bigr]
\eeqa
The second sum has only one term if $\Dim=3$, and no terms if $\Dim=2$.
\item
(ii) Interpolation in the normal direction.  
\begin{equation*}
\tilde{\varphi}_\ibold = I_q^B(\varphi^f,\varphi^{c,valid}) \equiv
4a + 2b + c ~ , ~ 
\tilde{x_d} = x_d - \half (r + 3)
1\end{equation*}
where $a,b,c$ are computed
to interpolate between the collinear data
\begin{align*}
&((\ibold \pm \half(n^l_{ref} -1)\ebold^d)h,
\hat{\varphi}_\ibold), \\ &((\ibold \mp \ebold^d)h, \varphi^l_{\ibold \mp
\ebold_d}), \\ &((\ibold \mp 2\ebold^d)h, \varphi^l_{\ibold \mp 2\ebold_d)}
\end{align*}
\end{trivlist}
In (i), the quantities $D^{1, d'}\varphi^c, D^{2, d'}\varphi^c$
and $D^{d'd''}\varphi^c$ are difference approximations to $
\frac{\partial}{\partial x_{d'}}, \frac{\partial^2}{\partial
x^2_{d'}}$, and $\frac{\partial^2}{\partial x_{d'}\partial x_{d''}}$,
respectively.  $D^{1, d}\varphi$ must be accurate to $O(h^2)$, while
the other two quantities need only be $O(h)$.  The
strategy for computing these quantities is to use only values in
$\Omega^c_{valid}$ to compute these difference approximations.  For
the case of $D^{1, d'}\varphi, D^{2, d'}\varphi$, we use 3-point
stencils, centered if possible, or shifted as required to consist of
points on $\Omega^c_{valid}$.

\beqa
(D^{1, d'}\varphi)_\ibold =
  \begin{cases}
    \frac{1}{2}(\varphi^c_{\ibold + \ebold^{d'}} -
                 \varphi^c_{\ibold - \ebold^{d'}}) &
      \text{if both $\ibold \pm \ebold^{d'} \in \Omega^c_{valid}$}
    \\
    \pm\frac{3}{2}(\varphi^c_{\ibold \pm  \ebold^{d'}} -
                    \varphi^c_\ibold                      )
    \mp\frac{1}{2}(\varphi^c_{\ibold \pm 2\ebold^{d'}} -
                    \varphi^c_{\ibold \pm  \ebold^{d'}}) &
      \text{if $\ibold \pm \ebold^{d'}     \in \Omega^c_{valid}$,
               $\ibold \mp \ebold^{d'} \not\in \Omega^c_{valid}$}
    \\
    0 &
      \text{otherwise}
  \end{cases}
\eeqa
\beqa
(D^{2, d'}\varphi)_\ibold =
  \begin{cases}
    \varphi^c_{\ibold+\ebold^{d'}} -
                 2\varphi^c_\ibold               +
                  \varphi^c_{\ibold-\ebold^{d'}} &
      \text{if both $\ibold \pm \ebold^{d'} \in \Omega^c_{valid}$}
    \\
    \varphi^c_\ibold                     -
                 2\varphi^c_{\ibold \pm   \ebold^{d'}} +
                  \varphi^c_{\ibold \pm 2 \ebold^{d'}} &
      \text{if $\ibold \pm \ebold^{d'}     \in \Omega^c_{valid}$,
               $\ibold \mp \ebold^{d'} \not\in \Omega^c_{valid}$}
    \\
    0 &
      \text{otherwise}
  \end{cases}
\eeqa

\begin{figure}[htp]
\epsfxsize=2.0in
\hspace{1.5in} \epsffile{../ChomboDesign/figs/mixed.eps}
\caption{Mixed-derivative approximation illustration.
        The upper-left corner is covered by a finer level
        so the mixed derivative in the upper left (the uncircled
        x) has a stencil which extends into the finer level.
        We therefore average the mixed derivatives centered
        on the other corners (the filled circles) to approximate
        the mixed derivatives for coarse-fine interpolation in three 
        dimensions.}
\label{MIXEDFIG}
\end{figure}

In the case of $D^{d'd''}\varphi^c$, we use an average of all of
the four-point difference approximations $\frac{\partial^2}{\partial
x_{d'}\partial x_{d''}}$ centered at $d', d''$ corners adjacent to
$\ibold$ such that all four points in the stencil are in $\Omega^c_{valid}$
(Figure \ref{MIXEDFIG})
\beqa
(D^{d'd''}_{corner}\varphi^c)_{\ibold + \half \ebold^{d'} + \half \ebold^{d''}} = 
  \begin{cases}
    \frac{1}{h^2}(\varphi_{\ibold+\ebold^{d'} + \ebold^{d''}} +
                  \varphi_\ibold -
                  \varphi_{\ibold + \ebold^{d'}} -
                  \varphi_{\ibold + \ebold^{d''}}) &
      \text{if $[\ibold, \ibold + \ebold^{d'} + \ebold^{d''}] \subset \Omega^c_{valid}$}
    \\
    0 &
      \text{otherwise}
  \end{cases}
\eeqa
\beqa
(D^{2, d'd''}\varphi^c)_\ibold = 
  \begin{cases}
    \frac{1}{N_{valid}} \sum_{s' = \pm 1} \sum_{ s'' = \pm 1}
      (D^{d'd''}\varphi^c)_{\ibold + \half s'\ebold^{d'} + \half s'' \ebold^{d''}} &
      \text{if $N_{valid} > 0$}
    \\
    0 &
      \text{otherwise}
  \end{cases}
\eeqa
where $N_{valid}$ is the number of nonzero summands.
To compute (ii), we need to compute the interpolation coefficients $a$
$b$, and $c$.

\begin{gather*}
a=\frac{\hat{\varphi} - 
		   (r\cdot |x_d| + 2) \varphi_{\ibold \mp \ebold^d} + 
		   (r\cdot |x_d| + 1) \varphi_{\ibold \mp 2 \ebold^d}}
		   {(r\cdot |x_d| + 2)(r\cdot |x_d| + 1)}
\\
b = \varphi_{\ibold \mp \ebold^d} - \varphi_{\ibold \mp 2 \ebold^d} - a
\\
c = \varphi_{\ibold \mp 2 \ebold^d}
\end{gather*}

\subsubsection{Interpolation to corner ghost cells}

We now discuss how we fill data on $\Omega^{p,f}$ the ghost cells over
the coarse fine interface which cannot be reached from a single move
in a coordinate direction.   Define  ${\cal D}$
to be the set of directions which have the requisite number of
uncovered, single-valued cells from a corner cell $\ibold$.
It is clear from the location of the corner which direction
one needs to extrapolate from.
$$
{\cal D}_\ibold = \{ d : \kappa_{\ibold \pm \ebold} > 0 \mbox{ and }
\kappa_{\ibold \pm 2\ebold} > 0\mbox{ and } \kappa_{\ibold \pm 3\ebold} > 0 \}
$$
We also exclude from ${\cal D}$ any directions with a multivalued cell
in the stencil.  We define
$N_D$ to be the number of directions contained in ${\cal D}$.   
$$
\phi_{\ibold} = \frac{1}{N_D} \sum\limits_{d \in {\cal D}}
3(\phi_{\ibold \pm \ebold^d} - \phi_{\ibold \pm 2\ebold^d}) +
\phi_{\ibold \pm 3\ebold^d} 
$$
We must exercise some care here to ensure
that our algorithm is independent of how we divide our region into
rectangles.    For this reason, after we do the above extrapolation,
we do a cornerCopier exchange operation to fill corner cells that
are covered by ghost cells of a neighboring fine grid.   Finally we do
an ordinary exchange operation to fill any ghost cells which are
covered by the valid fine grid.


%%%%%%%%%%%%5
\section{Redistribution}
\label{sec::levelredist}

To preserve stability and conservation in embedded boundary
calculations, we must redistribute a quantity of mass $\delta M$
(the difference between stable and conservative updates)
from irregular VoFs to their neighbors.  This mass is normalized
by $h^D$ where $h$ is the grid spacing on the level.  We define 
${\eta}_{\vbold}$ to be the set of neighbors (no farther away
than the redistribution radius) which can be reached by 
a monotonic path.  We then assign normalized weights to each of the
neighbors ${\vprime}$ and divide the mass accordingly:
\begin{equation}
\delta M_{\vbold} = \sum\limits_{\vprime \in {\eta}_{\vbold}} w_{\vbold, \prime}
        \kappa_{\vprime} \delta M_{\vbold}
\end{equation}
where 
\begin{equation}
\sum\limits_{\vprime \in {\eta}_{\vbold}} w_{\vbold, \vprime} \kappa_{\vprime} = 1
\end{equation}
We then update the solution $U$ at the neighboring cells ${\vprime}$
\begin{equation}
U^l_{\vprime} \pluseq  w_{\vbold,\vprime} \delta M^l_{\vbold}.
\label{eqn::levelredist}
\end{equation}
This operation occurs at all $\vbold \in \ind^{-1}(\Omega^l)$
without regard to valid or invalid regions.
If the irregular cell is within the redistribution radius of
a coarse-fine interface, we must account for mass that 
is redistributed across the interface.  

\subsection{Multilevel Redistribution Summary}

We begin with $\delta
M^l_\vbold, \vbold \in \ind^{-1} \Omega^l$, the redistribution
mass for level $l$. 

Define the redistribution radius to be $R^r$.  We define the 
coarsening operator to be $C_{\nref}$ and the refinement
operator to be $C^{-1}_{\nref}$.  We define the ``growth''
operator to be $G$.  The operator which produces the $Z^D$
index of a vof is $\ind$ and the operator to produces the 
VoFs for points in $Z^D$ is $\ind^{-1}$.  

If $\vbold$ is part of the valid region, 
the redistribution mass is divided into three parts,
\begin{equation}
\begin{array}{c}
 \delta M^l_\vbold = \delta^1 M^l_\vbold +  
\delta^2 M^{l,l+1}_\vbold +  \delta^2 M^{l,l-1}_\vbold, \\
\vbold \in \ind^{-1}(\Omega^{l, valid}).
\end{array}
\end{equation}
$\delta^1 M^l_\vbold$ is the part of the mass which is put onto the
$\Omega^{l, valid}$. $\delta^2 M^{l,l+1}_\vbold$ is the part of the
mass which is redistributed to $\Omega^l \cap C_{\nref}(\Omega^{l+1})$
(the part of the level covered by the next finer level).
$\delta M^{l,l-1}_\vbold$ is the part of the mass which is
redistributed off level $l$.  

If $\vbold$ is not part of the valid region, 
the redistribution mass is divided into two parts,
\begin{equation}
\begin{array}{c}
 \delta M^l_\vbold = \delta^I M^l_\vbold + \delta M^{l,l}_\vbold \\
\vbold \in \ind^{-1}(\Omega - \Omega^{l, valid}).
\end{array}
\end{equation}
$\delta^I M^l_\vbold$ is the portion of $\delta^l M^l_\vbold$ which
is redistributed to other invalid VoFs of level $l$.
$\delta^I M^P{l,l}\vbold$ is the portion of $\delta^l M^l_\vbold$ which
is redistributed to valid VoFs of level $l$ and must be removed
later from the solution.

We must account for $\delta M^{l,l-1}_\vbold$,
$\delta^2 M^{l,l+1}_\vbold$ and $\delta^3 M^{l,l}_\vbold$ 
to preserve conservation. $\delta^2 M^{l,l+1}_\vbold$ is 
added to the level $l+1$ solution.  $\delta^2 M^{l,l-1}_\vbold$ is 
added to the level $l-1$ solution. $\delta^3 M^{l,l}_\vbold$ 
is removed from the level $l$ solution.  

\subsection{Coarse to Fine Redistribution}

The mass going from coarse to fine is accounted for as follows.
Recall that the mass we store is normalized by $h^D_c$ where
$h_c$ is the grid spacing of the level of the source.  Define 
$h_f$ to be the grid spacing of the destination.
For all VoFs 
$\vboldc \in \ind^{-1}(C_{\nref}(G(\Omega^{l+1}, R^r) - \Omega^{l+1}))$,
we define the coarse-to-fine redistribution mass $\delta^2 M^{l,l+1}$
to be 
\begin{equation}
\begin{array}{c}
\delta^2 M^{l,l+1}_{\vboldc} = 
\sum\limits_{\vprimec \in S(\vboldc)}  \delta M^l_{\vboldc}
w_{\vboldc,\vprimec} \kappa_\vprimec
\\
S(\vboldc) = \eta_{\vboldc} \cap \ind^{-1}(C_{\nref}(\Omega^{l+1})).
\end{array}
\label{eqn::ctofredist}
\end{equation}
Define $\zeta^2_\vprimec$ to be the unnormalized mass that goes
to VoF $\vprimec$.   We distribute this mass to the VoFs $\vprimef$ that
cover $\vprimec$ ($\vprimef \in C^{-1}_\nref(\vprimec)$) in a
volume-weighted fashion.
\begin{equation}
\begin{array}{c}
\zeta^2_\vprimec = h^D_c w_{\vboldc,\vprimec} \kappa_\vprimec
  \delta M^l_{\vboldc} \\
\zeta^2_\vprimef = \frac{\kappa_\vprimef h^D_f}{\kappa_c h^D_c}\zeta^2_\vprimec \\
\zeta^2_\vprimef = \kappa_\vprimef h^D_f w_{\vboldc,\vprimec} 
\delta M^l_{\vboldc} 
\end{array}
\end{equation}
The change in the fine solution is the given by
\begin{equation}
\begin{array}{c}
\delta U^{l+1}_{\vprimef}=\frac{\zeta^2_\vprimef}{\kappa_\vprimef h^D_f}
 = \delta M^l_{\vboldc} w_{\vboldc,\vprimec} \\
U^{l+1}_{\vprimef} \pluseq  
\delta M^l_{\vboldc} w_{\vboldc,\vprimec} \\
\vboldc \in \ind^{-1}(C_{\nref}(G(\Omega^{l+1}, R^r) - \Omega^{l+1})) \\
\vprimec = \eta_{\vboldc} \cap \ind^{-1}(C_{\nref}(\Omega^{l+1})) \\
\vprimef \in C^{-1}_\nref(\vprimec) 
\end{array}
\end{equation}
This can be interpreted as a piecewise-constant interpolation of the
solution density.

\subsection{Fine to Coarse Redistribution}

The mass going from fine to coarse is accounted for as follows.
Recall that the mass we store is normalized by $h^D_f$ where
$h_f$ is the grid spacing of the level of the source.  Define 
$h_c$ to be the grid spacing of the destination.
For all VoFs 
$\vboldf \in \ind^{-1}(\Omega^{l}-G(\Omega^{l}, -R^r))$,
we define the fine-to-coarse redistribution mass $\delta^2 M^{l,l-1}$
to be 
\begin{equation}
\begin{array}{c}
\delta^2 M^{l,l-1}_{\vboldf} = 
\sum\limits_{\vprimef \in Q(\vboldf)}  \delta M^l_{\vboldf}
w_{\vboldf,\vprimef} \kappa_\vprimef
\\
Q(\vboldf) = \eta_{\vboldf} \cap 
\ind^{-1}(C^{-1}_{\nref}(\Omega^{l-1}) - \Omega^{l}).
\end{array}
\end{equation}
For all VoFs 
$\vboldc \in \ind^{-1}(C_{\nref}(G(\Omega^{l+1}, R^r) - \Omega^{l+1}))$,
we define the coarse-to-fine redistribution mass $\delta^2 M^{l,l+1}$
to be 
\begin{equation}
\begin{array}{c}
\delta^2 M^{l,l+1}_{\vboldc} = 
\sum\limits_{\vprimec \in S(\vboldc)}  \delta M^l_{\vboldc}
w_{\vboldc,\vprimec} \kappa_\vprimec
\\
S(\vboldc) = \eta_{\vboldc} \cap \ind^{-1}(C_{\nref}(\Omega^{l+1})).
\end{array}
\label{eqn::ctofredist2}
\end{equation}
Define $\zeta^2_\vprimef$ to be the unnormalized mass that goes
to VoF $\vprimef$.   We distribute this mass to the VoF 
$\vprimec = C_\nref(\vprimef)$.
\begin{equation}
\zeta^2_\vprimef = \zeta^2_\vprimec = h^D_f w_{\vboldf,\vprimef} \kappa_\vprimef
  \delta M^l_{\vboldf} 
\end{equation}
We define $\delta U^{l-1}_\vprimec$ to be the change in 
the coarse solution density
due to $\delta^w M_{\vboldf, \vprimef}$:
\begin{equation}
\begin{array}{c}
\delta U^{l-1}_{\vprimec} = 
\frac{\zeta^2_\vprimef}{\kappa_{\vprimec} h^D_c} 
\end{array}
\end{equation}
Substituting from above, we  increment 
the coarse solution as follows
\begin{equation}
\begin{array}{c}
U^{l-1}_{\vprimec} \pluseq  
\frac{\kappa_\vprimef}{\kappa_\vprimec N^D_{ref}}  
\delta M^l_{\vboldf} w_{\vboldf,\vprimef}\\
\vboldf \in \ind^{-1}(\Omega^{l}-G(\Omega^{l}, -R^r)),\\
\vprimef \in \eta_{\vboldf} \cap 
\ind^{-1}(C^{-1}_{\nref}(\Omega^{l-1}) - \Omega^{l}) \\
\vprimec = C_\nref(\vprimef)
\end{array}
\end{equation}

\subsection{Coarse to Coarse Redistribution}

The re-redistribution algorithm proceeds as follows.
Given $\vbold \in \ind^{-1}(C_\nref(\Omega^{l+1})$,
we define the re-redistribution mass $\delta^3 M{l,l}$ to be 
\begin{equation}
\begin{array}{c}
\delta^3 M^{l,l}_\vbold = 
\sum\limits_{\vprime \in T(\vbold)}  \delta M^l_{\vbold}
w_{\vbold,\vprime} \kappa_\vprime
\\
T(\vbold) = \eta_{\vbold} \cap \ind^{-1}(\Omega^l).
\end{array}
\label{eqn::doublesmush}
\end{equation}
In the level redistribution step, we have added this mass
to the solution density using equation \ref{eqn::levelredist}.
Re-redistribution is the process of removing it so
that the solution is not modified by invalid regions
\begin{equation}
\begin{array}{c}
U^{l}_{\vprime} \minuseq  
\delta M^l_{\vbold} w_{\vbold,\vprime} \\
\vbold \in \ind^{-1}(C_\nref(\Omega^{l+1}))
\end{array}
\end{equation}



\section{Refluxing}
\label{sec::reflux}
First we describe the refluxing algorithm which, along with
redistribution, preserves conservation at coarse-fine interfaces.
The standard refluxing algorithm
Given a level vector field $F$ on $\Omega$, we define a discrete
divergence operator $D$ as follows:
\begin{equation}
\begin{array}{c}
\kappa_\vbold(D \cdot \vec{F}) = \frac{1}{h}(
\sum\limits^{D-1}_{d=0} (
\sum\limits_{\fbold \in {\cal F}^+_d(\vbold)} \alpha_\fbold \tilde{F}_\fbold-
\sum\limits_{\fbold \in {\cal F}^-_d(\vbold)} \alpha_\fbold \tilde{F}_\fbold)
 + \alpha^B_\vbold F^B_\vbold) \\ 
\tilde{F}_\fbold = F_\fbold + \sum\limits_{d:d \neq dir(\fbold)}|x_{\fbold,
d}|(F_{\fbold<<sign(x_{\fbold, d})\ebold^d}-F_\fbold),
\end{array}
\label{eqn::discdiv}
\end{equation}
where $\kappa_\vbold$ is the volume fraction of VoF $\vbold$
and $\alpha_\fbold$ is the area fraction of face $\fbold$.
Equation \ref{eqn::discdiv}  consists of a summation of interpolated
fluxes and a boundary flux.  The flux interpolation is described
in \cite{JohansenColella1998}.
Let $\vec{F}^{comp} = \{ \vec{F}^f, \vec{F}^{c, valid} \}$ be a
two-level composite vector field. We want to define a composite
divergence $D^{comp}(\vec{F}^f, \vec{F}^{c, valid})_\vbold$, for
$\vbold \in V^c_{valid}$. We do this by extending $F^{c, valid}$ to
the faces adjacent to $\vbold \in V^c_{valid}$, but are covered by
$\mathcal{F}^f_{valid}$. 
\begin{equation}
\begin{array}{c}
<F^f_d>_{\fbold^c} = \frac{1}{(n_{ref})^{(\Dim-1)}}
\sum\limits_{\fbold \in \mathcal{C}^{-1}_{n_{ref}}
(\fbold^c)}\alpha_\fbold F^f_d \\ 
\fbold^c \in \ind^{-1}(\ibold + \half \ebold^d), \ibold + \half
\ebold^d \in \zeta^f_{d, +} \cup \zeta^f_{d, -} \\ 
\zeta^f_{d, \pm} = \{ \ibold \pm \half \ebold^d : \ibold \pm \ebold^d
\in \Omega^c_{valid}, \ibold \in \mathcal{C}_{n_{ref}}(\Omega^f) \} 
\end{array}
\end{equation}
Then we can define $(D \cdot \vec{F})_\vbold, \vbold \in
\mathcal{V}^c_{valid}$, using the expression above, with
$\tilde{F}_\fbold = <F^f_d>$ on faces covered by $\mathcal{F}^f$. 
We can express the composite divergence in terms of a level
divergence, plus a correction. We define a flux register $\delta
\vec{F}^f$, associated with the fine level 
\begin{equation}
\begin{array}{c}
\delta \vec{F}^f = (\delta F^f_{0, \ldots} \delta F^f_{D-1}) \\
\delta F^f_d : \ind^{-1}(\zeta^f_{d, +} \cup \zeta^f_{d,-}) \rightarrow
\mathbb{R}^m 
\end{array}
\end{equation}
If $\vec{F}^c$ is any coarse level vector field that extends
$\vec{F}^{c, valid}$, i.e. $F^c_d = F^{c, valid}_d \mbox{ on }
\mathcal{F}^{c,d}_{valid}$ then for $\vbold \in \mathcal{V}^c_{valid}$ 
\begin{equation}
 D^{comp}(\vec{F}^f, \vec{F}^{c, valid})_\vbold = (D \vec{F}^c)_\vbold
+ D_R(\delta \vec{F}^c)_\vbold 
\end{equation}
Here $\delta \vec{F}^f$ is a flux register, set to be
\begin{equation}
\delta F^f_d = <F^f_d> -  \alpha_{\fbold^c} F^c_d \mbox{   on } \ind^{-1}(\zeta^c_{d, +}
\cup \zeta^c_{d, -}) 
\end{equation}
$D_R$ is the reflux divergence operator. For valid coarse vofs
adjacent to $\Omega^f$ it is given by 
\begin{equation}
\kappa_\vbold(D_R \delta \vec{F}^f)_\vbold = 
\sum\limits^{D-1}_{d=0}(
 \sum\limits_{\fbold:\vbold=\vbold^+(\fbold)} \delta F^f_{d, \fbold} -
 \sum\limits_{\fbold:\vbold=\vbold^-(\fbold)} \delta F^f_{d, \fbold})
\end{equation}
For the remaining vofs in $\mathcal{V}^f_{valid}$, 
\begin{equation}
(D_R \delta \vec{F}^f) \equiv 0
\end{equation}
We then add the reflux divergence
to adjust the coarse solution $U^c$ to preserve conservation.
\begin{equation}
U^c_{\vbold} \pluseq  \kappa_{\vbold} (D_R(\delta F))_{\vbold}
\end{equation}
At coarse cells which are also irregular, this leaves unaccounted-for 
the quantity of mass $\delta M^{Ref}$ given by
\begin{equation}
\delta M^{Ref} = (1-\kappa_{\vbold})(D_R(\delta F))_{\vbold}
\end{equation}
This mass must be redistributed to preserve conservation:
\begin{equation}
\delta M^{Ref,c}_{\vbold} = 
\sum\limits_{{\vprime} \in {\eta}_{\vbold} - C({{\cal V}^{l, valid}})}
\kappa_{\vprime} w_{\vbold,\vprime} \delta M^{Ref, c}_{\vbold}
\label{eqn::redisteqn1}
\end{equation}
We increment the solution in the neighboring VoFs with their portion
of $\delta M^{Ref}$:
\begin{equation}
\begin{array}{c}
U^{c}_{\prime} \pluseq \kappa_{\vprime} 
w_{\vbold,\vprime} \delta M^{Ref,c}_{\vbold} \\
{\vprime} \in {\eta}_{\vbold} -  C({{\cal V}^{f, valid}})
\end{array}
\label{eqn::adddmref1}
\end{equation}
Time steps and other factors have  been absorbed into the 
definition of $\delta M$.   Unfortunately, we are not finished.
In equation \ref{eqn::adddmref1}, some of the mass will be going back onto
the fine grid
\begin{equation}
\delta M^{RR,c} \pluseq
\delta M^{Ref} \sum\limits_{{\vprime} \in {\eta}_{\vbold} - {{\cal V}^{c,
valid}}} 
 \kappa_{\vbold} w_{\vbold,\vprime} 
\end{equation}
This mass must be accumulated at each fine time step.  When 
the fine level has caught up with the coarse level in time,
we adjust the fine solution to account for this mass:
\begin{equation}
\begin{array}{c}
U^{f}_{C^{-1}(\vprime)} \pluseq 
w_{\vbold,\vprime} \delta M^{RR,c}_{\vbold} \\
{\vprime} \in {\eta}_{\vbold} - {{\cal V}^{f, valid}}
\end{array}
\label{eqn::adddmref2}
\end{equation}

\section{Subcycling in time with embedded boundaries}

We use the subcycling-in-time algorithm specified by
Berger and Oliger  \cite{bergerOliger:1984} to advance
an AMR solution in time.
Embedded boundary synchronization substantially complicates
Berger-Oliger timestepping.
Here we present an overview of
Berger-Oliger subcycling in time for adaptive mesh refinement in the
context of embedded boundaries.
Say we are solving the hyperbolic system of equations
\begin{equation}
\frac{\partial U}{\partial t} + \grad \cdot F = 0
\end{equation}
in a domain discretized as described above.
Here is an outline of the Berger-Oliger algorithm for this equation.
First we perform the steps required to preserve stability and
conservation in the presence of embedded boundaries.
\begin{itemize}
\item Compute fluxes $F^l$ on $ {\cal F}$.
\item Compute the conservative and non-conservative solution updates
   ($D^{C}(F^l)$ and $D^{NCC}(F^l)$).
\item Update the solution on the level:
\begin{equation}
  U^{new, l}_\vbold = U^{old, l}_\vbold - \dt(\kappa
  D^{NC}(F^l)_\vbold +  (1-\kappa)D^{C}(F^l)_\vbold),
  ~~\vbold \in \ind^{-1}(\Omega^l)
\end{equation}
\item Initialize redistribution mass $\delta M^l$ to be the mass left
out in the previous step.
\begin{equation}
\begin{array}{c}
\delta M^l_\vbold = \dt
\kappa_\vbold(1-\kappa_\vbold)(D^{NC}(F^l)_\vbold-D^{C}(F^l)_\vbold)
\\
 \vbold \in \ind^{-1}{\cal I}^l
\end{array}
\end{equation}
\item Perform level redistribution of $\delta M^l$:
\begin{equation}
\begin{array}{c}
U^{new,l}_{\vprime} \pluseq w_{\vbold, \vprime} \delta M^l_v \\
\vprime \in \{ {\eta}_\vbold \cap \ind^{-1}(\Omega^l) \} \\
\sum\limits_{\vprime \in {\eta}_v} w_{\vbold,\vprime}\kappa_{\vprime}=1 
\end{array}
\label{eqn::redisteqn2}
\end{equation}
\end{itemize}
Second we perform the steps required to preserve 
conservation across coarse-fine interfaces.  We define $\delta F$
to be flux registers and $\delta^2 M$ to be redistribution registers.
\begin{itemize}
\item We increment the flux register between this level and the next
      coarser level.
\begin{equation}
\begin{array}{c}
\delta F^{l,l-1}_\fbold \pluseq <F^l>_\fbold \dt^l \\
\fbold \in \partial(C({\cal F}^{l-1}))
\end{array}
\end{equation}
\item We initialize the flux register between this level and the next
finer level.
\begin{equation}
\begin{array}{c}
\delta F^{l+1,l}_\fbold = <F^l>_\fbold \dt^l \\
\fbold \in \partial({\cal F}^{l+1})
\end{array}
\label{eqn::reflux}
\end{equation}
\item Increment redistribution registers between this level
      and the next coarser level.
\begin{equation}
\begin{array}{c}
\delta^2 M^{l, l-1}_\vbold = \delta M^l_\vbold
\vbold \in \ind^{-1}({\cal I}^l) 
\end{array}
\end{equation}
\item Initialize redistribution registers with next finer level
      and the coarse-coarse (``re-redistribution'') registers.
      for $\vbold \in \ind^{-1}({\cal I})^l$
\begin{equation}
\begin{array}{c}
\delta^2 M^{l, l+1}_\vbold = \delta M^l_\vbold \\
\delta^2 M^{l, l}_\vbold =-\delta M^l_\vbold \\
\delta^2 M^{l+1, l}_\vbold = 0 
\end{array}
\label{eqn::mlredist1}
\end{equation}
\item Advance level $l+1$ solution to time $t^{new,l}$ (requires
a minimum of $n_{ref}$ time steps.
\item Reflux a portion of the flux difference
in equation \ref{eqn::reflux} and save the extra mass into the
appropriate redistribution register.
\begin{equation}
\begin{array}{c}
U^{new,l}_\vbold \pluseq \kappa D_R(\delta F^{l+1})_\vbold \\
\delta^2 M^{l, l+1}_\vbold \pluseq
\kappa_\vbold(1-\kappa_\vbold)D_R(\delta F^{l+1})_\vbold  \\
\delta^3 M^{l, l}_\vbold \pluseq
\kappa_\vbold(1-\kappa_\vbold)D_R(\delta F^{l+1})_\vbold  
\end{array}
\label{eqn::mlredist2}
\end{equation}
\item Redistribute mass that was redistributed (in both directions)
      across coarse-fine interfaces.
\begin{equation}
\begin{array}{c}
U^{l+1}_{\vprimef} \pluseq  
\delta^2 M^{l,l+1}_{\vboldc} w_{\vboldc,\vprimec} \\
\vboldc \in \ind^{-1}(C_{\nref}(G(\Omega^{l+1}, R^r) - \Omega^{l+1})) \\
\vprimec = \eta_{\vboldc} \cap \ind^{-1}(C_{\nref}(\Omega^{l+1})) \\
\vprimef \in C^{-1}_\nref(\vprimec) \\
\end{array}
\end{equation}
\begin{equation}
\begin{array}{c}
U^{l-1}_{\vprimec} \pluseq  
\frac{\kappa_\vprimef}{\kappa_\vprimec N^D_{ref}}  
\delta^2 M^{l,l-1}_{\vboldf} w_{\vboldf,\vprimef}\\
\vboldf \in \ind^{-1}(\Omega^{l}-G(\Omega^{l}, -R^r)),\\
\vprimef \in \eta_{\vboldf} \cap 
\ind^{-1}(C^{-1}_{\nref}(\Omega^{l-1}) - \Omega^{l}) \\
\vprimec = C_\nref(\vprimef)
\end{array}
\end{equation}

\item Re-redistribute mass that was redistributed from invalid 
        regions.
\begin{equation}
\begin{array}{c}
U^{l}_{\vprime} \minuseq  
\delta^3 M^{l,l}_{\vbold} w_{\vbold,\vprime} \\
\vbold \in \ind^{-1}(C_\nref(\Omega^{l+1}))
\end{array}
\end{equation}

\item Finally average down the finer solution where appropriate
\begin{equation}
U^{new,l}_\vbold = <U^{new,l+1}>, ~~\vbold \in \ind^{-1}C_{\nref}(\Omega^l+1)
\end{equation}

\end{itemize}

\subsubsection{$O(h^3)$ Averaging } 
\label{sec::hoave}

The stencil for Dirichlet EB boundary conditions on a coarse level can
reach under the fine level.
Because of this, we need to average $\phi$ from the finer level to the
coarser level before evaluating $L\phi$.
We use a higher-order ($O(h^3)$) averaging operator because we
need a more accurate value at a coarse location than averaging
the fine values which cover the coarse cell would produce.  Martin
and Cartwright discuss this in detail. 
The standard averaging operator is second order accurate and the
truncation error analysis works such that to avoid making $O(1)$
errors in the Laplacian on coarse cells near the fine grid, we need a third order
estimate of the solution on regions covered by finer grids.  We
therefore use a  modified averaging operator in which we eliminate
term of the truncation error of the standard averaging operator.
Consider a coarse cell at $\vec i_c$. 
The coarse cell is covered by fine cells and the refinement ratio is
two, the fine grid spacing is $h_f$ and the coarse grid spacing is 
$h_c$.  
Suppose we have a smooth function $\phi^e$ which exists at all
points in space.
Away from coarse-fine interfaces, the Laplacian is discretized in the
standard way.  In two dimensions, this discretization is:
\begin{equation}
(L\phi)_\ij = \frac{1}{h^2}
    (\phi_{i+1,j}+\phi_{i-1,j}+\phi_{i,j+1}+\phi_{i,j-1}-4\phi_\ij)
\label{DISCLAPL}
\end{equation}
and in three dimensions
\begin{equation}
(L\phi)_\ij = \frac{1}{h^2}
       (\phi_{i+1,j,k}+\phi_{i-1,j,k}+
        \phi_{i,j+1,k}+\phi_{i,j-1,k}+
        \phi_{i,j,k+1}+\phi_{i,j,k-1}-6\phi_\ijk)
\label{DISCLAPL3D}
\end{equation}
At the coarse-fine interface, we interpolate values onto ghost cells
which surround the union of rectangles that correspond to the level's
domain and use equation \ref{DISCLAPL} to calculate the Laplacian. 
We define the standard averaging operator $A_{S}$ in two 
dimensions to be
\begin{equation}        
(A_S(\phi^e))(h_c \vec i_c) = \frac{1}{4}\left( \begin{array}{ll}
\phi^e(h_f i_f   ,h_f j_f) + \\
\phi^e(h_f(i_f+1),h_f j_f) + \\ 
\phi^e(h_f i_f   ,h_f(j_f+1)) + \\
\phi^e(h_f(i_f+1),h_f(j_f+1))
\end{array} \right)
\end{equation}
and in three dimensions to be
\begin{equation}        
(A_S(\phi^e))(h_c \vec i_c) = \frac{1}{8}\left( \begin{array}{ll}
\phi^e(h_f i_f   ,h_f j_f, h_f k_f) + \\
\phi^e(h_f(i_f+1),h_f j_f, h_f k_f) + \\ 
\phi^e(h_f i_f   ,h_f(j_f+1,), h_f k_f) + \\
\phi^e(h_f(i_f+1),h_f(j_f+1), h_f k_f)  + \\
\phi^e(h_f i_f   ,h_f j_f, h_f(k_f+1)) + \\
\phi^e(h_f(i_f+1),h_f j_f, h_f(k_f+1)) + \\ 
\phi^e(h_f i_f   ,h_f(j_f+1), h_f(k_f+1)) + \\
\phi^e(h_f(i_f+1),h_f(j_f+1), h_f(k_f+1))
\end{array} \right)
\end{equation}
where $\vec i_f = 2 \vec i_c$.
The truncation error $\tau$ of $A_S$ is given by
\begin{equation}
\tau = \phi^e(h_c \vec i_c)-(A_S(\phi^e))(h_c \vec i_c) =
\frac{h^2_f}{2} \grad^2 \phi_e(h_c \vec i_c) + O(h^3_f)
\end{equation}
Away from the embedded boundary,
we define the modified averaging operator $A_M$ to be
$A_S$ with the leading order in the truncation error subtracted off:
\begin{equation}
(A_M(\phi_f))_{\vec i_c} = 
        A_S(\phi_f)_{\vec i_c} - \frac{h^2_f}{2} L(\phi_f)_{\vec i_c}
\label{MODAVERAGE}        
\end{equation}
Near the embedded boundary, we extrapolate to $O(h^2)$ from fine cells
to the coarse cell and average the result.

