This section describes the numerical method for integrating systems of
conservation laws (e.g., the Euler equations of gas dynamics) on
an AMR grid hierarchy.  This is done using an unsplit, second-order Godunov
method.

\section{Notation}

Most of the notation used here is introduced in the Chombo design document
\cite{ChomboDesign}.  The main exception to that is a notation using $|$
symbols.  For computations at cell centers the notation
\beqa
CC = A ~ | ~ B ~ | ~ C
\eeqa
means that
the 3-point formula $A$ is used for $CC$ if all cell centered values it uses
are available,
the 2-point formula $B$ is used if current cell borders the high side
of the physical domain (i.e., no high side value), and
the 2-point formula $C$ is used if current cell borders the low side
of the physical domain (i.e., no low side value).
For computations at face centers the analogous notation
\beqa
FC = A ~ | ~ B ~ | ~ C
\eeqa
means that
the 2-point formula $A$ is used for $FC$ if all cell centered values it uses
are available,
the 1-point formula $B$ is used if current face coincides with the high side
of the physical domain (i.e., no high side value), and
the 1-point formula $C$ is used if current face coincided with the low side
of the physical domain (i.e., no low side value).

\section{Multidimensional higher-order Godunov method
\label{sec:algorithm}}

The methods developed here have their origins in Colella \cite{COLELLA1} and
Saltzman \cite{Saltzman1}.
We are solving a hyperbolic system of equations of the form
\begin{gather}
\frac{\partial U}{\partial t}
+ \sum^{\Dim-1}_{d=0} \frac{\partial F^d}{\partial x^d}
= S
\label{eqn:hyperCons}
\end{gather}
We also assume there may be a change of variables $W = W(U)$ ($W \equiv
$ ``primitive variables'') that can be applied to simplify the
calculation of the characteristic structure of the equations.
This leads to a similar system of equations in $W$.
\begin{gather}
\frac{\partial W}{\partial t}
+ \sum^{\Dim-1}_{d=0} A^d(W) ~ \frac{\partial W^d}{\partial x^d}
= S' \\
A^d = \nabla_U W \cdot \nabla_U F^d \cdot \nabla_W U \notag \\
S' = \nabla_U W \cdot S \notag
\label{eqn:hyperPrim}
\end{gather}
Note, this system is not in conservation form as the primitive variables, in
general, are not conserved quantities.

Further note: The algorithm (and the software implementation) use the
source term, $S$ and/or $S'$, to compute accurate fluxes which are used to
advance $U$ in time due to the hyperbolic portion of (\ref{eqn:hyperCons}).
$U$ still needs to be updated to incorporate these source terms.
Specifically, solving:
\begin{gather*}
\frac{\partial U}{\partial t} = S
\end{gather*}
is left to the user.  This may be solved directly/explicitly or using
indirect/implicit methods but it must be included in the overall algorithm.

\subsection{Outline}

Given $U^n_\ibold$ and $S^n_\ibold$, we want to compute a second-order accurate
estimate of the fluxes: $F^{n+\half}_{\ibold+\half \ebold^d} \approx
F^d(\xbold_0 +(\ibold+\half \ebold^d)h,t^n + \half \Delta t)$.
The transformations $\nabla_U W $ and $\nabla_W U$ are functions of both
space and time.  We shall leave the precise centering of these transformations
vague as this will be application dependent.
In outline, the method is given as follows.
\begin{enumerate}
\item Transform to primitive variables, and compute slopes (the definition of
$\Delta^d W_\ibold$ is given in section \ref{sec:slopeCalculation}):
\begin{gather*}
\text{Given} ~ W^n_\ibold = W(U^n_\ibold),
~ \text{compute} ~ \Delta^d W_\ibold,
~ \text{for} ~ 0 \le d < \Dim
\end{gather*}

\item Compute the effect of the normal derivative terms and the source
term on the extrapolation in space and time from cell centers to
faces.  For $0 \le d < \Dim$,
% \begin{gather}
\begin{alignat}{1}
W_{\ibold, \pm, d}
= & \; W^n_\ibold +
       \half (\pm I - \frac{\Delta t}{h} A^d_\ibold) P_\pm (\Delta^d W_\ibold) 
       \label{eqn:normalPred} \\
A^d_\ibold
= & \; A^d(W_{\ibold}) \notag \\
P_\pm(W)
= & \; \sum_{\pm \lambda_k > 0} (l_k \cdot W) r_k  \notag \\
W_{\ibold, \pm, d}
= & \; W_{\ibold, \pm, d} + \frac{\Delta t}{2}
       \grad_U W \cdot S^n_\ibold \label{eqn:sourcePred}
\end{alignat}
% \end{gather}
where $\lambda_k$ are eigenvalues of $A^d_\ibold$, and 
$l_k$ and $r_k$ are the corresponding left and right eigenvectors.

\item Compute estimates of $F^d$ suitable for computing 1D flux
derivatives $\frac{\partial F^d}{\partial x^d}$ using a Riemann solver
for the interior, $R$, and for the boundary, $R_B$. 
Here, and in what follows, $\grad_U W$ need only be first-order
accurate, e.g., differ from the value at $U^n_\ibold$ by $O(h)$.
\begin{equation} \label{eqn:riemann1}
\begin{split}
F^{\text{1D}}_{\ibold + \half \ebold^d} =
    \; & R  (W_{\ibold, +, d}, W_{\ibold + \ebold^d, -, d},           d) \\
| \,\; & R_B(W_{\ibold,            +, d}, (\ibold + \half \ebold^d)h, d) \\
| \,\; & R_B(W_{\ibold + \ebold^d, -, d}, (\ibold + \half \ebold^d)h, d)
\end{split}
\end{equation}

\item In 3D compute corrections to $W_{\ibold, \pm, d}$ corresponding to one
set of transverse derivatives appropriate to obtain $(1, 1, 1)$
diagonal coupling.  In 2D skip this step.
\begin{equation} \label{eqn:updateprim1}
W_{\ibold, \pm, d_1, d_2}
=  \; W_{\ibold, \pm, d_1}
    - \frac{\Delta t}{3h} \nabla_U W \cdot
      (F^{\text{1D}}_{\ibold + \half \ebold^{d_2}}
     - F^{\text{1D}}_{\ibold - \half \ebold^{d_2}})
\end{equation}

\item In 3D compute fluxes corresponding to corrections made in the previous
step.  In 2D skip this step.
\begin{equation} \label{eqn:riemann2}
\begin{split} 
F_{\ibold + \half \ebold^{d_1}, d_2}
=   \; & R(W_{\ibold, +, d_1, d_2}, 
           W_{\ibold + \ebold^{d_1}, -, d_1, d_2}, d_1) \\
| \,\; & R_B(W_{\ibold,                +, d_1, d_2}, 
         (\ibold + \half \ebold^{d_1})h, d_1) \\
| \,\; & R_B(W_{\ibold + \ebold^{d_1}, -, d_1, d_2}, 
         (\ibold + \half \ebold^{d_1})h, d_1)
\end{split}
\end{equation}
\begin{equation*}
d_1 \neq d_2, ~ 0 \le d_1, d_2 < \Dim
\end{equation*}

\item Compute final corrections to $W_{\ibold, \pm, d}$ due to
the final transverse derivatives.
\begin{alignat}{2}
\label{eqn:updateprim2-2D}
\text{2D:}~~~~
 W^{n+\half}_{\ibold, \pm, d} = \; & 
    W_{\ibold, \pm, d} \; & & - \; \frac{\Delta t}{2h} \nabla_U W \cdot
 (F^{\text{1D}}_{\ibold + \half \ebold^{d_1}}
                       - F^{\text{1D}}_{\ibold - \half \ebold^{d_1}}) \\
 & & & d \neq d_1, ~ 0 \le d, d_1 < \Dim \notag \\
\label{eqn:updateprim2-3D}
\text{3D:}~~~~
 W^{n+\half}_{\ibold, \pm, d} = \; & W_{\ibold, \pm, d}
 \; & & - \; \frac{\Delta t}{2h}\nabla_U W \cdot
                         (F_{\ibold + \half \ebold^{d_1}, d_2}
                        - F_{\ibold - \half \ebold^{d_1}, d_2}) \\
 & \; & & - \; \frac{\Delta t}{2h} \nabla_U W \cdot
                        (F_{\ibold + \half \ebold^{d_2}, d_1}
                       - F_{\ibold - \half \ebold^{d_2}, d_1}) \notag \\
 & & & d \neq d_1 \neq d_2, ~ 0 \le d, d_1, d_2 < \Dim \notag
\end{alignat}

\item Compute final estimate of fluxes.
\begin{equation} \label{eqn:riemann3}
\begin{split}
F^{n+\half}_{\ibold + \half \ebold^d} =
    \; & R  (W^{n+\half}_{\ibold, +, d},
             W^{n+\half}_{\ibold + \ebold^d, -, d}, d) \\ 
| \,\; & R_B(W^{n+\half}_{\ibold,            +, d},
             (\ibold + \half \ebold^d)h, d) \\ 
| \,\; & R_B(W^{n+\half}_{\ibold + \ebold^d, -, d},
             (\ibold + \half \ebold^d)h, d)
\end{split}
\end{equation}

\item Update the solution using the divergence of the fluxes.
\begin{equation} \label{eqn:updateCons}
U^{n+1}_\ibold = U^n_{\ibold} - \frac{\Delta t}{h}\sum\limits^{\Dim - 1}_{d=0} 
( F^{n+\half}_{\ibold + \half \ebold^d} -
  F^{n+\half}_{\ibold - \half \ebold^d} )
\end{equation}
\end{enumerate}

\subsection{Slope Calculation} \label{sec:slopeCalculation}

We will use the 4th order slope calculation in Colella and Glaz
\cite{ColellaGlaz} combined with characteristic limiting.
\begin{alignat*}{1}
\Delta^d W_\ibold
= & \; \zeta_\ibold ~ \delta^{vL}(\Delta^d_4 W_\ibold,
                                  \Delta^d_- W_\ibold,
                                  \Delta^d_+ W_\ibold)
                ~ | ~ \Delta^d_2 W_\ibold
                ~ | ~ \Delta^d_2 W_\ibold \\
\Delta^d_4 W_\ibold
= & \; \frac{2}{3} ((W - \frac{1}{4}\Delta^d_2 W)_{\ibold + \ebold^d} -
                    (W + \frac{1}{4}\Delta^d_2 W)_{\ibold - \ebold^d}) \\
\Delta^d_2 W_\ibold
= & \; \delta^{vL}(\widetilde{\Delta}^d_2 W_\ibold,
                   \Delta^d_- W_\ibold,
                   \Delta^d_+ W_\ibold)
 ~ | ~ \Delta^d_- W_\ibold
 ~ | ~ \Delta^d_+ W_\ibold \\
\widetilde{\Delta}^d_2 W_\ibold
= & \; \half(W^n_{\ibold + \ebold^d} - W^n_{\ibold - \ebold^d}) \\
\Delta^d_- W_\ibold
= & \; W^n_\ibold - W^n_{\ibold - \ebold^d} ~ , ~
\Delta^d_+ W_\ibold
=      W^n_{\ibold + \ebold^d} - W^n_\ibold
\end{alignat*}
At domain boundaries, $\Delta^d_- W_\ibold$ and $\Delta^d_+ W_\ibold$
may be overwritten by the application to provide application dependent
slopes at the boundaries (see section \ref{sec:physIBC}).
There are two versions of the van Leer limiter 
$\delta^{vL}(\delta W_C, \delta W_L, \delta W_R)$ that are commonly used. 
One is to apply a limiter to the differences in characteristic
variables.
\begin{enumerate}

\item Compute expansion of one-sided and centered differences in
characteristic variables.
\begin{align*}
\alpha^k_C &= l^k \cdot \delta W_C \\
\alpha^k_L &= l^k \cdot \delta W_L \\
\alpha^k_R &= l^k \cdot \delta W_R
\end{align*}

\item Apply van Leer limiter
\begin{align*}
\alpha^k =
  \begin{cases}
  \min \{  | \, \alpha^k_C \, |, \,
      2 | \, \alpha^k_L \, |, \,
      2 | \, \alpha^k_R \, | \} &
    \text{if $\alpha^k_L \cdot \alpha^k_R > 0$;}
  \\
  0 &
    \text{otherwise.}
  \end{cases}
\end{align*}

\item $\delta^{vL} = \sum_k \alpha^k r^k$
\end{enumerate}

\noindent
Here, $l^k= l^k(W^n_\ibold)$ and $r^k = r^k(W^n_\ibold)$.

For a variety of problems, it suffices to apply the van Leer limiter 
component-wise to the differences. Formally,
this can be obtained from the more general
case above by taking the matrices of left and right eigenvectors to be
the identity.

Finally, we give the algorithm for computing the flattening coefficient 
$\zeta_\ibold$. We assume that there is a quantity corresponding to the
pressure in gas dynamics (denoted here as $p$) 
which can act as a steepness indicator, and a
quantity corresponding to the bulk modulus (denoted here as $K$, given
as $\gamma p$ in a gas), that can be used to non-dimensionalize
differences in $p$.
\begin{alignat}{1}
\label{eqn:flattening}
\zeta_\ibold
= & \,
  \begin{cases}
  ~ \underset{0 \le d < \Dim}{\min} \, \zeta^d_\ibold ~ &
    \text{if $\sum^{\Dim - 1}_{d=0} \Delta^d_1 u^d_\ibold < 0$}
  \\
  ~ 1 &
    \text{otherwise}
  \end{cases} \\
\zeta^d_\ibold
= & \; {\min}_3(\widetilde{\zeta}^d,d)_\ibold \notag \\
\widetilde{\zeta}^d_\ibold
= & \; \eta(\Delta^d_1 p_\ibold, \,
            \Delta^d_2 p_\ibold, \,
            {\min}_3(K,d)_\ibold) \notag \\
\Delta^d_1 p_\ibold
= & \; \half(p_{\ibold + \ebold^d} - p_{\ibold - \ebold^d})
   ~ | ~     p_\ibold              - p_{\ibold - \ebold^d}
   ~ | ~     p_{\ibold + \ebold^d} - p_\ibold \notag \\
\Delta^d_2 p_\ibold
= & \; (\Delta^d_1 p_{\ibold + \ebold^d} + \Delta^d_1 p_{\ibold - \ebold^d})
 ~ | ~ 2 \Delta^d_1 p_\ibold
 ~ | ~ 2 \Delta^d_1 p_\ibold \notag
\end{alignat}
The functions ${\min}_3$ and $\eta$ are given below:
\begin{gather*}
\begin{split}
{\min}_3(K,d)_\ibold
= & \; \min(K_{\ibold + \ebold^d},K_\ibold,K_{\ibold - \ebold^d})
~|~ \min(K_\ibold,K_{\ibold - \ebold^d})
~|~ \min(K_{\ibold + \ebold^d},K_\ibold) ; \\
\eta(\delta p_1, \delta p_2, p_0)
= & \,
  \begin{cases}
  0 &
    \text{if $\frac{| \delta p_1 |}{p_0} > d$
         and $\frac{| \delta p_1 |}{| \delta p_2 |} > r_1$ ;} \\
  1 - \frac{\frac{| \delta p_1 |}{| \delta p_2 |} - r_0}{r_1 - r_0} &
    \text{if $\frac{| \delta p_1 |}{p_0} > d$
         and $r_1 \ge \frac{| \delta p_1 |}{| \delta p_2 |} > r_0$ ;} \\
  1 &
    \text{otherwise.}
  \end{cases}
\\
& r_0 = 0.75, ~ r_1 = 0.85, ~ d = 0.33 .
\end{split}
\end{gather*}

\section{Artificial Viscosity}

We add a small $O(h^2)$ diffusive term to the flux prior to the final
conservative difference step.  This "artificial viscosity" term serves to
suppress instabilities occurring in multidimensional shocks that are nearly
aligned with one of the coordinate directions; for a detailed discussion,
see \cite{colellaWoodward:1984} and \cite{woodwardColella:1984}.

\begin{align*}
F^{\eta+\frac{1}{2}}_{\ibold+\frac{1}{2}\ebold_d} =& ~
    F^{\eta+\frac{1}{2}}_{\ibold+\frac{1}{2}\ebold_d}
  - K_{\ibold+\frac{1}{2}\ebold^d}(U^n_{\ibold+\ebold^d}-U^\eta_\ibold)
\\
K_{\ibold+\frac{1}{2}\ebold^d} =& ~
    K_0~\max(-(D\vec{u})_{\ibold+\frac{1}{2}\ebold^d},0)
\\
\begin{split}
(D\vec{u})_{\ibold+\frac{1}{2}\ebold^d} =& ~
    (u^d_{\ibold+\ebold^d}-u^d_\ibold) +
\\
  & ~ \sum_{d'\neq d}\frac{1}{4}((\Delta^{d'}_+u^{d'})_\ibold
  + (\Delta^{d'}_-u^{d'})_\ibold
  + (\Delta^{d'}_+u^{d'})_{\ibold+\ebold^d}
  + (\Delta^{d'}_-u^{d'})_{\ibold+\ebold^d})
\end{split}
\end{align*}
For typical time-dependent calculations of shocks in gases, $K_0=0.1$.

\section{Extension to PPM}

We can extend this algorithm to the case of using the
piecewise-parabolic method of Colella and Woodward
\cite{colellaWoodward:1984} to perform the normal predictor step
\cite{millerColella:2002}. We begin by computing 
spatially extrapolated face-centered values at the low and high edges 
of the cells.
\begin{gather*}
W_\pm = \half (W^n_{\ibold \pm \ebold} + W^n_{\ibold})
                      \pm \frac{1}{6} 
		      (\Delta^d_2 W_{\ibold} -
		       \Delta^d_2 W_{\ibold \pm \ebold} ) 
		      \ | \  W^n_{\ibold} - \half \Delta^d_2 W_{\ibold}
		      \ | \  W^n_{\ibold} + \half \Delta^d_2 W_{\ibold}
\\
\alpha^k_\pm = l^k \cdot (W_\pm - W^n_\ibold)
\end{gather*}

The van Leer slopes $\Delta^d_2 W$ can be limited component-wise, or 
by using limiting in characteristic variables. Similarly, 
there are two options for limiting the parabolic profile. One is
to apply the PPM limiter to the characteristic variables $\alpha^k_\pm$: if 
$\alpha^k_+ \alpha^k_- < 0$, then
\begin{alignat*}{2}
\alpha^k_+ ~:=~ &
    s \cdot \min \{ s \cdot \alpha^k_+, -2 s \cdot \alpha^k_- \} &
    & ~~~ \hbox{ if } (\alpha^k_+)^2 > (\alpha^k_-)^2 ; \\
\alpha^k_- ~:=~ &
    s \cdot \min \{ s \cdot \alpha^k_-, -2 s \cdot \alpha^k_+ \} &
    & ~~~ \hbox{ otherwise. } 
\end{alignat*}
where $s = sign(\alpha^k_+ - \alpha^k_-)$. If $\alpha^k_+ \alpha^k_- \geq
0$, then we set $\alpha^k_+ , \alpha^k_- := 0$. An alternative approach is to
apply the limiter above component-wise to the differences 
$W_\pm - W^n_\ibold$, and then compute the characteristic
amplitudes $\alpha^k_\pm$. If appropriate, we also apply the flattening
coefficients (\ref{eqn:flattening}) to the parabolic profiles after the
limiting for monotonicity has been applied:
$\alpha^k_\pm := \alpha^k_\pm \cdot \zeta_\ibold$.

Finally, we use the PPM predictor to compute the normal predictor
corresponding to (\ref{eqn:normalPred}).
\begin{align*}
W_{\ibold, \pm, d} = \, &
    W^n_{\ibold} + \sum_k (\alpha^k_\pm + \half \sigma^k_{\pm} 
    (\pm (\alpha^k_{-} - \alpha^k_{+}) - (\alpha^k_{-} + \alpha^k_{+})
    (3 - 2 \sigma^k_\pm)) \cdot r^k \\
\sigma^k_\pm = \, &
  \begin{cases}
    \pm \lambda^k_d(W^n_{\ibold}) \frac{\Delta t}{\Delta x}
    \space
  & \hbox{ if } \space \pm \lambda^k_d(W^n_{\ibold}) > 0 \\
    \max \{ \pm \lambda^{\pm}(W^n_{\ibold}), 0 \}
    \frac{\Delta t}{\Delta x}
  & \hbox{ otherwise. }
  \end{cases}
\end{align*}
Here $\lambda^{\{+,-\}}$ is the \{maximum , minimum\} of the wave speeds
over all of the wave families.

%% \subsection{Currently Omitted Details}
%% There are a number of items which have not been discussed which, nonetheless,
%% are part of the computation and update of the conserved quantities, $U$.
%% Specifically:
%% \begin{enumerate}
%% \item What domains/boxes over which computations take place, how many ghost
%%       cells are needed, etc?
%% \item How is artificial viscosity used and implemented?
%% \end{enumerate}
%% This section is intended to be a reminder of these issues and eventually it
%% should be removed when all the issues are addressed.

\section{Recursive AMR Update}

We extend this method to an adaptive mesh hierarchy using the
Berger--Oliger algorithm.  We define
\beqa
\{ U^l \}^{l_{\max}}_{l=0}, U^l : \Omega^l \rightarrow {\mathbb{R}}^m
\eeqa
$U^l = U^l(t^l)$.  Here $\{ t^l \}$ are a collection of discrete times
that satisfy the temporal analogue of proper nesting.  $ \{ t^l \} =
\{ t^{l-1} + k \Delta t^l : 0 \leq k < n^l_{ref} \}$  The
algorithm in \cite{bergerColella:1989} 
for advancing the solution in time is given in
pseudo-code in figure \ref{fig:HSCLcode}.
The discrete fluxes $\vec{F}$ are computed
by using piecewise linear interpolation to define an extended solution
on:
\begin{align*}
\tilde{\Omega} = \,\, & {\mathcal{G}}(\Omega^l, p) \cap \Gamma^l 
\mbox{ , } \tilde{U} : \tilde{\Omega} \rightarrow {\mathbb{R}}^m \\
\tilde{U}_{\ibold} = \, &
  \begin{cases}
    U^l_\ibold (t^l) &
      \text{for $\ibold \in \Omega^l$}
    \\ 
    I_{pwl} ((1 - \alpha)  U^{l-1}(t^{l-1}                 ) +
                  \alpha ~ U^{l-1}(t^{l-1} + \Delta t^{l-1}))_\ibold &
      \text{otherwise}
  \end{cases} \\
\alpha = \,\, & \frac{t^l - t^{l-1}}{\Delta t^{l-1}}
\end{align*}
and then computing fluxes for the advance as outlined in Section
\ref{sec:algorithm}. 

% The regridding step is performed in these steps.  First, one
% constructs ${\mathcal{I}}^l \subset \Omega^l, l = l_{base}, ..., l_{\max} -1$
% corresponding to those cells for which a user-specified  measure of
% the error exceeds a specified tolerance.  Second, one generates new
% grids $\Omega^{l, new}, l = l_{base} +1, ..., l_{\max}$ on which the
% new solution be defined.  These new grids should satisfy
% ${\mathcal{C}}_{n^l_{ref}}(\Omega^{l+1, new}) \supset {\mathcal{I}}^l$, and should
% be properly nested, as well as satisfying any other required nesting
% conditions.  This imposes some constraints on ${\mathcal{I}}^l$
% if $l_{base} > 0$.  These constraints are met typically by reducing
% the size of the ${\mathcal{I}}^l$'s prior to the grid generation step.  Finally,
% one initializes the new data $U^{l, new} (t^l)$.  For $l = l_{base} ,
% U^l = U^{l, new}$.
% For $l = l_{base} +1, ..., l_{\max}$,
% \begin{equation*}
% U^{l, new} (t_{\rm regrid})_\ibold =
%   \begin{cases}
%     U^{l, new} (t_{\rm regrid})_\ibold = U^l(t_{\rm regrid})_\ibold &
%       \ibold \in \Omega^l \cap \Omega^{l, new}
%     \\
%     I_{pwl} (U^{l-1, new} (t_{\rm regrid}))_\ibold &
%       \text{otherwise.}
%   \end{cases}
% \end{equation*}

\begin{figure}[thp]
\hrulefill
\begin{tabbing} 
xxxx\=xxxx\=xxxx\=xxxx\=\kill
\>procedure advance $(l)$ \\
\>$U^l(t^l + \Delta t^l) = U^l(t^l) - \Delta t D \vec{F}^l \mbox{ on }
\Omega^l$ \\
\>if $l<l_{\max}$ \\
\>\>$\delta F^{l+1}_d = -F^l_d \mbox{ on }
{\zeta}^{l+1}_{+,d} \cup {\zeta}^{l+1}_{-,d} , d=0, ..., \Dim-1$ \\
\> end if \\
\>if $l>0$ \\
\>\> $\delta F^l_d := \delta F^l_d + \frac{1}{n^{l-1}_{ref}}\langle F^l_d \rangle \mbox{ on }
{\zeta}^l_{+,d} \cup {\zeta}^l_{-,d} , d=0, ..., \Dim-1$  \\
\> end if \\
\>for $q = 0, ..., n^l_{ref} -1$ \\
\>\> advance$(l+1)$ \\
\> end for \\
\>$U^l(t^l + \Delta t^l) = Average(U^{l+1}(t^l + \Delta t^l), n^l_{ref})
\mbox{ on } {\mathcal{C}}_{n^l_{ref}} (\Omega^{l+1})$ \\
\>$U^l(t^l + \Delta t^l) := U^l(t^l + \Delta t^l) - \Delta t^l D_R
(\delta F^{l+1})$ \\
\>$t^l := t^l + \Delta t^l$ \\
\>$n^l_{\rm step} := n^l_{\rm step}+ 1$ \\
\>if $(n^l_{\rm step} = 0 \mod n_{\rm regrid}) \mbox{ and } (n^{l-1}_{\rm step} \neq 0 \mod n_{\rm regrid}) $\\
\>\>regrid($l$) \\
\> end if
\end{tabbing}
\hrulefill
\caption{Pseudo-code description of the Berger--Colella AMR algorithm for 
hyperbolic conservation laws.}
\label{fig:HSCLcode}
\end{figure}


