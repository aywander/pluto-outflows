% -*- Mode: TeX; Modified: "Mon 05 Apr 1999 22:39:15 by dbs"; -*- 

\bibliographystyle{plain}

\oddsidemargin=.25in
\evensidemargin=.25in
\textwidth=6.0in
\topmargin=-0.5in
\textheight=8.5in

%%% `captions.sty' overrides the default figure captions layout.
%%% These variables are used in this style file.
\setlength{\abovecaptionskip}{10pt}
\setlength{\belowcaptionskip}{0pt}

%%% force \subsubsection to be numbered and appear in the table of contents
%%% in report style
\setcounter{secnumdepth}{3}
\setcounter{tocdepth}{3}

%%% In `article' style, remove the chapter number from section numbers and force a
%%% page break at the end of the table of contents.
%%% In `report' style, force \chapter to print ``Part'' instead of ``Chapter'' in headings.
%%dbs odd %% \ifx\chaptername\undefined
%%dbs odd %%   %%%article style has no \chaptername
%%dbs odd %%   \renewcommand{\thesection}{\arabic{section}}%  %%dont put chapter number in the section number
%%dbs odd %%   \renewcommand{\tableofcontents}{%
%%dbs odd %%     \section*{\contentsname
%%dbs odd %%         \@mkboth{%
%%dbs odd %%            \MakeUppercase\contentsname}{\MakeUppercase\contentsname}}%
%%dbs odd %%     \@starttoc{toc}%
%%dbs odd %%     \newpage%
%%dbs odd %%   }
%%dbs odd %% \else
%%dbs odd %%   \renewcommand{\chaptername}{Part}             %%%report style
%%dbs odd %% \fi
%%dbs odd %% 
%%% Macro, variable and command (re)definitions 

\input epsf
%%\input latexsym


%%yikes
%\newcount\ncellsx \newcount\ncellsy   % number of cells in base box
%\newcount\cellsize                    % size of each cell in picture units
%\newcount\halfcell                    % cellsize / 2
%\newcount\circlesize                  % size of base level circles
%\newcount\x \newcount\y               % scratch variables
%
%%%
%%% command: withBaseBox
%%% purpose: draw a picture with a base box and allow other
%%%          Box drawing commands to be executed on the base box
%%% arg1: unit of length for picture (should include length type
%%%       (in,mm,cm,pt,...)), a cell is \cellsize * this value
%%% arg2: number of cells in X (horizontal) direction
%%% arg3: number of cells in Y (vertical) direction
%%% arg4: other drawing commands
%%% NOTE: this opens and closes a picture and defines counters:
%%%       \ncellsx, \ncellsy, \cellsize, \halfcell, \circlesize
%%%
%\newcommand{\withBaseBox}[4]{{
%  % set up constants for this picture
%  \setlength{\unitlength}{#1} \ncellsx=#2 \ncellsy=#3
%  \cellsize=100   %cell size in \unitlength's
%  \halfcell=\cellsize \divide\halfcell by 2
%  \circlesize=\cellsize \multiply\circlesize by 2 \divide\circlesize by 5
%  % make the picture size a little larger than the grid itself
%  \count101=\ncellsx \multiply\count101 by \cellsize \advance\count101 by 10
%  \count102=\ncellsy \multiply\count102 by \cellsize \advance\count101 by 10
%  \begin{picture}(\count101,\count102)
%  % this computes the grid size and draws the grid
%  \count101=\ncellsx \multiply\count101 by \cellsize
%  \count102=\ncellsy \multiply\count102 by \cellsize
%  \put(0,0){\grid(\count101,\count102)(\cellsize,\cellsize)}
%  #4
%  \end{picture}
%}}
%
%%%
%%% command: drawBox
%%% purpose: after a picture has been opened, draw a box with the specified
%%%          lower and upper corners at the specified location with the
%%%          specified refinement ratio
%%% arg1: lower X index (Level 0 index space)
%%% arg2: lower Y index ( " )
%%% arg3: upper X index ( " )
%%% arg4: upper Y index ( " )
%%% arg5: refinement ratio
%%% NOTE: this must be preceded by a \drawBaseBox{}; it leaves the picture open.
%%%
%\newcommand{\drawBox}[5]{{
%  % set the lower corner
%  \x=#1 \multiply\x by \cellsize
%  \y=#2 \multiply\y by \cellsize
%  % set the grid size (upper-lower corners) (in the Level 0 index space)
%  \count151=#3 \advance\count151 by -#1 \advance\count151 by 1 \multiply\count151 by \cellsize
%  \count152=#4 \advance\count152 by -#2 \advance\count152 by 1 \multiply\count152 by \cellsize
%  % refine the grid
%  \count153=\cellsize \divide\count153 by #5
%  % draw the refined grid
%  \put(\x,\y){\grid(\count151,\count152)(\count153,\count153)}
%  % draw a thick outline of the box if the refinement ratio >1
%  \ifnum #5>1
%   { \thicklines\put(\x,\y){\grid(\count151,\count152)(\count151,\count152)}}
%  \fi
%}}
%
%%%
%%% command: drawTag
%%% purpose: draw a Tag in the cell at the specified indices corresponding to
%%%          the specified refinement ratio
%%% arg1: X index of tag
%%% arg2: Y index of tag
%%% arg3: refinement ratio
%%% NOTE: this must be preceded by a \drawBaseBox{}; it leaves the picture open.
%%%
%\newcommand{\drawTag}[3]{{
%  \divide \cellsize by #3 \divide\halfcell by #3
%  \divide\circlesize by #3
%  \x=#1 \multiply\x by \cellsize \advance\x by \halfcell
%  \y=#2 \multiply\y by \cellsize \advance\y by \halfcell
%  \put(\x,\y){\circle*{\circlesize}}  %circle* draws a filled circle
%}}
