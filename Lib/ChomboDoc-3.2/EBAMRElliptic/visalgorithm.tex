\newcommand{\vb}{\vec v }
\newcommand{\vx}{\vec x }

\section{Viscous Tensor Equation}
This section describes the method for solving the elliptic partial
differential equation
$$
\kappa L \vb = \kappa \alpha \vb + \beta \nabla \cdot F = \kappa \rho.
$$
$\alpha$ is a constant and $\beta = \beta(\vx)$. $F$ is given by 
\begin{equation}
F = \eta(\nabla \vb + \nabla \vb^T) +  \lambda (I \nabla \cdot \vb)
\label{eqn::visfluxform}
\end{equation}
where $I$ is the identity matrix, $\eta = \eta(\vx)$, and $\lambda =
\lambda(\vx)$.
\subsection{Discretization}

We discretize normal components of the face-centered gradient  using 
an average of cell-centered gradients for tangential components and a
centered-difference approximation to the normal gradient.
$$
(\nabla \vb)^{d'}_{\ibold + \half \ebold^d} = \left( \begin{array}{c}
                      \frac{1}{h}(\vb_{\ibold+ \ebold^d}  -\vb_{\ibold})  \mbox{ if } d = d'\\
                      \frac{1}{2}((\nabla \vb)^{d'}_{\ibold+\ebold^d}  +
                      (\nabla \vb)^{d'}_{\ibold})  \mbox{ if } d \neq d'
                      \end{array}
                      \right)
$$
where 
$$
(\nabla \vb)^{d}_{\ibold} = \frac{1}{2 h}(\vb_{\ibold + \ebold^d} - \vb_{\ibold - \ebold^d}).
$$
We discretize the divergence as follows
$$
(\kappa \nabla \cdot F)_{\ibold} = \sum\limits^{D}_{d'=1} 
(\alpha \nabla F)^{d'}_{\ibold + \half \ebold^d} + \alpha_B F^B
$$

where $\kappa$ and $\alpha$ are the volume and area fractions.

We use equation \ref{eqn::visfluxform} get the flux at cell face
centers.   We then interpolate the flux to face centroids.
In two dimensions, this interpolation takes the form
\begin{gather}
\widetilde{F}^\nph_\fbold = 
F^\nph_\fbold  + |\bar x| 
(F^\nph_{\ebshift{\fbold}{\,sign(\bar x) \ebold^d}}-F^\nph_\fbold) 
\end{gather}
where $\bar x$ is the centroid in the direction $d$ perpendicular to the
face normal. In three dimensions, define $(\bar x, \bar y)$ to be the
coordinates of the centroid in the plane $(d^1, d^2)$ perpendicular 
to the face normal. 
\begin{align}
\widetilde{F}^\nph_\fbold =& 
F^\nph_\fbold( 1 -\bar x \bar y + |\bar x \bar y | ) + \\
&F^\nph_{\ebshift{\fbold}{\,sign(\bar x) \ebold^{d^1}}} 
(|\bar x| - |\bar x \bar y|) + \\
&F^\nph_{\ebshift{\fbold}{\,sign(\bar x) \ebold^{d^2}}} 
(|\bar y| - |\bar x \bar y|) + \\
&F^\nph_{\fbold << sign(\bar x) \ebold^{d^1} << sign(\bar x)
\ebold^{d^2}}  (|\bar x \bar y |)
\end{align}
Centroids in any dimension are normalized by $\dx$ and centered at the
cell center.   This interpolation is only done if the shifts that are
used in the interpolation are uniquely-defined and single-valued.
We use a conservative discretization for the flux divergence.
\begin{gather}
\kappa_\vbold \nabla \cdot \vec{F} \equiv (D \cdot \vec{F}) = 
((\sum^{\Dim-1}_{d=0} \sum_{\pm = +,-} \sum_{\fbold \in
\mathcal{F}^{d, \pm}_\vbold} \pm \alpha_\fbold \widetilde{F}^\nph_\fbold) +
\alpha^B_\vbold F^{B, \nph}_\vbold) 
\end{gather}
where 
where $F^B$ is the flux at the irregular boundary, wherein lies most
of the difficulty in this operator.

\subsection{Flux at the Boundary}
In all cases, we construct the gradient at the boundary and use 
equation \ref{eqn::visfluxform} to construct the flux.

For Neumann boundary conditions, the gradient of the solution is
specified at the boundary.

For Dirichlet boundary conditions, the gradient normal to the boundary
is determined using the value at the boundary.  The gradients
tangential to the boundary are specified.  For irregular
boundaries, the procedure for calculating the gradient normal to the
boundary is given in section \ref{sec::ebbc}.   For domain boundaries,
we construct a quadratic function with the value at the boundary and
the two adjacent points along the normal to construct the gradient.
For example, say we are at the low side of the domain with a value 
$\phi_0$ at the boundary.  The normal gradient is given by.
that means that normal gradient  is given by
$$
(\nabla \phi)^x_{-\half, j, k} = \frac{9(\phi_{0,j,k}-\phi_0)  -
  (\phi_{1,j,k}-\phi_0)}{3 \dx}
$$
