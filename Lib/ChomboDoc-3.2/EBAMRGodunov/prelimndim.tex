\section{Introduction}

This document describes our numerical method for integrating systems of
conservation laws (e.g., the Euler equations of gas dynamics) on
an AMR grid hierarchy with embedded boundaries.  
We use an unsplit, second-order Godunov method, extending the 
algorithms developed by Colella \cite{COLELLA1} and
Saltzman \cite{Saltzman1}.

\section{Notation}


All these operations take place in a very similar context to
that presented in \cite{ChomboDesign}.  For non-embedded boundary
notation, refer to that document. 

The standard $(i,j,k)$ is not sufficient here to denote a
computational cell as there can be multiple VoFs per cell.
We define $\vbold$ to be the notation for a VoF and $\fbold$
to be a face. The function $ind(\vbold)$ produces the 
cell which the VoF lives in.  We define $\vbold^+(\fbold)$ to be the
VoF on the high side of face~$\fbold$; $\vbold^-(\fbold)$ 
is the VoF on the low side
of face~$\fbold$; $\fbold^+_d(\vbold)$ is the set of faces on the high side of
VoF~$\vbold$; $\fbold^-_d(\vbold)$ is the set of faces on the low side of VoF~$\vbold$,
where $d\in\{x,y,z\}$ is a coordinate direction (the number of
directions is $D$).  Also, we compose these operators to represent the
set of VoFs directly connected to a given VoF:  $\vbold^+_d(\vbold) =
\vbold^+(\fbold^+_d(\vbold))$ and $\vbold^-_d(\vbold) = \vbold^-(\fbold^-_d(\vbold))$.
The $\ebshift{}{}$ operator shifts data in the direction of the right hand
argument.   The shift operator can yield multiple VoFs.  In this 
case, the shift operator includes averaging the values at the shifted-to
VoFs.

We follow the same approach in the EB case in defining multilevel data
and operators as we did for ordinary AMR. Given an AMR mesh hierarchy
$\{ \Omega^l \}^{lmax}_{l=0}$, we define the valid VoFs on level $l$ to be 
\begin{equation}
\mathcal{V}^l_{valid} = ind^{-1}(\Omega^l_{valid})
\end{equation}
and composite cell-centered data
\begin{equation}
\varphi^{comp} = \{ \varphi^{l, valid} \}^{lmax}_{l=0}, 
\varphi^{l,valid} : \mathcal{V}^l_{valid} \rightarrow \mathbb{R}^m
\end{equation}
For face-centered data,
\begin{equation}
\begin{split}
\mathcal{F}^{l, d}_{valid} = & \; ind^{-1}(\Omega^{l, \ebold^d}_{valid}) \\
\vec{F}^{l, valid} = & \; (F^{l, valid}_0, \ldots, F^{l, valid}_{D-1}) \\
F^{l, valid}_d : & \; \mathcal{F}^{l, d}_{valid} \rightarrow \mathbb{R}^m
\end{split}
\end{equation}
For computations at cell centers the notation
\beqa
CC = A ~ | ~ B ~ | ~ C
\eeqa
means that
the 3-point formula $A$ is used for $CC$ if all cell centered values it uses
are available,
the 2-point formula $B$ is used if current cell borders the high side
of the physical domain (i.e., no high side value), and
the 2-point formula $C$ is used if current cell borders the low side
of the physical domain (i.e., no low side value).  A value is
``available'' if its VoF is not covered and is within the domain 
of computation. For computations at face centers the analogous notation
\beqa
FC = A ~ | ~ B ~ | ~ C
\eeqa
means that
the 2-point formula $A$ is used for $FC$ if all cell centered values it uses
are available,
the 1-point formula $B$ is used if current face coincides with the high side
of the physical domain (i.e., no high side value), and
the 1-point formula $C$ is used if current face coincided with the low side
of the physical domain (i.e., no low side value).

\section{Equations of Motion}

We are solving a hyperbolic system of equations of the form
\begin{equation}
\frac{\partial U}{\partial t}
+ \sum\limits^{\Dim-1}_{d=0} \frac{\partial F^d}{\partial x^d}
= S
\end{equation}
For 3D polytropic gas dynamics, 
\begin{align}
\begin{split}
U     =& \left(\rho      , \rho u_x      , \rho u_y    , \rho u_z     , \rho E \right)^T \\
F^x   =& \left(\rho u_x  , \rho u^2_x    , \rho u_x u_y, \rho u_x u_z , \rho u_x E +u_x  p  \right)^T  \\
F^y   =& \left(\rho u_y  , \rho u_x u_y  , \rho u^2_y  , \rho u_y u_z , \rho u_y E +u_y  p  \right)^T  \\
F^z   =& \left(\rho u_z  , \rho u_x u_z  , \rho u_z u_y, \rho u^2_z   , \rho u_z E +u_z  p  \right)^T  \\
E     =&  \frac{\gamma p }{(\gamma -1) \rho} + \frac{|\vec u|^2}{2}  
\end{split}
\end{align}
We are given boundary
conditions on faces at the boundary of the domain and
on the embedded boundary.
We also assume there may be a change of variables $W = W(U)$ ($W \equiv
$ ``primitive variables'') that can be applied to simplify the
calculation of the characteristic structure of the equations.
This leads to a similar system of equations in $W$.
\begin{equation}
\begin{split}
\frac{\partial W}{\partial t}
+ & \; \sum\limits^{\Dim-1}_{d=0} A^d(W) ~ \frac{\partial W^d}{\partial x^d}
= S' \\
A^d = & \; \nabla_U W \cdot \nabla_U F^d \cdot \nabla_W U \\
S' = & \; \nabla_U W \cdot S
\end{split}
\end{equation}
For 3D polytropic gas dynamics, 
\begin{equation*}
W   = \left(\rho,  u_x, u_y, u_z , p \right)^T \\
\end{equation*}
\begin{equation*}
A^x = \left( \begin{array}{ccccc} 
 u_x       & \rho      & 0        &  0       & 0                \\
 0         & u_x       & 0        &  0       & \frac{1}{\rho}   \\
 0         & 0         & u_x      &  0       & 0                \\
 0         & 0         & 0        & u_x      & 0                \\
 0         & \rho c^2  & 0        &  0       &  u_x
\end{array}        \right)
\end{equation*}
\begin{equation*}
A^y = \left( \begin{array}{ccccc} 
 u_y       & 0         & \rho     &  0       & 0                \\
 0         & u_y       & 0        &  0       & 0                \\
 0         & 0         & u_y      &  0       & \frac{1}{\rho}   \\
 0         & 0         & 0        & u_y      & 0                \\
 0         & 0         & \rho c^2 &  0       & u_y  
\end{array}        \right)
\end{equation*}
\begin{equation*}
A^z = \left( \begin{array}{ccccc} 
 u_z       & 0         & 0        & \rho     & 0                \\
 0         & u_z       & 0        &  0       & 0                \\
 0         & 0         & u_z      &  0       & 0                \\
 0         & 0         & 0        & u_z      & \frac{1}{\rho}   \\
 0         & 0         & 0        &\rho c^2  & u_z
\end{array}        \right)
\end{equation*}

\section{Approximations to $\nabla \cdot F$.}
\label{sec::divf}

To obtain a second-order approximation of the flux divergence in
conservative form, first we must interpolate the flux to the face
centroid.  In two dimensions, this interpolation takes the form
\begin{gather}
\widetilde{F}^\nph_\fbold = 
F^\nph_\fbold  + |\bar x| 
(F^\nph_{\ebshift{\fbold}{\,sign(\bar x) \ebold^d}}-F^\nph_\fbold) 
\end{gather}
where $\bar x$ is the centroid in the direction $d$ perpendicular to the
face normal. In three dimensions, define $(\bar x, \bar y)$ to be the
coordinates of the centroid in the plane $(d^1, d^2)$ perpendicular 
to the face normal. 
\begin{align}
\widetilde{F}^\nph_\fbold =& 
F^\nph_\fbold( 1 -\bar x \bar y + |\bar x \bar y | ) + \\
&F^\nph_{\ebshift{\fbold}{\,sign(\bar x) \ebold^{d^1}}} 
(|\bar x| - |\bar x \bar y|) + \\
&F^\nph_{\ebshift{\fbold}{\,sign(\bar x) \ebold^{d^2}}} 
(|\bar y| - |\bar x \bar y|) + \\
&F^\nph_{\fbold << sign(\bar x) \ebold^{d^1} << sign(\bar x)
\ebold^{d^2}}  (|\bar x \bar y |)
\end{align}
Centroids in any dimension are normalized by $\dx$ and centered at the
cell center.   This interpolation is only done if the shifts that are
used in the interpolation are uniquely-defined and single-valued.

We then define the conservative divergence approximation.
\begin{gather}
\nabla \cdot \vec{F} \equiv (D \cdot \vec{F})^c = \frac{1}{k_\vbold
h}((\sum^{\Dim-1}_{d=0} \sum_{\pm = +,-} \sum_{\fbold \in
\mathcal{F}^{d, \pm}_\vbold} \pm \alpha_\fbold \widetilde{F}^\nph_\fbold) +
\alpha^B_\vbold F^{B, \nph}_\vbold) 
\end{gather}
The non-conservative divergence approximation is defined below.
\begin{align}
\nabla \cdot \vec{F} =&\; (D \cdot \vec{F})^{NC} = \frac{1}{h}
\sum_{\pm = +, -} \sum^{\Dim-1}_{d=0} \pm \bar{F}^\nph_{\vbold, \pm,d} 
 \\ 
\bar{F}^\nph_{\vbold, \pm, d} =&\;
  \begin{cases}
\frac{1}{N(\mathcal{F}^{d, \pm}_\vbold)} \sum_{\fbold \in
\mathcal{F}^{d, \pm}_\vbold} F^\nph_\fbold & \text{if
$N(\mathcal{F}^{d, \pm}_\vbold) > 0 $} \\ 
F^{\text{covered}, \nph}_{\vbold, \pm, d} & \text{otherwise}
  \end{cases}
\label{eqn::divnc}
\end{align}

The preliminary update update of the solution of the solution 
takes the form:
\begin{gather}
U^{n+1}_\vbold = U^n_\vbold - \Delta t((1-k_\vbold)(D \cdot
\vec{F})^{NC}_\vbold + k_\vbold(D \cdot \vec{F})^c_\vbold) \\ 
\delta M = -\Delta t k_\vbold (1-k_\vbold)((D \cdot \vec{F})^c_\vbold
- (D \cdot \vec{F})^{NC}_\vbold) 
\label{eqn::prelimup}
\end{gather}
$\delta M$ is the total mass increment that has been unaccounted for
in the preliminary update.  See the EBAMRTools document for how this
mass gets redistributed in an AMR context.  On a single level, the
redistribution takes the following form:
\begin{gather}
U^{n+1}_\vprime := U^{n+1}_\vprime + w_{\vbold, \vprime}, \delta
M_\vbold \\ \vprime \in \mathcal{N}(\vbold), 
\label{eqn::redist}
\end{gather}
where $\mathcal{N}(\vbold)$ is the set of VoFs that can be connected
to $\vbold$ with a monotone path of length $\leq 1$. The weights are
nonnegative, and satisfy $\underset{\vprime \in
\mathcal{N}(\vbold)}{\sum} \kappa_\vprime w_{\vbold, \vprime} = 1$. 

\section{Flux Estimation}

Given $U^n_\ibold$ and $S^n_\ibold$, we want to compute a second-order accurate
estimate of the fluxes: $F^{n+\half}_\fbold \approx
F^d(\xbold_0 +(\ibold+\half \ebold^d)h,t^n + \half \Delta t)$.  
Specifically, we want to compute the fluxes at the center
of the Cartesian grid faces corresponding to the faces of the embedded
boundary geometry. In addition, we want to compute fluxes at the
centers of Cartesian grid faces corresponding to faces adjacent to
VoFs, but that are completely covered.
Pointwise operations are 
conceptually the same for both regular and irregular VoFs. In
other operations we specify both the regular and irregular 
VoF calculation. The transformations $\nabla_U W $ and
$\nabla_W U$ are functions of both space and time.  We shall
leave the precise centering of these transformations vague
as this will be application-dependent.
In outline, the method is given as follows.

