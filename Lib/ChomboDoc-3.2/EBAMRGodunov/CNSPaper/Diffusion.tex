\section{Diffusion\label{sec:Diffusion}}

Unlike the Euler equations, the Navier-Stokes equations include the effects 
of viscosity and thermal heat exchange in fluids. In this section we describe 
the elliptical operators that represent these contributions and describe how 
they are integrated with respect to their hyperbolic counterparts.

\subsection{Viscous Tensor and Thermal Conduction Operators}
The viscous term in \refEq{momentum} is 

\begin{equation}
\ddxj{}\left(\mu\left[\ddxj{v_i} + \ddxi{v_j}\right] + \lambda\deltaij\ddxk{v_k}\right)
\end{equation}

\noindent
and may be expressed in terms of the linear elliptical operator $\Lmu$ that 
operates on the 3-dimensional velocity vector $\vec{v}$:

\begin{equation}
\Lmu(\vec{v}) = \alpha \tens{I} + \beta\diverg{\left(\mu\left[\grad{\vec{v}} + 
                   \grad{\vec{v}}^T\right] + \lambda (\diverg{\vec{v}})\tens{I}\right)} \label{eq:Lmu}
\end{equation}

\noindent
where $\alpha$ and $\beta$ are time integration parameters (discussed in 
\refApp{TGA}), $\tens{I}$ is the $3\times 3$ identity matrix, $\grad{\vec{v}}$ 
and $\grad{\vec{v}}^T$ are matrices representing the deformation gradient, and
$\diverg{\vec{v}}$ is a scalar representing the velocity divergence.  We refer 
to the discrete form of this operator as $L_\mu$. In this discrete form, the 
spatial derivatives of the velocity are approximated with the finite volume 
method we have mentioned earlier.

The contribution to \refEq{energy} from thermal conduction is

\begin{equation}
\rho\ddxi{}\left(K\ddxi{T}\right).
\end{equation}

\noindent
We can rewrite this term using the linear elliptic operator $\LK$, which 
operates upon the scalar temperature $T$:

\begin{equation}
\LK(T) = \left(\alpha A T + \beta \diverg{\left[B \grad{T}\right]}\right)
\end{equation}

\noindent
where $\alpha$ and $\beta$ are the integration parameters and $B = \rho c_v K$.
\em What is A? Just a generalization? \em The discrete form of $\LK$ is $L_K$, 
in which the spatial derivatives of $T$ are replaced by finite volume 
approximations.

\subsection{Implicit Time Integration}

To avoid the $(\dx)^2$ constraint on the time step from these elliptic 
operators, we integrate their terms implicitly to compute stable updates for 
$\rho\vec{v}$ and $\rho E$. Because we have already computed contributions 
from the hyperbolic terms, we account for these contributions as sources in 
their respective diffusion equations. For example, the viscous contribution to 
the momentum equation is obtained given by integrating

\begin{equation}
\rho_\ib\ddt{\vec{v}_\ib} = L_\mu(\vec{v}_\ib) + \rho_\ib\left(\ddt{\vec{v}_\ib}\right)_{hb} \label{eq:viscContrib}
\end{equation}

\noindent
where $(\iddt{\vec{v}_\ib})_{hb} = -(\vec{v}\cdot\grad{\vec{v}})_\ib - (\grad{p}/\rho)_\ib$ 
represents the hyperbolic portion of $\iddt{\vec{v}}$ on the VoF $\ib$. 
\em Check on this!\em

We integrate \refEq{viscContrib} using an algorithm by Twizell, Gummel and Arign 
\cite{TGA}, which we refer to henceforth as the TGA algorithm. This algorithm 
is second-order accurate and $L^\infty$-stable and has been successfully used 
to treat incompressible Navier-Stokes flows on embedded boundaries\cite{???}.
By contrast, the more well-known Crank-Nicholson algorithm can become unstable 
near boundaries of irregular domains in the presence of the discrete Laplacian 
operator\cite{???}. We outline the TGA algorithm in \refApp{TGA}. 

In any case, integrating \refEq{viscContrib} gives us an updated velocity 
$\vec{v}^{n+1}_\ib$ which we can use to create a stabilized value for 
$L_\mu(\vec{v})$:

\begin{equation}
\hat{L}_\mu(\vec{v}_\ib) \rightarrow \frac{\rho^{n+\half}_\ib(\vec{v}^{n+1}_\ib - \vec{v}^n_\ib)}{\dt} \label{eq:stableLmu}
\end{equation}

\noindent
Then we add the viscous contribution to the momentum:

\begin{equation}
(\rho\vec{v})^{n+1}_\ib \rightarrow (\rho\vec{v})^{n+1}_\ib + \dt \hat{L}_\mu(\vec{v}_\ib) \label{eq:viscUpdate}
\end{equation}

The treatment of the thermal contribution to the energy is similar. The 
rate of heat transfer through thermal contribution in the energy equation is

\begin{equation}
\rho_\ib\ddt{T_\ib} = L_K(T_\ib) + \frac{1}{\rho_\ib}\left(\ddt{\rho_\ib E_\ib}\right)_{hb} \label{eq:thermContrib}
\end{equation}

\noindent
where $(\iddt{\rho E_\ib})_{hb} = -\diverg{([\rho E + p]\vec{v})}_\ib$ 
represents the (hyperbolic) transport of energy on the VoF $\ib$. We integrate
\refEq{thermContrib} to obtain $T^{n+1}$, the updated temperature, which we 
use to stabilize $L_K(T)$:

\begin{equation}
\hat{L}_K(T_\ib) = \frac{\rho^{n+\half}_\ib c_v (T^{n+1}_\ib - T^n_\ib)}{\dt} \label{eq:stableLK}
\end{equation}

\noindent
Finally, we add the thermal contribution to the specific energy:

\begin{equation}
(\rho E)^{n+1}_\ib \rightarrow (\rho E)^{n+1}_\ib + \dt \hat{L}_K(T_\ib) \label{eq:thermUpdate}
\end{equation}
