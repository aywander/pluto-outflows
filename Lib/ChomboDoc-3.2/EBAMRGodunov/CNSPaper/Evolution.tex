\section{Evolution of Equations\label{sec:Evolution}}

In this section we describe the time integration algorithm schematically, 
giving details of each step of the calculation in later sections.
At the beginning of a time step $n$, we integrate the hyperbolic terms in 
\refEq{continuity} - \refEq{energy} explicitly, treating the elliptic terms in 
\refEq{momentum} and \refEq{energy} as sources. To compute the source 
contributions, we evaluate the discrete form of \refEq{LU} within \refEq{advectionDiffusion} at
the current timestep.  In this language, the operator $\mathcal{L}$ can 
be written

\begin{equation}
\mathcal{L}(U) = [0, \Lmu(\vec{v}), \LK(T)]^T. \label{eq:LU}
\end{equation}

\noindent
We denote the discrete form of this operator evaluated at time step $n$ on the 
VoF $\ib$ as $L(U^n_\ib)$, which consists of the separate discrete Helmholtz 
operators $L_\mu(\vec{v}^n_\ib)$ and $L_K(T^n_\ib)$. We represent the 
elliptic terms as a source term in \refEq{advectionDiffusion}:

\begin{equation}
S^n_\ib = L(U^n_\ib) = [0, L_\mu(\vec{v}^n_\ib), L_K(T^n_\ib)]^T \label{eq:hyperbolicSources}
\end{equation}

The details of the evaluation of these ``hyperbolic sources" are given in 
\refSec{HyperbolicSources}.

Next, we must calculate the flux divergence 
$(\diverg{\vec{F}})^{n+1/2}_\ib = \diverg{F}(U^{n+1/2}_\ib, S^n_\ib)$,
incorporating the source terms computed above. This step is similar to the 
calculation of the flux divergence described in \cite{???} for solving the 
Euler equations; we review it in \refSec{FluxDivergence}. 

When we have computed the flux divergence, we can predict a value for 
$U^{n+1}_\ib$ using \refEq{advectionDiffusion}:

\begin{equation}
U^{n+1,*}_\ib = U^b_\ib - \dt(\diverg{\vec{F}})^{n+\half}_\ib
\end{equation}

\noindent
We ignore the implicit terms in \refEq{advectionDiffusion} in this explicit 
advance, apart from incorporating them as sources in the calculation of the 
flux divergence. We then update the momentum and energy in $U$ by solving the 
appropriate diffusion equations and backing out time derivatives. For example, 
implicitly integrating the viscous contribution to the momentum equation 
produces a value for $\vec{v}^{n+1}_\ib$, which can be used to compute 
$L_\mu(\vec{v}^{n+1/2}_\ib)$:

\begin{equation}
L_\mu(\vec{v}^{n+\half}_\ib) = \frac{\rho^{n+\half}(\vec{v}^{n+1} - \vec{v}^n)}{\dt}
\end{equation}

\noindent
Similarly, the implicit treatment of thermal conduction for the energy equation
gives us $T^{n+1}_\ib$, which is used to compute $L_K(T^{n+1/2}_\ib)$:

\begin{equation}
L_K(T^{n+\half}_\ib) = \frac{\rho^{n+\half} c_v (T^{n+1} - T^n)}{\dt}
\end{equation}

The integration of these equations is described in \refSec{Diffusion}. 

Finally, the solution variable is updated with the diffusive contributions:

\begin{equation}
U^{n+1}_\ib = U^{n+1,*} + \dt L(U^{n+\half}_\ib) \label{eq:Uupdate}
\end{equation}

\noindent
or, in terms of the conserved quantities:

\begin{eqnarray}
(\rho\vec{v})^{n+1}_\ib &=& (\rho\vec{v})^{n+1,*} + \dt L_\mu(\vec{v}^{n+\half}_\ib) \label{eq:rhovupdate} \\
(\rho E)^{n+1}_\ib &=& (\rho E)^{n+1,*} + \dt L_K(T^{n+\half}_\ib) \label{eq:rhoEupdate}
\end{eqnarray}

