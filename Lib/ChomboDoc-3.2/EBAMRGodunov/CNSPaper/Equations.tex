\section{Governing Equations\label{sec:Equations}}

We are concerned with solving systems of advection-diffusion equations on 
irregular domains. A system of this sort can be written as

\begin{equation}
\ddt{U} + \diverg{\vec{F}} = \mathcal{L}(U) \label{eq:advectionDiffusion}
\end{equation}

\noindent
where $U = U(\vec{x}, t)$ is a solution vector, $\vec{F} = \vec{F}(U)$ is an 
advective flux, and $\mathcal{L}$ is a linear diffusion operator. The presence 
of diffusion (elliptical) terms in such a system presents a problem for the
explicit time integration methods often used in advective (hyperbolic) systems, 
since discretizations containing these terms have time step constraints 
that scale with $\dx^2$ (where $\dx$ is the grid spacing). Our goal is to 
treat the elliptical terms implicitly while integrating the hyperbolic terms
explicitly.

We illustrate this implicit/explicit method by solving the compressible 
Navier-Stokes equations. These conservation laws describe the behavior of a 
compressible fluid with thermal conduction and viscosity. Written
in conservative form, the Navier-Stokes equations are

\begin{eqnarray}
\ddt{\rho} + \diverg{(\rho \vec{v})} &=& 0 \notag \\ 
\ddt{(\rho \vec{v})} + \diverg{(\rho \vec{v}\vec{v})} &=& -\grad{p} + \diverg{\tens{\sigma}} \label{eq:CNS} \\
\ddt{(\rho E)} + \diverg{(\rho E \vec{v})} &=& -\diverg{(p\vec{v})} + \rho\diverg{\vec{Q}} \notag
%\ddt{\rho} + \ddxi{(\rho v_i)} &=& 0 \label{} \\
%\ddt{(\rho v_i)} + \ddxi{(\rho v_i v_j)} &=& -\ddxi{p} + \ddxi{\sigmaij} \\
%\ddt{(\rho E)} + \ddxi{(\rho v_i E)} &=& -\ddxi{(v_i p)} + \rho\ddxi{Q_i}
\end{eqnarray}

\noindent
These equations are (respectively) the laws of conservation of mass, momentum, 
and energy. Above, $\rho$ is the mass density of the fluid, $\vec{v}$ is its 
bulk velocity, $p$ its pressure, and $E$ its total energy per unit volume. 
$\tens{\sigma}$ is the viscous stress tensor representing the dissipation of 
kinetic energy into heat, and $\vec{Q}$ is the flow of heat within the fluid. 

In a Newtonian fluid, the viscous stress can be expressed in terms of the 
viscosity coefficients $\mu$ and $\lambda$ and the strain rate tensor 
$\tens{e}$ whose components are defined by $e_{ij} = (\iddxj{v_i} + \iddxi{v_j})/2$\cite{Bachelor}. 
In Einstein notation, in which repeated indices are summed:

\begin{eqnarray}
\sigmaij &=& \mu\left(\ddxj{v_i} + \ddxi{v_j}\right) + \lambda \deltaij \ddxk{v_k} \notag \\
         &=& 2\mu e_{ij} + \lambda \deltaij e_{kk} \label{eq:viscTensor}
\end{eqnarray}

\noindent
$\mu$ represents the rate at which shear flows generate heat at the expense of 
the fluid's mechanical energy. This parameter can depend upon the pressure 
and temperature of the fluid and thus is allowed to vary spatially. $\lambda$,
often called the \em second viscosity\em, is the rate at which this conversion 
occurs in the presence of compression, It is customary to prescribe 
$\lambda$ in terms of $\mu$ in order to make $\tens{\sigma}$ traceless:

\begin{equation}
\lambda = -\frac{2}{3}\mu \label{eq:lambda}
\end{equation}

\noindent
We adopt this practice for the present study. In reality, $\lambda$ can depend 
on the frequencies of compression waves, so its behavior can be much more 
complicated\cite{LandauFM}.

The heat flow $\vec{Q}$ is related to gradients in the fluid's temperature $T$ 
by the thermal conductivity $K$ of the fluid, which can also vary in space:

\begin{equation}
\vec{Q} = K \grad{T} \label{eq:heatFlow}
\end{equation}

When \refEq{viscTensor} and \refEq{heatFlow} are used in \refEq{CNS}, they 
result in diffusion terms within the momentum and energy equations. 
We can rearrange the resulting equations to place them into a form resembling 
\refEq{advectionDiffusion}:

\begin{eqnarray}
\ddt{\rho} + \ddxi{(\rho v_i)} &=& 0 \label{eq:continuity} \\
\ddt{(\rho v_i)} + \ddxj{(\rho v_i v_j + p\deltaij)} &=& 
   \ddxj{}\left(\mu\left[\ddxj{v_i} + \ddxi{v_j} - \frac{2}{3}\deltaij\ddxk{v_k}\right]\right) \label{eq:momentum} \\
\ddt{(\rho E)} + \ddxi{(\rho E v_i + p v_i)} &=& \rho\ddxi{}\left(K\ddxi{T}\right) \label{eq:energy} 
\end{eqnarray}

\noindent
To express \refEq{continuity} - \refEq{energy} in the form of 
\refEq{advectionDiffusion}, we identify the solution vector $U$ and the flux 
$F^d$ in the $d$th direction:

\begin{eqnarray}
U &=& (\rho, \rho\vec{v}, \rho E)^T \label{eq:solnVector} \\
F^d(U) &=& (\rho v^d, \rho v^d\vec{v} + p\tens{e}^d,\rho v^d E + v^d p)^T \label{eq:Fd}
\end{eqnarray}

\noindent
Since the diffusion terms on the right hand sides of \refEq{momentum} and 
\refEq{energy} do not couple the conserved quantites, the momentum and energy 
equations are not coupled in the operator $\mathcal{L}$. This means we are 
allowed to treat the diffusion terms separately, defining the a viscous 
operator $\Lmu$ and a thermal conduction operator $\LK$. Because $\Lmu$ and 
$\LK$ are decoupled, each operates on its respective primitive variable and 
not on $U$. The specific forms of these operators can be found by ignoring 
the advection terms in \refEq{momentum} and \refEq{energy}.  For instance, the 
dissipation in the momentum of an element of fluid with mass density $\rho$ 
due to viscous heating is

\begin{eqnarray}
\left(\ddt{(\rho v_i)}\right)_{\mu} &=& \ddxj{}\left(\mu\left[\ddxj{v_i} + \ddxi{v_j} - \frac{2}{3}\deltaij\ddxk{v_k}\right]\right) \notag \\
                                    &\equiv& \Lmu(\vec{v}). \label{eq:Lmu}
\end{eqnarray}

\noindent
For the energy equation, we need to relate the fluid's temperature $T$ to its 
energy per unit mass, or \em specific energy\em, $E$. If, under no compression, 
it takes $c_v$ units of energy to increase the temperature of a unit of mass of 
fluid by a single unit, that fluid is said to have a specific heat (at constant 
volume) $c_v$. The thermal specific energy of a unit mass of such a fluid is 
$E_{therm} = c_v T$.  We then express the thermal diffusion equation as

\begin{eqnarray}
\left(\ddt{(\rho c_v T)}\right)_K &=& \rho\ddxi{}\left(K\ddxi{T}\right) \notag \\
                                      &\equiv& \LK(T). \label{eq:LK}
\end{eqnarray}

\noindent
For simplicity, we assume that $c_v$ is constant within the fluid.  

