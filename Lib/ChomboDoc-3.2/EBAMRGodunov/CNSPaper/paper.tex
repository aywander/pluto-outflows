% Template article for preprint document class `elsart'
% with harvard style bibliographic references
% SP 2001/01/05

\documentclass{elsart}

%
% macros.tex: define macros 
%
\def\aveetoc{Av^{E \rightarrow C}}
\def\avectoe{Av^{C \rightarrow E}}
\def\ten#1#2{$#1 \cdot 10^{#2}$}
\def\prt{\partial}
\def\appx{\sim}
\def\inorm#1{\| #1 \|_\infty}
\def\L2norm#1{\| #1 \|_2}
\def\T{\verb+<+{\bf T}\verb+>+}
\def\Ti#1{\verb+<+{\bf #1}\verb+>+}
\def\pluseq{\mathbin{+\mkern-7mu=}}
\def\minuseq{\mathbin{-\mkern-7mu=}}

\let\realpar=\par%              %%%[NOTE: I don't know what this does. -dbs Apr99]
\newcommand{\Fvec}{\mbox{\boldmath $F$}}
\newcommand{\Gvec}{\mbox{\boldmath $G$}}
\newcommand{\ivec}{{\mbox{\boldmath $i$}}}
\newcommand{\fvec}{{\mbox{\boldmath $f$}}}
\newcommand{\pvec}{\mbox{\boldmath p}}
\newcommand{\vvec}{{\mbox{\boldmath $v$}}}
\newcommand{\uvec}{{\mbox{\boldmath $u$}}}
\newcommand{\evec}{{\mbox{\boldmath $e$}}}
\newcommand{\xvec}{{\mbox{\boldmath $x$}}}
\newcommand{\jvec}{\mbox{\boldmath j}}
\newcommand{\kvec}{\mbox{\boldmath k}}
\newcommand{\normal}{\mbox{\boldmath $n$}}

\newcommand{\vpmd}{{\vbold, \pm, d}}
\newcommand{\vpd}{{\vbold, +, d}}
\newcommand{\vmd}{{\vbold, -, d}}
\newcommand{\zbold}{{\boldsymbol{z}}}
\newcommand{\nbold}{{\boldsymbol{n}}}
\newcommand{\vprime}{{\boldsymbol{v}^{'}}}
\newcommand{\vprimec}{{\boldsymbol{v'_c}}}
\newcommand{\vprimef}{{\boldsymbol{v'_f}}}
\newcommand{\vboldc}{{\boldsymbol{v_c}}}
\newcommand{\vboldf}{{\boldsymbol{v_f}}}
\newcommand{\nref}{{{N}_{ref}}}
\newcommand{\ind}{{\it ind}}

\newcommand{\Abold}{{\mbox{\boldmath $A$}}}
\newcommand{\Bbold}{{\mbox{\boldmath $B$}}}
\newcommand{\Jbold}{{\mbox{\boldmath $J$}}}
\newcommand{\Fbold}{{\mbox{\boldmath $F$}}}
%\newcommand{\ebold}{{\mbox{\boldmath $e$}}}
\newcommand{\ebold}{{\boldsymbol{e}}}
\newcommand{\xbold}{{\boldsymbol{x}}}
\newcommand{\ybold}{{\boldsymbol{y}}}
\newcommand{\betabold}{{\boldsymbol{\beta}}}
\newcommand{\pbold}{{\boldsymbol{p}}}
\newcommand{\qbold}{{\boldsymbol{q}}}
\newcommand{\rbold}{{\boldsymbol{r}}}
\newcommand{\dbold}{{\boldsymbol{d}}}
\newcommand{\jbold}{{\boldsymbol{j}}}
\newcommand{\sbold}{{\boldsymbol{s}}}
\newcommand{\fbold}{{\boldsymbol{f}}}
\newcommand{\wbold}{{\boldsymbol{w}}}
\newcommand{\vbold}{{\boldsymbol{v}}}
\newcommand{\ubold}{{\boldsymbol{u}}}
\newcommand{\ibold}{{\boldsymbol{i}}}
\newcommand{\lbold}{{\boldsymbol{l}}}
\newcommand{\Dim}{{\mathbf{D}}}
\newcommand{\Ident}{{\mathbf{I}}}
%\newcommand{\xbold}{{\mbox{\boldmath $x$}}}
%\newcommand{\ibold}{{\mbox{\boldmath $i$}}}
%\newcommand{\ubold}{{\mbox{\boldmath $u$}}}
%\newcommand{\vbold}{{\mbox{\boldmath $v$}}}
%\newcommand{\sbold}{{\mbox{\boldmath $s$}}}
%\newcommand{\jbold}{{\mbox{\boldmath $j$}}}
\newcommand{\beqa}{\begin{eqnarray*}}
\newcommand{\eeqa}{\end{eqnarray*}}

\newcommand{\ebshift}[2]{{#1} \! < \!\! < \! {#2}}
\newcommand{\ebaver}[1]{< \!\! {#1} \!\! >}

\newcommand{\MeshRefine}{\textsf{MeshRefine}{}}
\newcommand{\ChomboVersion}{0.99}
\newcommand{\gmake}{{\tt gmake}}
\newcommand{\Chombo}{{\tt Chombo}}
\newcommand{\AN}{[(U \cdot \nabla)U]^{n+\frac{1}{2}}}
\newcommand{\ChFVersion}{1.6}
\newcommand{\ChF}{{\tt ChF}{}}
\newcommand{\chfpp}{{\tt chfpp}}
\newcommand{\cpp}{{\tt cpp}}
\newcommand{\argv}{$<${\em arg}$>$}
\newcommand{\dimv}{$<${\em dim}$>$}
\newcommand{\lenv}{$<${\em len}$>$}
\newcommand{\compv}{$<${\em comp}$>$}
\newcommand{\ncomp}{$<${\em ncomp}$>$}
\newcommand{\arity}[1]{#1.\fnc{arity}{\relax}}
\newcommand{\Beta}{\beta}
\newcommand{\cdh}{{\cal D_H}}
\newcommand{\cd}{{\cal D}}
\newcommand{\cg}{{\cal G}}
\newcommand{\cq}{{\cal Q}}
\newcommand{\clh}{{\cal L_H}}
\newcommand{\cl}{{\cal L}}
\newcommand{\cph}{{\cal P_H}}
\newcommand{\cp}{{\cal P}}
\newcommand{\del}{\nabla}
\newcommand{\deriv}[2]{\mathop{\frac{\partial #1}{\partial #2}}}
\newcommand{\matDeriv}[2]{\mathop{\frac{D #1}{D #2}}}
\newcommand{\display}[1]{\begin{itemize}\item[]#1\end{itemize}}
\newcommand{\domain}[1]{#1.\fnc{domain}{\relax}}
\newcommand{\Domain}[1]{\ifmmode\mbox{\tt Domain<}#1\mbox{\tt>}\else{\tt Domain<$#1$>}\fi}
\newcommand{\dr}{\Delta r}
\newcommand{\ds}{\displaystyle}
\newcommand{\dth}{\Delta \theta}
\newcommand{\dt}{\Delta t}
\newcommand{\dx}{\Delta x}
\newcommand{\dy}{\Delta y}
\newcommand{\eps}{\epsilon}
\newcommand{\ES}{{\tt EdgeStencil}}
\newcommand{\fab}{{\tt FArrayBox}} 
\newcommand{\fnc}[2]{\ifmmode\mbox{\tt#1(}#2\mbox{\tt)}\else{\tt#1(}$#2${\tt)}\fi}
\newcommand{\fourth}{\frac{1}{4}}
\newcommand{\grad}{\nabla}
\newcommand{\half}{\frac{1}{2}}
\newcommand{\hexm}[2]{{$#1 \cdot 10^{-#2}$}}
\newcommand{\IEFab}[1]{\ifmmode\mbox{\tt IEFab<}#1\mbox{\tt>}\else{\tt IEFab<$#1$>}\fi}
\newcommand{\IG}{{\tt IrregGeom}}
\def \ij {{i  ,j    }}
\newcommand{\ijk  }{{i,j,k}}
\newcommand{\ijkmh}{{i,j,k-\half}}
\newcommand{\ijkmo}{{i,j,k-1}}
\newcommand{\ijkph}{{i,j,k+\half}}
\newcommand{\ijkpo}{{i,j,k+1}}
\newcommand{\ijmh }{{i,j-\half  }}
\newcommand{\ijmhk}{{i,j-\half,k}}
\newcommand{\ijmho}{{r,i,j-\half}}
\newcommand{\ijmht}{{\theta,i,j-\half}}
\newcommand{\ijmok}{{i,j-1,k}}
\newcommand{\ijo  }{{r,i  ,j  }}
\newcommand{\ijp  }{{i  ,j+1  }}
\newcommand{\ijphk}{{i,j+\half,k}}
\newcommand{\ijpho}{{r,i,j+\half}}
\newcommand{\ijpht}{{\theta,i,j+\half}}
\newcommand{\ijpo }{{r,i  ,j+1}}
\newcommand{\ijpok}{{i,j+1,k}}
\newcommand{\ijpt }{{\theta,i  ,j+1}}
\newcommand{\ijt  }{{\theta,i  ,j  }}
\newcommand{\imhj }{{i-\half,j  }}
\newcommand{\imhjk}{{i-\half,j,k}}
\newcommand{\imhjo}{{r,i-\half,j}}
\newcommand{\imhjt}{{\theta,i-\half,j}}
\newcommand{\imojk}{{i-1,j,k}}
\newcommand{\imojmok}{{i-1,j-1,k}}
\newcommand{\imojpok}{{i-1,j+1,k}}
\newcommand{\iphj }{{i+\half,j  }}
\newcommand{\iphjk}{{i+\half,j,k}}
\newcommand{\iphjo}{{r,i+\half,j}}
\newcommand{\iphjt}{{\theta,i+\half,j}}
\newcommand{\ipj  }{{i+1,j    }}
\newcommand{\ipjo }{{r,i+1,j  }}
\newcommand{\ipjp }{{i+1,j+1  }}
\newcommand{\ipjpo}{{r,i+1,j+1}}
\newcommand{\ipjpt}{{\theta,i+1,j+1}}
\newcommand{\ipjt }{{\theta,i+1,j  }}
\newcommand{\ipojk}{{i+1,j,k}}
\newcommand{\ipojmok}{{i+1,j-1,k}}
\newcommand{\ipojpok}{{i+1,j+1,k}}
\newcommand{\lacute}{\mathopen{<}}
\newcommand{\la}{\leftarrow}
%\newcommand{\blbox}{{\tt Box}} 
%\newcommand{\boxarray}{{\tt  BoxArray}} 
\newcommand{\boxlib}{{\tt  BoxLib}} 
\newcommand{\BoxLib}{{\tt  BoxLib}} 
%\newcommand{\BranchNode}{{\tt BranchNode}}
%\newcommand{\cfstencil}{{\tt CFStencil}} 
%\newcommand{\BaseFab}[1]{\ifmmode\mbox{\tt BaseFab<}#1\mbox{\tt>}\else{\tt BaseFab<$#1$>}\fi}
%\newcommand{\FabArray}[1]{\ifmmode\mbox{\tt FabArray<}#1\mbox{\tt>}\else{\tt FabArray<$#1$>}\fi}
%\newcommand{\basefab}[1]{\ifmmode\mbox{\tt BaseFab<}#1\mbox{\tt>}\else{\tt BaseFab<$#1$>}\fi}
%\newcommand{\fabarray}[1]{\ifmmode\mbox{\tt FabArray<}#1\mbox{\tt>}\else{\tt FabArray<$#1$>}\fi}
%\newcommand{\ebamrellip}{{\tt EBAMREllip}} 
%\newcommand{\ebamrlevelmg}{{\tt EBAMRLevelMG}} 
%\newcommand{\ebcellfab}{{\tt EBCellFAB}} 
%\newcommand{\ebmulticellfab}{{\tt EBMultiCellFAB}} 
%\newcommand{\ebfacefab}{{\tt EBFaceFAB}} 
%\newcommand{\ebcfinterp}{{\tt EBCFInterp}} 
%\newcommand{\eblevelcfstencil}{{\tt EBLevelCFStencil}} 
%\newcommand{\eblevelfluxregister}{{\tt EBLevelFluxRegister}} 
%\newcommand{\ebmultifacefab}{{\tt EBMultiFaceFAB}} 
%\newcommand{\ebamrsolver}{{\tt EBAMRSolver}} 
%\newcommand{\eblevelsolver}{{\tt EBLevelSolver}} 
%\newcommand{\eblevelop}{{\tt EBLevelOp}} 
%\newcommand{\eblevelmg}{{\tt EBLevelMG}} 
%\newcommand{\eblib}{{\tt EBLib}} 
%\newcommand{\lowvofs}[2]{\mbox{\rm lowVoFs(#1,#2)}} 
%\newcommand{\highvofs}[2]{\mbox{\rm highVoFs(#1,#2)}} 
%\newcommand{\amrlevel}{{\tt AmrLevel}} 
%\newcommand{\amrpoisson}{{\tt AMRPoisson}} 
%\newcommand{\amrremesh}{{\tt MeshRefine}} 
%\newcommand{\amrsolver}{{\tt AMRSolver}} 
%\newcommand{\infab}{{\tt INFab}} 
%\newcommand{\iefab}{{\tt IEFab}} 
%\newcommand{\amr}{{\tt Amr}} 
%\newcommand{\gridcfstencil}{{\tt GridCFStencil}} 
%\newcommand{\gridhoavecfstencil}{{\tt  GridHOAveCFStencil}} 
%\newcommand{\gridfluxregister}{{\tt GridFluxRegister}} 
%\newcommand{\LeafNode}{{\tt LeafNode}}
%\newcommand{\EBIS}{{\tt EBIndexSpace}}
%\newcommand{\ebis}{{\tt EBIndexSpace}}
%\newcommand{\Edge}{{\tt Edge}}
%\newcommand{\hoavecfstencil}{{\tt  HOAveCFStencil}} 
%\newcommand{\oscfstencil}{{\tt OneSideCFStencil}} 
%\newcommand{\onesidepoiss}{{\tt OneSidePoiss}} 
%\newcommand{\oscfinterp}{{\tt OneSideCFInterp}} 
%\newcommand{\domainghostbc}{{\tt DomainGhostBC}} 
%\newcommand{\boxghostbc}{{\tt BoxGhostBC}} 
%\newcommand{\farraybox}{{\tt FArrayBox}} 
%\newcommand{\onesidecfinterp}{{\tt OneSideCFInterp}} 
%\newcommand{\hoacfinterp}{{\tt HOACFInterp}} 
%\newcommand{\fluxregister}{{\tt FluxRegister}} 
%\newcommand{\INFab}[1]{\ifmmode\mbox{\tt INFab<}#1\mbox{\tt>}\else{\tt INFab<$#1$>}\fi}
%\newcommand{\intvectset}{{\tt IntVectSet}} 
%\newcommand{\intvect}{{\tt IntVect}} 
%%\newcommand{\levelcfstencil}{{\tt LevelCFStencil}} 
%\newcommand{\levelhoavecfstencil}{{\tt  LevelHOAveCFStencil}} 
%\newcommand{\levelfluxregister}{{\tt LevelFluxRegister}} 
%\newcommand{\levelmg}{{\tt LevelMG}} 
%\newcommand{\levelop}{{\tt LevelOp}} 
%\newcommand{\levelopfactory}{{\tt LevelOpFactory}} 
%\newcommand{\levelsolver}{{\tt LevelSolver}} 
%\newcommand{\amrlevelmg}{{\tt AMRLevelMG}} 
%\newcommand{\meshrefine}{{\tt MeshRefine}} 
%\newcommand{\multifab}{{\tt MultiFab}} 
\newcommand{\nmh}{{n - \half}}
\newcommand{\nochapters}{%                        %%% Use this when you're not using \chapter commands.
    \renewcommand \thesection {\arabic{section}}  %%% Redefine section headings to leave out the chapter number.
    \newpage                                      %%% Force a page break because there is
}                                                 %%%  no \chapter command to do it for us.
\newcommand{\Node}{{\tt Node}}
\newcommand{\nph}{{n + \half}}
\newcommand{\NS}{{\tt NodeStencil}}
\newcommand{\pargraph}[1]{\par\noindent{\bf #1.}}
\newcommand{\parmparse}{{\tt ParmParse}} 
\newcommand{\Point}[1]{\ifmmode\mbox{\tt Point<}#1\mbox{\tt>}\else{\tt Point<$#1$>}\fi}
\newcommand{\racute}{\mathclose{>}}
\newcommand{\ra}{\rightarrow}
\newcommand{\rdh}{{\rm D_H}}
\newcommand{\rdo}{{\rm D_O}}
\newcommand{\rd}{{\rm D}}
\newcommand{\RectDomain}[1]{\ifmmode\mbox{\tt RectDomain<}#1\mbox{\tt>}\else{\tt RectDomain<$#1$>}\fi}
\newcommand{\rgo}{{\rm G_O}}
\newcommand{\rg}{{\rm G}}
\newcommand{\rimo}{(R_{out}-R_{in})}
\newcommand{\rlh}{{\rm L_H}}
\newcommand{\rl}{{\rm L}}
\newcommand{\romi}{(R_{out}-R_{in})}
\newcommand{\rph}{{\rm P^H}}
\newcommand{\rpo}{{\rm P^O}}
\newcommand{\rp}{{\rm P}}
\newcommand{\sign}[1]{\ifmmode\mbox{sign}(#1)\else sign(#1)\fi}
\newcommand{\Vector}[1]{\ifmmode\mbox{\tt Vector<}#1\mbox{\tt>}\else{\tt Vector<#1>}\fi}
\newcommand{\vu}{\vec u}

\newcommand{\bi}{\begin{itemize}}
\newcommand{\ei}{\end{itemize}}
\newcommand{\I}{\item}
\newcommand{\D}{\begin{itemize} \item[]}
\newcommand{\bv}{\begin{verbatim}}
\newcommand{\ev}{\end{verbatim}}
\newcommand{\bt}{\bf \tt}

\newcommand{\ijph}{{i,j+\half}}
\newcommand{\ipoj}{{i+1,j}}
\newcommand{\imoj}{{i-1,j}}

\newcommand{\iphjph}{{i+\half,j+\half}}
\newcommand{\iphjmh}{{i+\half,j-\half}}
\newcommand{\imhjph}{{i-\half,j+\half}}
\newcommand{\imhjmh}{{i-\half,j-\half}}
\newcommand{\ipojpo}{{i+1,j+1}}
\newcommand{\ipojmo}{{i+1,j-1}}
\newcommand{\imojpo}{{i-1,j+1}}
\newcommand{\imojmo}{{i-1,j-1}}
\newcommand{\dxot}{\frac {\triangle x}{2}}
\newcommand{\dyot}{\frac {\triangle x}{2}}
\newcommand{\gradxi}{\nabla_\xi}
%\newcommand{\del}{{\nabla \cdot}}
\newcommand{\delxi}{{\nabla_\xi \cdot}}
\newcommand{\AH}{{\del (U \otimes U)^\nph}}
%\renewcommand{\theequation}{\arabic{section}.\arabic{equation}}

%\def\eval#1#2{\left. {#1} \right|_{#2}}
%\def\deriv#1#2{\frac{\partial #1}{\partial #2}}
%\def\derivv#1#2{\frac{\partial^2 #1}{\partial #2^2}}
%\def\derivvv#1#2{\frac{\partial^3 #1}{\partial #2^3}}
%\def\derivvvv#1#2{\frac{\partial^4 #1}{\partial #2^4}}
%\newcommand{\tderiv}[2]{\mathop{\frac{d #1}{d #2}}}
%\def\l({\left(}
%\def\r){\right)}
%\def\prt{\partial}
%\def\appx{\sim}
%\def\inorm#1{\| #1 \|_\infty}
%\def\L2norm#1{\| #1 \|_2}
%%\def\dspace{\vspace{50pt}}
%\def\dspace#1{\noalign{\vskip#1}}
%\def\eqspace{\dspace{8pt}}

%\newcommand{\draft}{
%\special{!userdict begin /bop-hook{gsave 200 30 translate
%65 rotate /Times-Roman findfont 216 scalefont setfont
%0 0 moveto 0.85 setgray (DRAFT) show grestore}def end}
%}


%\def\p{\partial}
%\def\hgapmed{\hspace{25pt}}
%\def\hgapsm{\hspace{15pt}}
%\def\be{ \begin{equation} }
%\def\ee{ \end{equation} }
%\def\bea{ \begin{eqnarray} }
%\def\eea{ \end{eqnarray} }
%\def\beas{ \begin{eqnarray*} }
%\def\eeas{ \end{eqnarray*} }
%\def\dd#1#2{\frac{\partial #1}{\partial #2}}
%\def\({\left(}
%\def\){\right)}
%\def\[{\left[}
%\def\]{\right]}
%\def\u#1#2#3{U_{#1,#2}^{#3}}
%\def\v#1#2#3{V_{#1,#2}^{#3}}
%\def\vtxt#1{V_{{\rm #1}}}
%\def\uh#1#2#3{\hat{U}_{#1,#2}^{#3}}
%\def\vh#1#2{\hat{V}_{#1,#2}}
%\def\utilde{\tilde{U}}
%\def\uT{U^{\rm T}}
%\def\D{\Delta}
%\def\vec#1{{\bf #1}}
%\def\vecx{{\bf x}}
%\def\avg{{\rm avg}}
%\def\calF{{\cal F}}
%\def\calU{{\cal U}}
%\def\calN{{\cal N}}
%\def\calL{{\cal L}}
%\def\cvmgp{{\rm cvmgp}}
%\def\cvmgz{{\rm cvmgz}}
%\def\lamo{\lambda_o}
%\def\lams{\lambda^*}
%\def\dfrac{\displaystyle\frac}
%\def\sign{{\rm sign}}
%\def\O{\Omega}
%\def\ol{\Omega^\ell}
%\def\supers#1{\raise+0.7ex\hbox{\ninerm #1}}

%\def\Um#1#2{{}^{#1}U^{#2}}
%\def\Fm#1#2{{}^{#1}{\cal F}^{#2}}

\newcommand{\vel}{\vec u}

%\newcommand{\Lamdba}{\Lambda}
\newcommand{\alfven}{Alfv\'{e}n }



\bibliographystyle{plain}

% Use the option doublespacing or reviewcopy to obtain double line spacing
% \documentclass[doublespacing]{elsart}


% if you use PostScript figures in your article
% use the graphics package for simple commands
% \usepackage{graphics}
% or use the graphicx package for more complicated commands
% \usepackage{graphicx}
% or use the epsfig package if you prefer to use the old commands
\usepackage{epsfig}

% The amssymb package provides various useful mathematical symbols
\usepackage{amssymb}
\usepackage{amsmath}
% -*- Mode: TeX; Modified: "Mon 05 Apr 1999 22:39:15 by dbs"; -*- 

\bibliographystyle{plain}

\oddsidemargin=.25in
\evensidemargin=.25in
\textwidth=6.0in
\topmargin=-0.5in
\textheight=8.5in

%%% `captions.sty' overrides the default figure captions layout.
%%% These variables are used in this style file.
\setlength{\abovecaptionskip}{10pt}
\setlength{\belowcaptionskip}{0pt}

%%% force \subsubsection to be numbered and appear in the table of contents
%%% in report style
\setcounter{secnumdepth}{3}
\setcounter{tocdepth}{3}

%%% In `article' style, remove the chapter number from section numbers and force a
%%% page break at the end of the table of contents.
%%% In `report' style, force \chapter to print ``Part'' instead of ``Chapter'' in headings.
%%dbs odd %% \ifx\chaptername\undefined
%%dbs odd %%   %%%article style has no \chaptername
%%dbs odd %%   \renewcommand{\thesection}{\arabic{section}}%  %%dont put chapter number in the section number
%%dbs odd %%   \renewcommand{\tableofcontents}{%
%%dbs odd %%     \section*{\contentsname
%%dbs odd %%         \@mkboth{%
%%dbs odd %%            \MakeUppercase\contentsname}{\MakeUppercase\contentsname}}%
%%dbs odd %%     \@starttoc{toc}%
%%dbs odd %%     \newpage%
%%dbs odd %%   }
%%dbs odd %% \else
%%dbs odd %%   \renewcommand{\chaptername}{Part}             %%%report style
%%dbs odd %% \fi
%%dbs odd %% 
%%% Macro, variable and command (re)definitions 

\input epsf
%%\input latexsym


%%yikes
%\newcount\ncellsx \newcount\ncellsy   % number of cells in base box
%\newcount\cellsize                    % size of each cell in picture units
%\newcount\halfcell                    % cellsize / 2
%\newcount\circlesize                  % size of base level circles
%\newcount\x \newcount\y               % scratch variables
%
%%%
%%% command: withBaseBox
%%% purpose: draw a picture with a base box and allow other
%%%          Box drawing commands to be executed on the base box
%%% arg1: unit of length for picture (should include length type
%%%       (in,mm,cm,pt,...)), a cell is \cellsize * this value
%%% arg2: number of cells in X (horizontal) direction
%%% arg3: number of cells in Y (vertical) direction
%%% arg4: other drawing commands
%%% NOTE: this opens and closes a picture and defines counters:
%%%       \ncellsx, \ncellsy, \cellsize, \halfcell, \circlesize
%%%
%\newcommand{\withBaseBox}[4]{{
%  % set up constants for this picture
%  \setlength{\unitlength}{#1} \ncellsx=#2 \ncellsy=#3
%  \cellsize=100   %cell size in \unitlength's
%  \halfcell=\cellsize \divide\halfcell by 2
%  \circlesize=\cellsize \multiply\circlesize by 2 \divide\circlesize by 5
%  % make the picture size a little larger than the grid itself
%  \count101=\ncellsx \multiply\count101 by \cellsize \advance\count101 by 10
%  \count102=\ncellsy \multiply\count102 by \cellsize \advance\count101 by 10
%  \begin{picture}(\count101,\count102)
%  % this computes the grid size and draws the grid
%  \count101=\ncellsx \multiply\count101 by \cellsize
%  \count102=\ncellsy \multiply\count102 by \cellsize
%  \put(0,0){\grid(\count101,\count102)(\cellsize,\cellsize)}
%  #4
%  \end{picture}
%}}
%
%%%
%%% command: drawBox
%%% purpose: after a picture has been opened, draw a box with the specified
%%%          lower and upper corners at the specified location with the
%%%          specified refinement ratio
%%% arg1: lower X index (Level 0 index space)
%%% arg2: lower Y index ( " )
%%% arg3: upper X index ( " )
%%% arg4: upper Y index ( " )
%%% arg5: refinement ratio
%%% NOTE: this must be preceded by a \drawBaseBox{}; it leaves the picture open.
%%%
%\newcommand{\drawBox}[5]{{
%  % set the lower corner
%  \x=#1 \multiply\x by \cellsize
%  \y=#2 \multiply\y by \cellsize
%  % set the grid size (upper-lower corners) (in the Level 0 index space)
%  \count151=#3 \advance\count151 by -#1 \advance\count151 by 1 \multiply\count151 by \cellsize
%  \count152=#4 \advance\count152 by -#2 \advance\count152 by 1 \multiply\count152 by \cellsize
%  % refine the grid
%  \count153=\cellsize \divide\count153 by #5
%  % draw the refined grid
%  \put(\x,\y){\grid(\count151,\count152)(\count153,\count153)}
%  % draw a thick outline of the box if the refinement ratio >1
%  \ifnum #5>1
%   { \thicklines\put(\x,\y){\grid(\count151,\count152)(\count151,\count152)}}
%  \fi
%}}
%
%%%
%%% command: drawTag
%%% purpose: draw a Tag in the cell at the specified indices corresponding to
%%%          the specified refinement ratio
%%% arg1: X index of tag
%%% arg2: Y index of tag
%%% arg3: refinement ratio
%%% NOTE: this must be preceded by a \drawBaseBox{}; it leaves the picture open.
%%%
%\newcommand{\drawTag}[3]{{
%  \divide \cellsize by #3 \divide\halfcell by #3
%  \divide\circlesize by #3
%  \x=#1 \multiply\x by \cellsize \advance\x by \halfcell
%  \y=#2 \multiply\y by \cellsize \advance\y by \halfcell
%  \put(\x,\y){\circle*{\circlesize}}  %circle* draws a filled circle
%}}


% Convenient stuff.
\newcommand{\ib}{{\bf i}}
\newcommand{\jb}{{\bf j}}
\newcommand{\kb}{{\bf k}}
\newcommand{\eb}{{\bf e}}
\newcommand{\dt}{{\Delta t}}
\newcommand{\dx}{{\Delta x}}
\newcommand{\nph}{{n + \frac{1}{2}}}
\newcommand{\ddt}[1]{\frac{\partial #1}{\partial t}}
\newcommand{\iddt}[1]{\partial #1/\partial t}
\renewcommand{\vec}[1]{\mathbf{#1}}
\newcommand{\tens}[1]{\mathbf{#1}}
\newcommand{\grad}[1]{\nabla{#1}}
\newcommand{\diverg}[1]{\nabla\cdot{#1}}
\newcommand{\ddxi}[1]{\frac{\partial #1}{\partial x_i}}
\newcommand{\ddxj}[1]{\frac{\partial #1}{\partial x_j}}
\newcommand{\ddxk}[1]{\frac{\partial #1}{\partial x_k}}
\newcommand{\ddxd}[1]{\frac{\partial #1}{\partial x_d}}
\newcommand{\iddxi}[1]{\partial #1/\partial x_i}
\newcommand{\iddxj}[1]{\partial #1/\partial x_j}
\newcommand{\iddxk}[1]{\partial #1/\partial x_k}
\newcommand{\iddxd}[1]{\partial #1/\partial x_d}
\newcommand{\deltaij}{\delta_{ij}}
\newcommand{\sigmaij}{\sigma_{ij}}
\newcommand{\Lmu}{\mathcal{L}_\mu}
\newcommand{\LK}{\mathcal{L}_K}
\newcommand{\refEq}[1]{(\ref{eq:#1})}
\newcommand{\refSec}[1]{Section \ref{sec:#1}}
\newcommand{\refApp}[1]{Appendix \ref{app:#1}}


\begin{document}

\begin{frontmatter}

% Title, authors and addresses

\title{A Cartesian Grid Embedded Boundary Method for the Compressible
  Navier Stokes Equations}

% use optional labels to link authors explicitly to addresses:
% \author[label1,label2]{}
% \address[label1]{}
% \address[label2]{}

\author{Phillip Colella, \thanksref{lbl}}
\author{Daniel T. Graves,\thanksref{lbl}}
\author{Bjorn Sjogreen,  \thanksref{lbl}}
\author{Jeffrey Johnson  \thanksref{lbl}}
\thanks[lbl]{Research supported at the Lawrence Berkeley National
Laboratory by the U.S. Department of Energy: Director, Office of
Science, Office of Advanced Scientific Computing, Mathematical,
Information, and Computing Sciences Division under Contract
DE-AC03-76SF00098.}
\thanks[ben]{Research supported by the Computational Science Graduate Fellowship program of the Department of Energy, under grant number DE-FG02-97ER25308.}
% \ead[url]{home page}
\address{Lawrence Berkeley National Laboratory, Berkeley, CA\thanksref{lbl}}
\address{Mathematics Department, University of Michigan, Ann Arbor, MI
\thanksref{ben}}


\begin{abstract}
We present a second-order Godunov algorithm to solve the
compressible Navier-Stokes equations.
\end{abstract}

\begin{keyword}
% keywords here, in the form: keyword \sep keyword

% PACS codes here, in the form: \PACS code \sep code

\end{keyword}

\end{frontmatter}

\section{Introduction}

In this paper, we extend the unsplit embedded boundary method for hyperbolic 
conservation laws described in \cite{???}, allowing the addition of elliptic 
operators for advection-diffusion processes. We use these elliptic terms 
to represent the viscous and conductive terms in the compressible 
Navier-Stokes equations.

This paper is organized as follows. In \refSec{Equations} we discuss the 
compressible Navier-Stokes equations and relate them to their more general 
advection-diffusion conservation laws. In \refSec{Evolution} we outline an 
algorithm for treating the hyperbolic and elliptic terms, integrating them 
separately for stability. \refSec{DivF} describes the process by which the 
hyperbolic (transport) terms are treated in greater detail, whereas the 
elliptic (diffusion) terms are discussed in \refSec{Diffusion}. We explain 
some important details of the implementation in \refSec{Implementation}. In 
\refSec{Results}, we demonstrate the performance of the algorithm and its 
implementation on some interesting problems in gasdynamics.



\section{Governing Equations\label{sec:Equations}}

We are concerned with solving systems of advection-diffusion equations on 
irregular domains. A system of this sort can be written as

\begin{equation}
\ddt{U} + \diverg{\vec{F}} = \mathcal{L}(U) \label{eq:advectionDiffusion}
\end{equation}

\noindent
where $U = U(\vec{x}, t)$ is a solution vector, $\vec{F} = \vec{F}(U)$ is an 
advective flux, and $\mathcal{L}$ is a linear diffusion operator. The presence 
of diffusion (elliptical) terms in such a system presents a problem for the
explicit time integration methods often used in advective (hyperbolic) systems, 
since discretizations containing these terms have time step constraints 
that scale with $\dx^2$ (where $\dx$ is the grid spacing). Our goal is to 
treat the elliptical terms implicitly while integrating the hyperbolic terms
explicitly.

We illustrate this implicit/explicit method by solving the compressible 
Navier-Stokes equations. These conservation laws describe the behavior of a 
compressible fluid with thermal conduction and viscosity. Written
in conservative form, the Navier-Stokes equations are

\begin{eqnarray}
\ddt{\rho} + \diverg{(\rho \vec{v})} &=& 0 \notag \\ 
\ddt{(\rho \vec{v})} + \diverg{(\rho \vec{v}\vec{v})} &=& -\grad{p} + \diverg{\tens{\sigma}} \label{eq:CNS} \\
\ddt{(\rho E)} + \diverg{(\rho E \vec{v})} &=& -\diverg{(p\vec{v})} + \rho\diverg{\vec{Q}} \notag
%\ddt{\rho} + \ddxi{(\rho v_i)} &=& 0 \label{} \\
%\ddt{(\rho v_i)} + \ddxi{(\rho v_i v_j)} &=& -\ddxi{p} + \ddxi{\sigmaij} \\
%\ddt{(\rho E)} + \ddxi{(\rho v_i E)} &=& -\ddxi{(v_i p)} + \rho\ddxi{Q_i}
\end{eqnarray}

\noindent
These equations are (respectively) the laws of conservation of mass, momentum, 
and energy. Above, $\rho$ is the mass density of the fluid, $\vec{v}$ is its 
bulk velocity, $p$ its pressure, and $E$ its total energy per unit volume. 
$\tens{\sigma}$ is the viscous stress tensor representing the dissipation of 
kinetic energy into heat, and $\vec{Q}$ is the flow of heat within the fluid. 

In a Newtonian fluid, the viscous stress can be expressed in terms of the 
viscosity coefficients $\mu$ and $\lambda$ and the strain rate tensor 
$\tens{e}$ whose components are defined by $e_{ij} = (\iddxj{v_i} + \iddxi{v_j})/2$\cite{Bachelor}. 
In Einstein notation, in which repeated indices are summed:

\begin{eqnarray}
\sigmaij &=& \mu\left(\ddxj{v_i} + \ddxi{v_j}\right) + \lambda \deltaij \ddxk{v_k} \notag \\
         &=& 2\mu e_{ij} + \lambda \deltaij e_{kk} \label{eq:viscTensor}
\end{eqnarray}

\noindent
$\mu$ represents the rate at which shear flows generate heat at the expense of 
the fluid's mechanical energy. This parameter can depend upon the pressure 
and temperature of the fluid and thus is allowed to vary spatially. $\lambda$,
often called the \em second viscosity\em, is the rate at which this conversion 
occurs in the presence of compression, It is customary to prescribe 
$\lambda$ in terms of $\mu$ in order to make $\tens{\sigma}$ traceless:

\begin{equation}
\lambda = -\frac{2}{3}\mu \label{eq:lambda}
\end{equation}

\noindent
We adopt this practice for the present study. In reality, $\lambda$ can depend 
on the frequencies of compression waves, so its behavior can be much more 
complicated\cite{LandauFM}.

The heat flow $\vec{Q}$ is related to gradients in the fluid's temperature $T$ 
by the thermal conductivity $K$ of the fluid, which can also vary in space:

\begin{equation}
\vec{Q} = K \grad{T} \label{eq:heatFlow}
\end{equation}

When \refEq{viscTensor} and \refEq{heatFlow} are used in \refEq{CNS}, they 
result in diffusion terms within the momentum and energy equations. 
We can rearrange the resulting equations to place them into a form resembling 
\refEq{advectionDiffusion}:

\begin{eqnarray}
\ddt{\rho} + \ddxi{(\rho v_i)} &=& 0 \label{eq:continuity} \\
\ddt{(\rho v_i)} + \ddxj{(\rho v_i v_j + p\deltaij)} &=& 
   \ddxj{}\left(\mu\left[\ddxj{v_i} + \ddxi{v_j} - \frac{2}{3}\deltaij\ddxk{v_k}\right]\right) \label{eq:momentum} \\
\ddt{(\rho E)} + \ddxi{(\rho E v_i + p v_i)} &=& \rho\ddxi{}\left(K\ddxi{T}\right) \label{eq:energy} 
\end{eqnarray}

\noindent
To express \refEq{continuity} - \refEq{energy} in the form of 
\refEq{advectionDiffusion}, we identify the solution vector $U$ and the flux 
$F^d$ in the $d$th direction:

\begin{eqnarray}
U &=& (\rho, \rho\vec{v}, \rho E)^T \label{eq:solnVector} \\
F^d(U) &=& (\rho v^d, \rho v^d\vec{v} + p\tens{e}^d,\rho v^d E + v^d p)^T \label{eq:Fd}
\end{eqnarray}

\noindent
Since the diffusion terms on the right hand sides of \refEq{momentum} and 
\refEq{energy} do not couple the conserved quantites, the momentum and energy 
equations are not coupled in the operator $\mathcal{L}$. This means we are 
allowed to treat the diffusion terms separately, defining the a viscous 
operator $\Lmu$ and a thermal conduction operator $\LK$. Because $\Lmu$ and 
$\LK$ are decoupled, each operates on its respective primitive variable and 
not on $U$. The specific forms of these operators can be found by ignoring 
the advection terms in \refEq{momentum} and \refEq{energy}.  For instance, the 
dissipation in the momentum of an element of fluid with mass density $\rho$ 
due to viscous heating is

\begin{eqnarray}
\left(\ddt{(\rho v_i)}\right)_{\mu} &=& \ddxj{}\left(\mu\left[\ddxj{v_i} + \ddxi{v_j} - \frac{2}{3}\deltaij\ddxk{v_k}\right]\right) \notag \\
                                    &\equiv& \Lmu(\vec{v}). \label{eq:Lmu}
\end{eqnarray}

\noindent
For the energy equation, we need to relate the fluid's temperature $T$ to its 
energy per unit mass, or \em specific energy\em, $E$. If, under no compression, 
it takes $c_v$ units of energy to increase the temperature of a unit of mass of 
fluid by a single unit, that fluid is said to have a specific heat (at constant 
volume) $c_v$. The thermal specific energy of a unit mass of such a fluid is 
$E_{therm} = c_v T$.  We then express the thermal diffusion equation as

\begin{eqnarray}
\left(\ddt{(\rho c_v T)}\right)_K &=& \rho\ddxi{}\left(K\ddxi{T}\right) \notag \\
                                      &\equiv& \LK(T). \label{eq:LK}
\end{eqnarray}

\noindent
For simplicity, we assume that $c_v$ is constant within the fluid.  


\section{Evolution of Equations\label{sec:Evolution}}

In this section we describe the time integration algorithm schematically, 
giving details of each step of the calculation in later sections.
At the beginning of a time step $n$, we integrate the hyperbolic terms in 
\refEq{continuity} - \refEq{energy} explicitly, treating the elliptic terms in 
\refEq{momentum} and \refEq{energy} as sources. To compute the source 
contributions, we evaluate the discrete form of \refEq{LU} within \refEq{advectionDiffusion} at
the current timestep.  In this language, the operator $\mathcal{L}$ can 
be written

\begin{equation}
\mathcal{L}(U) = [0, \Lmu(\vec{v}), \LK(T)]^T. \label{eq:LU}
\end{equation}

\noindent
We denote the discrete form of this operator evaluated at time step $n$ on the 
VoF $\ib$ as $L(U^n_\ib)$, which consists of the separate discrete Helmholtz 
operators $L_\mu(\vec{v}^n_\ib)$ and $L_K(T^n_\ib)$. We represent the 
elliptic terms as a source term in \refEq{advectionDiffusion}:

\begin{equation}
S^n_\ib = L(U^n_\ib) = [0, L_\mu(\vec{v}^n_\ib), L_K(T^n_\ib)]^T \label{eq:hyperbolicSources}
\end{equation}

The details of the evaluation of these ``hyperbolic sources" are given in 
\refSec{HyperbolicSources}.

Next, we must calculate the flux divergence 
$(\diverg{\vec{F}})^{n+1/2}_\ib = \diverg{F}(U^{n+1/2}_\ib, S^n_\ib)$,
incorporating the source terms computed above. This step is similar to the 
calculation of the flux divergence described in \cite{???} for solving the 
Euler equations; we review it in \refSec{FluxDivergence}. 

When we have computed the flux divergence, we can predict a value for 
$U^{n+1}_\ib$ using \refEq{advectionDiffusion}:

\begin{equation}
U^{n+1,*}_\ib = U^b_\ib - \dt(\diverg{\vec{F}})^{n+\half}_\ib
\end{equation}

\noindent
We ignore the implicit terms in \refEq{advectionDiffusion} in this explicit 
advance, apart from incorporating them as sources in the calculation of the 
flux divergence. We then update the momentum and energy in $U$ by solving the 
appropriate diffusion equations and backing out time derivatives. For example, 
implicitly integrating the viscous contribution to the momentum equation 
produces a value for $\vec{v}^{n+1}_\ib$, which can be used to compute 
$L_\mu(\vec{v}^{n+1/2}_\ib)$:

\begin{equation}
L_\mu(\vec{v}^{n+\half}_\ib) = \frac{\rho^{n+\half}(\vec{v}^{n+1} - \vec{v}^n)}{\dt}
\end{equation}

\noindent
Similarly, the implicit treatment of thermal conduction for the energy equation
gives us $T^{n+1}_\ib$, which is used to compute $L_K(T^{n+1/2}_\ib)$:

\begin{equation}
L_K(T^{n+\half}_\ib) = \frac{\rho^{n+\half} c_v (T^{n+1} - T^n)}{\dt}
\end{equation}

The integration of these equations is described in \refSec{Diffusion}. 

Finally, the solution variable is updated with the diffusive contributions:

\begin{equation}
U^{n+1}_\ib = U^{n+1,*} + \dt L(U^{n+\half}_\ib) \label{eq:Uupdate}
\end{equation}

\noindent
or, in terms of the conserved quantities:

\begin{eqnarray}
(\rho\vec{v})^{n+1}_\ib &=& (\rho\vec{v})^{n+1,*} + \dt L_\mu(\vec{v}^{n+\half}_\ib) \label{eq:rhovupdate} \\
(\rho E)^{n+1}_\ib &=& (\rho E)^{n+1,*} + \dt L_K(T^{n+\half}_\ib) \label{eq:rhoEupdate}
\end{eqnarray}


\section{The Flux Divergence\label{sec:DivF}}

We begin with the calculation of the flux divergence, which we need to 
evolve the hyperbolic terms. The details of this procedure were given 
originally in \cite{???}; we include them here for a self-contained 
discussion. 

\subsection{Stabilized conservative update}
The discrete divergence theorem \refEq{discreteDivTheorem} gives us an 
approximation of $\diverg{\vec{F}}$ on a VoF $\ib$:

\begin{equation}
(\diverg{\vec{F}})^C = \frac{1}{\kappa_\ib h}
   \sum_{d=1}^D \left(\pm\alpha_{\ib\pm\half\eb^d}F^d_{\ib\pm\half\eb^d} + 
                      \alpha_\ib^B\vec{F}_\ib^B\cdot\vec{n}_\ib\right) \label{eq:conservativeDivF}
\end{equation}

\noindent
This approximation is conservative in the sense that the explicit finite 
difference update $U^{n+1}_\ib = U^n_\ib - \dt (\diverg{\vec{F}})^C_\ib $ exactly 
conserves each of the quantities in $U$ in a manner that is consistent with 
the behavior of $\vec{F}$ at the boundary. %That is,
%
%\begin{equation}
%\sum_{\ib\in\Gamma}\kappa_\ib U^{n+1}_\ib = 
%   \sum_{\ib\in\Gamma}\kappa_\ib U^n_\ib - 
%   \frac{\dt}{h}\sum_{\ib\pm\half\eb^d\in\partial\Gamma} 
%      \alpha_{\ib\pm\half\eb^d}\vec{F}_{\ib\pm\half\eb^d}\cdot\vec{n}_{\ib\pm\half\eb^d}
%\end{equation}
%
%\noindent
%where $\Gamma$ is a collection of control volumes and $\partial\Gamma$ is 
%the set of cell faces and boundary faces that compose the boundary of $\Gamma$.
We have used the superscript $C$ in \refEq{conservativeDivF} to highlight
this desireable property. Unfortunately, this approximation carries with it a
severe CFL constraint on the time step on irregular cells:

\begin{equation}
\dt = \min \frac{h}{|\vec{v}_\ib|}\left(\kappa_\ib\right)^{\frac{1}{D}}. \label{eq:smallCellCFL}
\end{equation}

\noindent
This coupling of the time step to the volume fraction on irregular cells is 
known as the the small-cell problem for embedded boundary methods. An effective 
way to avoid this problem is to update $U$ using a weighted average of 
conservative and non-conservative approximations for $\diverg{\vec{F}}$, 
denoted respectively by superscripts $C$ and $NC$:

\begin{equation}
U^{n+1}_\ib = U^n_\ib - \dt\left[\eta_\ib(\diverg{\vec{F}})^C_\ib +
                                 (1 - \eta_\ib)(\diverg{\vec{F}})^{NC}_\ib\right] \label{eq:stableUpdate}
\end{equation}

\noindent
where the stable, non-conservative approximation is the flux difference

\begin{equation}
(\diverg{\vec{F}})^{NC}_\ib = \frac{1}{h}\sum_{d=1}^D\pm F^d_{\ib\pm\half\eb_d} \label{eq:nonconservativeDivF}
\end{equation}

If the weight parameter $\eta_\ib$ is set to $\kappa_\ib$, then the 
corresponding factor in the denominator of \refEq{conservativeDivF} is 
cancelled and the CFL constraint is eased. However, this means that $U$ is 
not updated conservatively, as it relies upon the non-conservative 
approximation $(\diverg{\vec{F}})^{NC}$. We can obtain the conservation error, 
$\delta M_\ib$ (the global change in the conserved quantities $U$), from the 
difference in the updates between the conserved and non-conserved forms of 
$\diverg{\vec{F}}$:

\begin{equation}
\delta M_\ib = -\kappa_\ib(1 - \eta_\ib)\left[(\diverg{\vec{F}}^C_\ib - \diverg{\vec{F}}^{NC}_\ib\right]
\end{equation}

\noindent
Once the update has been performed on all VoFs, we distribute each $\delta M_\ib$ 
amongst the set of its neighboring VoFs $N(\ib)$ to restore conservation:

\begin{eqnarray}
U^{n+1}_\jb \rightarrow U^{n+1}_\jb + w_{\ib,\jb}\delta M_\ib, \jb\in N(\ib)
\end{eqnarray}

\noindent
where $w_{\ib,\jb}$ is a distribution function that conservatively partitions 
the contributions to $\jb$ from $\ib$. One strategy that seems to 
perform well for problems in gas dynamics involving shocks is to favor 
neighboring VoFs with higher densities in order to minimize the effects of the 
redistribution on the solution:

\begin{equation}
w_{\ib, \jb} = \frac{\rho_\jb^{NC}}{\sum_{\kb\in N(\ib)}\rho_\kb^{NC}\kappa_\kb}
\end{equation}

\noindent 
where $\rho^{NC}_\ib = \rho^n_\ib - \dt(\diverg{\vec{F}})^{NC}_\ib$ is a 
non-conservative estimate of the mass density at the new time.

\subsection{Edge-centered fluxes}

To compute $(\diverg{\vec{F}})^C$, we need a second-order accurate approximation 
of $\vec{F}$ on the faces of a cell. For a VoF $\ib$,

\begin{equation}
F^{n+\half}_{\ib\pm\half\eb_d} = F(U^{n+\half}_{\ib\pm\half\eb_d}) \label{eq:edgeCenteredFluxes}
\end{equation}

\noindent
where we use the upstream-centered Taylor expansion

\begin{equation}
U^{n+\half}_{\ib\pm\half\eb_d} = U^n_\ib + \frac{\dx}{2}\left(\ddxd{U}\right)^n_\ib + 
                                 \frac{\dt}{2}\left(\ddt{U}\right)^n_\ib. \label{eq:edgeCenteredU}
\end{equation}

\noindent
Here, $(\iddt{U})^n_\ib = L(U^n_\ib)$.

\subsection{Inclusion of elliptic terms as sources}

\section{Diffusion\label{sec:Diffusion}}

Unlike the Euler equations, the Navier-Stokes equations include the effects 
of viscosity and thermal heat exchange in fluids. In this section we describe 
the elliptical operators that represent these contributions and describe how 
they are integrated with respect to their hyperbolic counterparts.

\subsection{Viscous Tensor and Thermal Conduction Operators}
The viscous term in \refEq{momentum} is 

\begin{equation}
\ddxj{}\left(\mu\left[\ddxj{v_i} + \ddxi{v_j}\right] + \lambda\deltaij\ddxk{v_k}\right)
\end{equation}

\noindent
and may be expressed in terms of the linear elliptical operator $\Lmu$ that 
operates on the 3-dimensional velocity vector $\vec{v}$:

\begin{equation}
\Lmu(\vec{v}) = \alpha \tens{I} + \beta\diverg{\left(\mu\left[\grad{\vec{v}} + 
                   \grad{\vec{v}}^T\right] + \lambda (\diverg{\vec{v}})\tens{I}\right)} \label{eq:Lmu}
\end{equation}

\noindent
where $\alpha$ and $\beta$ are time integration parameters (discussed in 
\refApp{TGA}), $\tens{I}$ is the $3\times 3$ identity matrix, $\grad{\vec{v}}$ 
and $\grad{\vec{v}}^T$ are matrices representing the deformation gradient, and
$\diverg{\vec{v}}$ is a scalar representing the velocity divergence.  We refer 
to the discrete form of this operator as $L_\mu$. In this discrete form, the 
spatial derivatives of the velocity are approximated with the finite volume 
method we have mentioned earlier.

The contribution to \refEq{energy} from thermal conduction is

\begin{equation}
\rho\ddxi{}\left(K\ddxi{T}\right).
\end{equation}

\noindent
We can rewrite this term using the linear elliptic operator $\LK$, which 
operates upon the scalar temperature $T$:

\begin{equation}
\LK(T) = \left(\alpha A T + \beta \diverg{\left[B \grad{T}\right]}\right)
\end{equation}

\noindent
where $\alpha$ and $\beta$ are the integration parameters and $B = \rho c_v K$.
\em What is A? Just a generalization? \em The discrete form of $\LK$ is $L_K$, 
in which the spatial derivatives of $T$ are replaced by finite volume 
approximations.

\subsection{Implicit Time Integration}

To avoid the $(\dx)^2$ constraint on the time step from these elliptic 
operators, we integrate their terms implicitly to compute stable updates for 
$\rho\vec{v}$ and $\rho E$. Because we have already computed contributions 
from the hyperbolic terms, we account for these contributions as sources in 
their respective diffusion equations. For example, the viscous contribution to 
the momentum equation is obtained given by integrating

\begin{equation}
\rho_\ib\ddt{\vec{v}_\ib} = L_\mu(\vec{v}_\ib) + \rho_\ib\left(\ddt{\vec{v}_\ib}\right)_{hb} \label{eq:viscContrib}
\end{equation}

\noindent
where $(\iddt{\vec{v}_\ib})_{hb} = -(\vec{v}\cdot\grad{\vec{v}})_\ib - (\grad{p}/\rho)_\ib$ 
represents the hyperbolic portion of $\iddt{\vec{v}}$ on the VoF $\ib$. 
\em Check on this!\em

We integrate \refEq{viscContrib} using an algorithm by Twizell, Gummel and Arign 
\cite{TGA}, which we refer to henceforth as the TGA algorithm. This algorithm 
is second-order accurate and $L^\infty$-stable and has been successfully used 
to treat incompressible Navier-Stokes flows on embedded boundaries\cite{???}.
By contrast, the more well-known Crank-Nicholson algorithm can become unstable 
near boundaries of irregular domains in the presence of the discrete Laplacian 
operator\cite{???}. We outline the TGA algorithm in \refApp{TGA}. 

In any case, integrating \refEq{viscContrib} gives us an updated velocity 
$\vec{v}^{n+1}_\ib$ which we can use to create a stabilized value for 
$L_\mu(\vec{v})$:

\begin{equation}
\hat{L}_\mu(\vec{v}_\ib) \rightarrow \frac{\rho^{n+\half}_\ib(\vec{v}^{n+1}_\ib - \vec{v}^n_\ib)}{\dt} \label{eq:stableLmu}
\end{equation}

\noindent
Then we add the viscous contribution to the momentum:

\begin{equation}
(\rho\vec{v})^{n+1}_\ib \rightarrow (\rho\vec{v})^{n+1}_\ib + \dt \hat{L}_\mu(\vec{v}_\ib) \label{eq:viscUpdate}
\end{equation}

The treatment of the thermal contribution to the energy is similar. The 
rate of heat transfer through thermal contribution in the energy equation is

\begin{equation}
\rho_\ib\ddt{T_\ib} = L_K(T_\ib) + \frac{1}{\rho_\ib}\left(\ddt{\rho_\ib E_\ib}\right)_{hb} \label{eq:thermContrib}
\end{equation}

\noindent
where $(\iddt{\rho E_\ib})_{hb} = -\diverg{([\rho E + p]\vec{v})}_\ib$ 
represents the (hyperbolic) transport of energy on the VoF $\ib$. We integrate
\refEq{thermContrib} to obtain $T^{n+1}$, the updated temperature, which we 
use to stabilize $L_K(T)$:

\begin{equation}
\hat{L}_K(T_\ib) = \frac{\rho^{n+\half}_\ib c_v (T^{n+1}_\ib - T^n_\ib)}{\dt} \label{eq:stableLK}
\end{equation}

\noindent
Finally, we add the thermal contribution to the specific energy:

\begin{equation}
(\rho E)^{n+1}_\ib \rightarrow (\rho E)^{n+1}_\ib + \dt \hat{L}_K(T_\ib) \label{eq:thermUpdate}
\end{equation}

\section{Implementation Details\label{sec:Implementation}}

This section contains important details about our implementation of the 
algorithm we have described.

\subsection{Volume Fraction Normalization for Operators}

A discrete operator $L$ acting upon conserved quantities $U_\ib, \ib \in \Omega$ 
actually computes $\kappa_\ib L(U_\ib)$ on the VoF $\ib$. On regular 
cells, $\kappa_\ib = 1$ and the correct value of $L(U_\ib)$ is obtained. 
However, on each irregular cell, $\kappa_\ib < 1$ and we need to compute an 
average value for $L(U_\ib)$ over a set of surrounding VoFs $\{\jb\}$. If we 
denote the monotone path radius of a VoF $\ib$ by $\mathcal{S}(\ib)$, the 
averaged value of $L(U_\ib)$ is 

\begin{equation}
L(U_\ib) := \kappa_\ib L(U_\ib) + \frac{1-\kappa_\ib}
{(\sum\limits_{\jb \in \mathcal{S}(\ib)} \kappa_\jb)} 
\sum\limits_{\jb \in \mathcal{S}(\ib)} \kappa_\jb L(U_\jb).
\end{equation}

\noindent 
This correction must be applied whenever an operator is evaluated upon 
irregular cells.

\subsection{Adaptive Mesh Refinement, Level Splitting and Refluxing}

In the preceeding sections, we have discussed the approximation of 
continuous equations by discrete representations, but we have made no mention
of Adaptive Mesh Refinement (AMR). That is, we have described our algorithm 
in such a way as to compute quantities on a single level of refinement at a 
time. Specifically, in a grid with several levels of refinement, we step through 
each level and perform our calculations without regard for the other levels of 
refinement. This \em level splitting \em allows us to simplify the 
implementation of the algorithm significantly. However, it means that we must 
separately and explicitly account for interactions between operators at 
different refinement levels. The process of applying inter-level corrections 
to an operator after level splitting is called \em refluxing\em.  Refluxing 
must be done for both the hyperbolic transport operators and the elliptic 
diffusion operators. We describe these two processes presently.

\subsubsection{Hyperbolic terms: explicit refluxing}

\subsubsection{Elliptic terms: implicit refluxing}
Our equations for energy and momentum have elliptic terms.   These
terms need to be advanced implicitly in a way that preserves the
conservation of momentum and energy.   To do this, we advance the
primitive variables to form stable estimates of $L^T(T)$ and $L^v(v)$.
We then use these stable estimates to update energy and momentum.
For momentum we advance the following one step using TGA:
$$
\frac{\partial \rho u}{\partial t} = L^u u + R_u
$$
where 
$$
R_u = -\rho(u \cdot \nabla u) - \nabla p.
$$
We then compute the stable evaluation of $L^v v$:
$$
L^v v = \frac{\rho^{n+\half}(u^{n+1}-u^n)}{\dt}
$$
and update the solution  with the diffusive contribution (the
hyperbolic part has already been added).
$$
(\rho v)^{n+1} += \dt L^v v
$$
Similarly, with energy we solve the following equation for a
temerpature update 
$$
(\rho^{n+\half} C_v I - \dt L^{T}) T = R_T
$$
where
$$
R_T  = -\rho C_v u \cdot \nabla T- u L^v v - p \nabla \cdot v 
$$
and compute the stable evaluation of $L^T T$
$$
L^T = \frac{\rho^{n+\half} C_v(T^{n+1}-T^n)}{\dt} - D^r F^e
$$
$$
(\rho E)^{n+1} += \dt L^T T
$$

\section{Results\label{sec:Results}}

TBD


% The Appendices part is started with the command \appendix;
% appendix sections are then done as normal sections

\appendix
\section{TGA Time Integration \label{app:TGA}}


% \section{}
% \label{}


\bibliography{references}

%\begin{thebibliography}{00}

% \bibitem{label}
% Text of bibliographic item

% notes:
% \bibitem{label} \note

% subbibitems:
% \begin{subbibitems}{label}
% \bibitem{label1}
% \bibitem{label2}
% If there is a note, it should come last:
% \bibitem{label3} \note
% \end{subbibitems}

%\bibitem{}

%\end{thebibliography}

\end{document}

