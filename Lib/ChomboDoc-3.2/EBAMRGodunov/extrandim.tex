%%%%%%%%%%%%%%%%%%%%%%%%%%%%%%
\subsection{Artificial Viscosity} 

The artificial viscosity coefficient
is $K_0$, the velocity is $\vec u$ and $d = dir(\fbold)$.

\begin{align*}
(D \vec u)_\fbold &= (u^d_{\vbold^+ (\fbold)}-u^d_{\vbold^-(\fbold)}) +
\sum_{{d^{'}}\neq d } \frac{1}{2}(\Delta^{d^{'}} u^{d^{'}}_{\vbold^+(\fbold)}
+ \Delta^{d^{'}}u^{d^{'}}_{\vbold^-(\fbold)}) \\
K_\fbold &= K_0~\text{max}(-(D\vec u)_\fbold, 0)  \\
F^{n+\half}_{\fbold} &=  F^{n+\half}_{\fbold} 
- K_\fbold(U^n_{\vbold^+(\fbold)} - U^n_{\vbold^-(\fbold)})  \\
F^{\text{covered}}_{\vbold, \pm, d} &= F^{\text{covered}}_{\vbold, \pm, d}-
K_\fbold(U^n_{\vbold^+(\fbold)} -U^n_{\vbold^-(\fbold)}) 
\end{align*}
We modify the covered face with  the same divergence used
in the adjacent uncovered face.
\begin{align*}
F^{\text{covered}}_{\vbold, \pm, d} =&
F^{\text{covered}}_{\vbold,\pm, d} -
K_\fbold(U^n_{\vbold^+(\fbold)} -U^n_{\vbold^-(\fbold)}) \\
\fbold =& \fbold(\vbold,\mp, d) 
\end{align*}
This has the effect of negating the effect of artificial viscosity 
on the non-conservative divergence of the flux at irregular cells.
We describe later that the solid wall boundary condition at the
embedded boundary is also modified with artificial viscosity.

\section{Slope Calculation} \label{sec::slopeCalculation}

We will use the 4th order slope calculation in Colella and Glaz
\cite{ColellaGlaz} combined with characteristic limiting.  
\begin{align*}
\Delta^d W_\vbold = & \; \zeta_\vbold ~ \widetilde{\Delta}^d W_\vbold \\
\widetilde{\Delta}^d W_\vbold = & \; 
\Delta^{vL}( \Delta^B W_\vbold,\Delta^L W_\vbold,\Delta^R W_\vbold ) ~
| ~ \Delta^d_2 W_\vbold ~ | ~ \Delta^d_2 W_\vbold \\ 
\Delta^d_2 W_\vbold = & \;
\Delta^{vL}(\Delta^C W_\vbold, \Delta^L W_\vbold, \Delta^R
W_\vbold)~|~\Delta^{VLL} W_\vbold~|~\Delta^{VLR} W_\vbold \\ 
\Delta^B W_\vbold = & \; \frac{2}{3}(\ebshift{(W -
\frac{1}{4}\Delta^d_2 W)}{\ebold^d})_\vbold - 
(\ebshift{(W + \frac{1}{4}\Delta^d_2 W)}{-\ebold^d})_\vbold) \\
\Delta^C {W}_\vbold = & \; \half((\ebshift{W^n}{ \ebold^d})_\vbold -
                                 (\ebshift{W^n}{-\ebold^d})_\vbold) \\
\Delta^L {W}_\vbold = & \; W^n_\vbold -
(\ebshift{W^n}{-\ebold^d})_\vbold \\ 
\Delta^R {W}_\vbold = & \; (\ebshift{W^n}{\ebold^d})_\vbold - W^n_\vbold \\
\Delta^{3L} {W}_\vbold = & \; \frac{1}{2}(3 W^n_\vbold -
4 (\ebshift{W^n}{-\ebold^d})_\vbold + (\ebshift{W^n}{-2\ebold^d})_\vbold )\\ 
\Delta^{3R} {W}_\vbold = & \; \frac{1}{2}(-3 W^n_\vbold +
4 (\ebshift{W^n}{\ebold^d})_\vbold -
                                 (\ebshift{W^n}{2\ebold^d})_\vbold)  \\
\Delta^{VLL} W_\vbold =&
  \begin{cases}
  \text{min}(\Delta^{3L} W_\vbold,  \Delta^L_{\vbold}) &
    \text{if $\Delta^{3L} W_\vbold \cdot \Delta^L W_{\vbold} > 0$}
  \\
  0 &
    \text{otherwise} \\
  \end{cases}  \\
\Delta^{VLR} W_\vbold =&
  \begin{cases}
  \text{min}(\Delta^{3R} W_\vbold,  \Delta^R_{\vbold}) &
    \text{if $\Delta^{3R} W_\vbold \cdot \Delta^R W_{\vbold} > 0$}
  \\
  0 &
    \text{otherwise}
  \end{cases} 
\end{align*}
At domain boundaries, $\Delta^L W_\vbold$ and $\Delta^R W_\vbold$ may
be overwritten by the application. There are two versions of the van Leer limiter 
$\Delta^{vL}(\delta W_C, \delta W_L, \delta W_R)$ that are commonly used. 
One is to apply a limiter to the differences in characteristic
variables.
\begin{enumerate}

\item Compute expansion of one-sided and centered differences in
characteristic variables.
\begin{align}
\alpha^k_L &= l^k \cdot \delta W_L \\
\alpha^k_R &= l^k \cdot \delta W_R \\
\alpha^k_C   &= l^k \cdot \delta W
\end{align}

\item Apply van Leer limiter
\begin{align}
\alpha^k =
  \begin{cases}
  \text{min}(2 \, | \, \alpha^k_L \, |, \,
      2 \, | \, \alpha^k_R \, |, \,
           | \, \alpha^k_C \, |) &
    \text{if $\alpha^k_L \cdot \alpha^k_R > 0$}
  \\
  0 &
    \text{otherwise}
  \end{cases}
\end{align}

\item $\Delta^{vL} = \sum_k \alpha^k r^k$
\end{enumerate}

\noindent
Here, $l^k= l^k(W^n_\ibold)$ and $r^k = r^k(W^n_\ibold)$.

For a variety of problems, it suffices to apply the van Leer limiter 
componentwise to the differences. Formally,
this can be obtain from the more general
case above by taking the matrices of left and right eigenvectors to be
the identity.

\subsection{Flattening}

Finally, we give the algorithm for computing the flattening coefficient 
$\zeta_\ibold$. We assume that there is a quantity corresponding to the
pressure in gas dynamics (denoted here as $p$) 
which can act as a steepness indicator, and a
quantity corresponding to the bulk modulus (denoted here as $K$, given
as $\gamma p$ in a gas), that can be used to non-dimensionalize
differences in $p$.
\begin{alignat}{1}
\label{eqn::flattening}
\zeta_\vbold = & \;
  \begin{cases}
  ~ \underset{0 \le d < \Dim}{min} \, \zeta^d_\vbold ~ &
    \text{if $\sum^{\Dim - 1}_{d=0} \Delta^d_1 u^d_\vbold < 0$}
  \\
  ~ 1 &
    \text{otherwise}
\end{cases} \\
\zeta^d_\vbold = & \; min_3(\widetilde{\zeta}^d,d)_\vbold \notag \\
\widetilde{\zeta}^d_\vbold = & \; \eta(\Delta^d_1 p_\vbold, \,
                                   \Delta^d_2 p_\vbold, \,
                              min_3(K,d)_\vbold ) \notag \\
\Delta^d_1 p_\vbold = & \; \Delta^C p_\vbold ~ | ~ \Delta^L p_\vbold ~
| ~ \Delta^R p_\vbold \notag \\ 
\Delta^d_2 p_\vbold = & \; 
(\ebshift{\Delta^d_1 p}{ \ebold^d})_\vbold
 +     (\ebshift{\Delta^d_1 p}{-\ebold^d})_\vbold
                  ~ | ~ 2 \Delta^d_1 p_\vbold
                  ~ | ~ 2 \Delta^d_1 p_\vbold \notag
\end{alignat}
The functions $min_3$ and $\zeta$ are given below.
\begin{gather}
\begin{split}
min_3(q,d)_\vbold = & \; \text{min}((\ebshift{q}{ \ebold^d})_\vbold,
                                       q             _\vbold,
                             (\ebshift{q}{-\ebold^d})_\vbold)
~|~ \text{min}q_\vbold,(\ebshift{q}{-\ebold^d})_\vbold)
~|~ \text{min}(\ebshift{q}{\ebold^d})_\vbold,q_\vbold) \\
\zeta(\delta p_1, \delta p_2, p_0) = & \;
  \begin{cases}
  0 &
    \text{if $\frac{| \delta p_1 |}{p_0} > d$
         and $\frac{| \delta p_1 |}{| \delta p_2 |} > r_1$} \\
  1 - \frac{\frac{| \delta p_1 |}{| \delta p_2 |} - r_0}{r_1 - r_0} &
    \text{if $\frac{| \delta p_1 |}{p_0} > d$
         and $r_1 \ge \frac{| \delta p_1 |}{| \delta p_2 |} > r_0$} \\
  1 &
    \text{otherwise}
  \end{cases}
\\
& r_0 = 0.75, ~ r_1 = 0.85, ~ d = 0.33
\end{split}
\end{gather}
Note that $min_3$ is not the minimum over all available VoFs but
involves the minimum of shifted VoFs which includes an averaging
operation.


\section{Computing fluxes at the irregular boundary}

The flux at the embedded boundary is centered at the centroid of 
the boundary ${\bf \bar x}$.    We extrapolate the primitive solution
in space from the cell center.  We then transform to the conservative
solution and extrapolate in time using the stable, non-conservative
estimate of the flux divergence described in equation \ref{eqn::divnc}.
\begin{gather}
W_{\vbold, B} = W^n_\vbold + \sum^{D-1}_{d=0}(\bar x_d \Delta^d
W^n_\vbold   - \frac{\dt}{2 \Delta x} A^d \Delta^d W^n_\vbold )\\
F^\nph_{\vbold, B} = R_B(U^\nph_{\vbold, B}, \nbold^B_\vbold)
\label{eqn::ebirregflux}
\end{gather}
For polytropic gas dynamics, this becomes
\begin{align}
\begin{split}
\rho_{\vbold, B} =& \rho^n_\vbold + \sum^{D-1}_{d=0}(\bar x_d \Delta^d \rho^n_\vbold
   - \frac{\dt}{2 \Delta x}(u^d \Delta^d \rho  + \rho \Delta^d
   u^d)^n_\vbold) \\
p_{\vbold, B} =& p^n_\vbold + \sum^{D-1}_{d=0}(\bar x_d \Delta^d p^n_\vbold
  - \frac{\dt}{2 \Delta x}(u^d \Delta^d p  + \rho c^2 \Delta^d
  u^d)^n_\vbold)\\
u^{d1}_{\vbold, B} =& (u^{d1})^n_\vbold + \sum^{D-1}_{d=0}(\bar x_d
\Delta^d (u^{d1})^n_\vbold) \\
&- \frac{\dt}{2 \Delta x}(u^{d1} \Delta^{d1} u^{d1}  +\frac{1}{\rho} \Delta^{d1} p)^n_\vbold  \\
&- \frac{\dt}{2 \Delta x}(\sum_{d2 \ne d1} (u^{d2} \Delta^{d2} u^{d1}))
\label{eqn::ebirregflux}
\end{split}
\end{align}
If we are using solid-wall boundary conditions at the irregular boundary,
we calculate an approximation of the divergence of the velocity at the
irregular cell $D(\vec u)_\vbold$ and use it to modify the flux to 
be consistent with artificial viscosity.  The $d$-direction
momentum flux at the irregular boundary is given by $-p^r n^d$ where
$p^r$ is the pressure to emerge from the Riemann solution in equation 
\ref{eqn::ebirregflux}.  For artificial viscosity, we modify this flux
as follows.
\begin{align*}
(D \vec u)_\vbold &= 
\sum\limits^{D-1}_{{d^{'}}=0} \Delta^{d^{'}} u^{d^{'}}_{\vbold} \\
p^r & = p^r - 2K_0~\text{max}(-(D\vec u)_\vbold, 0) \vec u \cdot \hat n 
\end{align*}

