\section{Multilevel Operators.\label{sec:amrOps}}

In this section, we describe algorithmic and library support suitable
for implementing extensions of second-order accurate discretizations
of quasi-linear elliptic, parabolic, and hyperbolic PDE's in
conservation form to AMR.  Our approach will be to express the AMR
discretizations in terms of the corresponding uniform grid
discretizations at each level, using appropriate interpolation
operators to provide ghost cell values for points in the stencil
extending outside of the grids at that level. We will also define a
conservative discretization of the divergence operator on multilevel
data.

From a formal numerical analysis standpoint, a solution on an adaptive
mesh hierarchy $\{\Omega^l \}^{l_{max}}_{l=0}$ 
approximates the exact solution to the PDE only on those cells
that are not covered by a grid at a finer level.  We define the valid
region of $\Omega^l$
\beqa
\Omega^l_{valid} = \Omega^l - {\mathcal{C}}_{n^l_{ref}}(\Omega^{l+1})
\eeqa
A composite array $\varphi^{comp}$ is a collection of discrete values
defined on the valid regions at each of the levels of refinement.
\beqa
\varphi^{comp} = \{\varphi^{l, valid} \}^{l_{max}}_{l=0}
\mbox{  ,   }
\varphi^{l, valid} : \Omega^l_{valid} \rightarrow {\mathbb{R}}^m
\eeqa

We can also define valid regions and composite arrays for other
centerings.  $\Omega^{l, \ebold^d}_{valid} = \Omega^{l, \ebold^d} -
{\mathcal{C}}_{n^l_{ref}}(\Omega^{l+1, \ebold^d})$.  Thus $\Omega^{l,
\ebold^d}_{valid}$ consists of $d$-faces that are not covered by the
$d$-faces at the next finer level.  A composite vector field
$\vec{F}^{comp} = \{\vec{F}^{l, valid} \}^{l_{max}}_{l=0}$ is defined
as follows.
\beqa
\vec{F}^{l, valid} = (F^{l, valid}_0 \dots F^{l, valid}_{\Dim-1})
\\
F^{l, valid}_d : \Omega^{l, \ebold^d}_{valid} \rightarrow
{\mathbb{R}}^m
\eeqa
Thus a composite vector field has values at level $l$ on all of the
faces that are not covered by faces at the next finer level.

We want to compute finite difference approximations to differential
operators.  For example, let $L$ be a finite difference approximation to
a linear differential operator ${\mathcal{L}}$.
On a uniform grid, $L$ typically takes the form 
\begin{eqnarray}
(L \varphi)_\ibold = \sum_{|\sbold |\leq S} c_{\ibold,\sbold}
\varphi_{\ibold +\sbold}
\end{eqnarray}
Starting from this operator, we can extend $L$ to be defined on an AMR
grid hierarchy in the following fashion.
For each $\Omega^l_k \in \mathcal{R}(\Omega^l)$
\beqa
\varphi^{l,ext}_\ibold &=& \varphi^l_\ibold \mbox{ on } \Omega^l_{valid}
\\
&=& I(\varphi^{comp}, \xbold^l_0 + \ibold h^l) \mbox{ on }\mathcal{G}(\Omega^l_k,
S) \cap \Gamma^l - \Omega^l_{valid}
\eeqa
\beqa
(L\varphi)_\ibold = \sum_{|\sbold|\leq S} c_{\ibold,\sbold}
\varphi^{l,ext}_{\ibold+\sbold} \mbox{ on } \Omega^l_k
\eeqa

Here $I = I(\varphi^{comp}, \xbold)$ is an interpolation operator that
takes some combination of the valid composite data and constructs an
interpolant at the point $\xbold \in {\mathbb{R}}^\Dim$.

Let $\psi$ be a smooth function on 
${\mathbb{R}}^\Dim$, and define the level array
\beqa
\psi^l = \psi(\xbold^l_0 + \ibold h^l) \mbox{ on } \Gamma^l
\eeqa
and composite array
\beqa
\psi^{comp} = \{\psi^{l,valid}\}^{l_{max}}_{l=0} \\
\psi^{l,valid} = \psi^l \mbox{ on } \Omega^l_{valid}
\eeqa
Then the truncation error of the operator L can be computed as
follows.  For $\ibold \in \Omega^l_{valid}$
\beqa
\tau_{\ibold} &\equiv&
L^{comp}(\psi^{comp})_{\ibold} - {\mathcal{L}}(\psi)(\xbold^l_0 + \ibold
h^l)
\\ 
&=& \sum_{|\sbold|\leq{\mathcal{S}}} c_{\ibold,\sbold} 
\psi_{\ibold + \sbold}
 - {\mathcal{L}}(\psi)(\xbold^l_0 + \ibold
h^l) 
\\ 
&-& \sum_{\substack{|\sbold|\leq{\mathcal{S}} \\ \ibold + \sbold \not\in
\Omega^l_{valid}}} c_{\ibold,\sbold}(\psi_{\ibold + \sbold} -
I(\psi^{comp},\xbold^l_0 + (\ibold + \sbold) h^l))
\eeqa
The first sum is the truncation error on a uniform grid, while the
second sum gives the effect of replacing the uniform grid values of
the smooth function $\psi$ by those obtained by interpolation.

Unfortunately, this process, when used by itself, becomes unwieldy
for any but the simplest finite difference approximations.  Typically,
in order to obtain $\tau = O(h^q)$ it is necessary to compute
$I(\varphi^{comp}, \xbold^l_0 + \ibold h)$ to an accuracy of
$O(h^{p+q})$, where $p$ is order of the highest derivative of the
operator, due to the contributions of the second summand
($ \underset{| \sbold| \leq{\mathcal{S}}}{max}
(|c_{\ibold,\sbold}|) = O(h^{-p})$).
To obtain interpolants of such accuracy,
we must either use general polynomial interpolants using data located
on multiple levels of refinement, or impose minimum distance
requirements between grids at different levels of refinement.
The alternative is to accept a larger truncation error
near the boundary between levels of refinement.
In the AMR algorithms that motivate the design of Chombo, we use a
combination of all three techniques.  This approach is motivated
by the following mathematical and algorithmic considerations.

\begin{trivlist}

\item $\bullet$
Our target applications involve solving first - and second - order
quasi-linear systems of PDE's of classical type, i.e., elliptic,
parabolic, and hyperbolic.

\item $\bullet$
Our underlying uniform-grid discretizations are based on second-order
accurate methods, mainly in discrete conservation form.
\end{trivlist}

The latter 
property is one that we would like to preserve in the AMR versions of
these algorithms.  However, the requirement for discrete conservation form
leads to a loss of accuracy at coarse-fine boundaries.  Finite
difference methods rely on the cancellation of truncation error terms
in the differenced quantities in order to obtain a given accuracy on a
uniform grid.  This
is the mechanism, for example, by which the second divided difference
approximates the second derivative to $O(h^2)$, even though it is a
divided difference of quantities that are themselves accurate only to
$O(h^2)$.  This mechanism fails at the interface between different
levels of refinement.  If one is to approximate the divergence
operator with a divided difference of single-valued fluxes, it is not
possible to compute the flux so that the truncation error cancels that
of the fluxes on both the adjacent coarse and fine faces.

Fortunately, our choice of target applications makes this local loss
of accuracy acceptable.  For elliptic and parabolic problems, a
truncation error of $O(h^{p-1})$ on a set of codimension one induces a
solution error of $O(h^p)$, due to a discrete form of elliptic
regularity.  In hyperbolic problems, a truncation error
of $O(h^{p-1})$ on a set of codimension one induces a total error of
$O(h^p)$ in $L^1$ (and in $L^\infty$ as well if the boundary is
non-characteristic).

In the following, we give the details of the algorithms for 
interpolating between levels that arise in this approach. They include
averaging and interpolation methods for transferring information between
levels; specialized operators for interpolating boundary conditions at
boundaries between levels; and a conservative multilevel discretization 
of the divergence operator. For all of these cases, we will describe the
algorithms for the case of data defined on two successive levels 
$\Omega^f,\Omega^c_{valid}$. The resulting operators can all be extended
to the full AMR hierarchy by applying them to a pair of levels at a
time, provided that appropriate nesting conditions are met. For the
most part, only proper nesting is required. When that is not the case,
we will explicitly state the nesting conditions required on grids at
successive levels.

\subsection{Interlevel Transfer Operators \label{sec:ilt}}

\subsubsection{Conservative Averaging.} 

This operator is used to average
from finer levels on to coarser levels, or for constructing averaged
residuals in multigrid iterations.
\beqa
Average(\varphi,n_{ref})_{\ibold^c} = \frac{1}{(n_{ref})^d} \sum_{\ibold \in
{\mathcal{C}}^{-1}_{n_{ref}}(\ibold^c)} \varphi_\ibold
\eeqa 
This process produces values on the coarse grid that are an $O(h^2)$
estimate of the solution on the fine grid.


\subsubsection{Piecewise Constant Interpolation.} 

This operator is primarily
used in multigrid iteration to interpolate the correction from the
coarser level to the next finer level.
\beqa
I_{pwc}(\varphi)_{\ibold^f} = \varphi_{\ibold}
\eeqa 
where ${\ibold} = {\mathcal{C}}_{n_{ref}}(\ibold^f)$. This method has an interpolation error of $O(h)$.

\subsubsection{Piecewise Linear Interpolation.} \label{sec:pwl}
This method is primarily used to initialize fine grid data after
regridding. Given a level array $\varphi^c$ on $\Omega^c$, we want to 
compute  $I_{pwl}({\varphi})$ defined on an $\Omega^f$ properly nested
in $\Omega^c$.
\beqa
I_{pwl}(\varphi)_{\ibold^f} = \varphi_\ibold +
\sum^{\Dim-1}_{d=0}(\frac{(i^f_d + \half)}{n_{ref}} - (i_d + \half))
(\delta^d \varphi)_\ibold
\eeqa 
\beqa
(\delta^d \varphi)_\ibold =
  \begin{cases}
    \eta_\ibold(\delta^d_c\varphi)_\ibold &
      \text{if both $\ibold \pm \ebold^d \in \Gamma^c$}
    \\
    \varphi_{\ibold + \ebold^d} - \varphi_\ibold &
      \text{if $\ibold - \ebold^d \not\in \Gamma^c$}
    \\
    \varphi_\ibold - \varphi_{\ibold - \ebold^d} &
      \text{if $\ibold + \ebold^d \not\in \Gamma^c$}
  \end{cases}
\eeqa 
\beqa
\eta_\ibold = \chi(min(\varphi^{max}_\ibold - \varphi_\ibold,
\varphi_\ibold - \varphi^{min}_\ibold), \sum_{d = 0}^{\Dim - 1}
|\delta^d_c \varphi|_\ibold)
\eeqa 
\beqa
(\delta^d_c\varphi)_\ibold =
  \begin{cases}
    \half(\varphi_{\ibold+\ebold^d} - \varphi_{\ibold-\ebold^d}) &
      \text{if both $\ibold \pm \ebold^d \in \Gamma^c$}
    \\
    \varphi_{\ibold + \ebold^d} - \varphi_\ibold &
      \text{if $\ibold - \ebold^d \not\in \Gamma^c$}
    \\
    \varphi_\ibold - \varphi_{\ibold - \ebold^d} &
      \text{if $\ibold + \ebold^d \not\in \Gamma^c$}
  \end{cases}
\eeqa 
\beqa
\chi(a,b) = 
  \begin{cases}
    \frac{a}{b} & \text{if $a < b$} \\
    1           & \text{otherwise}
  \end{cases}
\eeqa 
\beqa
\varphi^{max}_\ibold = \underset{|\sbold | \leq 1}{max}
(\varphi_{\ibold +\sbold})~,~
\varphi^{min}_\ibold = \underset{|\sbold | \leq 1}{min}
(\varphi_{\ibold +\sbold})
\eeqa
At cells adjacent to the boundary of the computational domain
$\Gamma^c$,
the maximum and minimum are taken over the points $\ibold + \sbold$
that are contained in the computational domain. 
Also note, the arguments to $\chi$ are always non-negative.

We use the limiter $\eta$ to keep the
interpolated values from exceeding local estimates of the maximum and
minimum values of the solution on the coarser grid. As long as $\eta =
1$, i.e., the limiter does not reduce the values of the slopes, the
error in the interpolated values is $O(h^2)$.


\subsection{Coarse-Fine Boundary Interpolation}

\subsubsection{Piecewise Linear Interpolation \label{sec:pwlB}}

Assume there are two levels of grid $\Omega^c, \Omega^f$, and data
defined on the fine grid and on the valid region of the coarse grid:
\begin{gather*}
\varphi^f: \Omega^f \rightarrow {\mathbb{R}}
\\
\varphi^{c, valid}: \Omega^c_{valid} \rightarrow {\mathbb{R}}
\end{gather*}
We want to compute an extension $\tilde{\varphi}^f$ of $\varphi^f$ on
$\tilde{\Omega}^f = {\mathcal{G}}(\Omega^f, p) \cap \Gamma^f, p > 0$,
which is accurate to
$O(h^2)$, assuming only that
${\mathcal{C}}_r(\tilde{\Omega}^f) \cap \Gamma^c \subset \Omega^c$.
There must be enough points on the
coarse level to interpolate out to a distance of $p$ fine cells from 
$\Omega^f$. One way to achieve this goal is by choosing an appropriate
blocking factor, i.e., we assume that $\Omega^f$ is $n_{ref}(\lfloor
\frac{p}{n_{ref}}\rfloor + min(1,p \mbox{ } mod \mbox{ } n_{ref}))$ - blocked. Combined with proper
nesting, this ensures that there are sufficient cells in $\Omega^c$ to
perform the interpolation.

\noindent
We perform this calculation in these steps
\begin{trivlist}
\item
(i)  Extend $\varphi^{c, valid}$ to $\varphi^c$, defined on all of
$\Omega^c:  \varphi^c = Average(\varphi^f, n_{ref}) \mbox{ on }
{\mathcal{C}}_{n_{ref}}(\Omega^f)$
\item
(ii)  For each $\ibold^f \in \tilde{\Omega}^f - \Omega^f$, compute a
piecewise linear interpolant. For $\ibold = {\mathcal{C}}_{n_{ref}}(\ibold^f)$,
\beqa
\tilde{\varphi}^f_{\ibold^f} = \varphi_\ibold +
\sum^{\Dim-1}_{d=0}(\frac{(i^f_d + \half)}{n_{ref}} - (i_d + \half))
(\delta^d \varphi)_\ibold \equiv I^B_{pwl}(\varphi^f, \varphi^{c,
valid})_{\ibold^f}
\eeqa
Unlike in the interlevel transfer operator $I_{pwl}$, we use a minimal
stencil for $(\delta^d \varphi)_\ibold$
(Figure \ref{fig:coarse-interp}).
\beqa
(\delta^d \varphi)_\ibold =
  \begin{cases}
    \delta_{vL}(\varphi_{\ibold - \ebold^d}, \varphi_\ibold, \varphi_{\ibold + \ebold^d}) &
      \text{if both $\ibold \pm \ebold^d \in \Omega^c$}
    \\
    \varphi_{\ibold + \ebold^d} - \varphi_\ibold &
      \text{if $\ibold - \ebold^d \not\in \Omega^c$}
    \\
    \varphi_\ibold - \varphi_{\ibold - \ebold^d} &
      \text{if $\ibold + \ebold^d \not\in \Omega^c$}
  \end{cases}
\eeqa
\beqa
\delta_{vL}(a_l, a_c, a_r) =
  \begin{cases}
    min(\half|a_l - a_r|, |a_l - a_c|, |a_r - a_c|)sign(a_r - a_l) &
      \text{if $(a_l - a_c)(a_c - a_r) > 0$}
    \\
    0 &
      \text{otherwise}
  \end{cases}
\eeqa
\begin{figure}
\vspace{3in}
\special{psfile=figs/coarse-interp.eps hscale=75 vscale=75
hoffset=100 voffset=0}
\caption{Interpolation on the coarse grid.  One-sided differences are
used at cells marked with circles.}
\label{fig:coarse-interp}
\end{figure}
\item (iii) $\tilde{\varphi}^f = \varphi^f \mbox{ on } \Omega^f$
\end{trivlist}
The truncation error of this interpolation operator is $O(h^2)$, i.e.,
if $\psi = \psi(\xbold)$ is a smooth function, and
\begin{gather*}
\psi^f_\ibold = \psi(\xbold^f_0 + \ibold h^f) \mbox{ on } \Omega^f
\\
\psi^{c, valid}_\ibold = \psi(\xbold^c_0 + \ibold h^c) \mbox{ on }
\Omega^{c, valid} 
\end{gather*}
then
\beqa
\tilde{\psi}^f_{\ibold^f} = \psi(\xbold^f_0 + \ibold h^f)
+ O(h^2) \mbox{ for } \ibold \in \tilde{\Omega}^f - \Omega^f
\eeqa
where $\tilde{\psi}^f$ is the extension of $(\psi^f, \psi^c)$ computed
using the process outlined above.  The key point is that, as long as
the extension of $\psi^c$ to ${\mathcal{C}}_{n_{ref}}(\Omega^f)$ is accurate
to $O(h^2)$, the undivided difference formula approximates $h^c
\frac{\partial \psi}{\partial x^d}$ to $O(h^2)$, and differs from the
Taylor expansion of $\psi$ around $(\xbold^c + \ibold h^c)$ by
$O(h^2)$.

%\subsection*{Quadratic Interpolation}
%For each $\Omega^l_k \in \Omega^l$, we want to compute $O(h^3)$
%interpolated values on $\tilde{\Omega}^l_k - \Omega^l$, where
%$\tilde{\Omega}^l_k = \{ \ibold : \ibold + \ebold^d \in \Omega^l_k, 0
%\leq d < \Dim \}$
%Assumptions:
%\beqa
%{\mathcal{G}}({\mathcal{C}}_{n^l_{ref}}(\Omega^{l+1}), 2) \subset \Omega^l
%{\mathcal{G}}({\mathcal{C}}_{n^{l-1}_{ref}}(\Omega^l), 1) \subset
%\Omega^{l-1}
%\eeqa
%For each $\ibold \in \tilde{\Omega}^l_k - \Omega^l, \exists !\pm e^d$
%such that $\ibold \pm \ebold^d_d \in \Omega^l_k$

%\subsection*{Tangential interpolation}

%for $d=$

\subsubsection{Quadratic Coarse-Fine Boundary Interpolation
\label{sec:quadB}}

This interpolation scheme is motivated by the requirements of
constructing consistent discretizations of second-order operators.
Given $\varphi^f$, $\varphi^{c,valid}$, we want to compute a level
vector field $\vec{G}^f = (G^f_0, \dots G^f_{\Dim - 1})$ that
approximates the gradient to sufficient accuracy so that, when we take
its divergence, we obtain at least an $O(h)$ approximation to the
Laplacian. For each $\Omega^{f}_k \in \mathcal{R}(\Omega^f)$, we
construct an extension $\tilde{\varphi}$ of $\varphi^f$.
\begin{align*}
\tilde{\varphi} : &\tilde{\Omega}^f_k \rightarrow \mathbb{R}^m
\\
\tilde{\Omega}^f_k = (\underset{{\pm = +,-}}{\cup} 
& \overset{\Dim-1}{\underset{d=0}{\cup}} \Omega^f_k
\pm \ebold^d) 
\cap \Gamma^f
\end{align*}
Then, for each $\ibold + \half \ebold^d$ such that both
$\ibold, \ibold + \ebold^d \in \tilde{\Omega}^f_k$ 
we can compute
a centered difference approximation to the gradient on a staggered grid
\beqa
G^f_{d,\ibold + \half \ebold^d} =
\frac{1}{h^f}(\tilde{\varphi}_{\ibold + \ebold^d} -
\tilde{\varphi}_\ibold)
\eeqa
For this estimate of the gradient to be accurate to $O(h^2)$, it is
necessary to compute an $O(h^3)$ extension of $\varphi^f$.
On $\tilde{\Omega}^f_k \cap \Omega^f$, the values for
$\tilde{\varphi}$ will be given by $\tilde{\varphi}_\ibold =
\varphi^f_\ibold$.  The values for the remaining points in
$\tilde{\Omega}^f_k - \Omega^f$ will be obtained by interpolating data
from $\varphi^f$ and $\varphi^c$.

To perform this interpolation, we first observe that, given $\ibold
\in \tilde{\Omega}^f_k - \Omega^f$, there is a unique choice of $\pm$
and $d$, such that $\ibold \mp \ebold^d \in \Omega^f_k$. Having
specified that choice, the interpolant
is constructed in two steps (figure \ref{fig:CFInterp-corner}).
\begin{trivlist}
\item 
(i) Interpolation in the direction orthogonal to $\ebold^d$. We
compute
\beqa
\xbold = \frac{\ibold + \half \ubold}{n_{ref}} - (\ibold^c +
\half \ubold)
\eeqa
where $\ibold^c = {\mathcal{C}}_{n_{ref}}(\ibold)$. 
The real-valued vector $\xbold$ is the displacement of the cell center
$\ibold$ on the fine grid from the cell center at $\ibold^c$ on the
coarse
grid, scaled by $h^c$.
\beqa
\hat{\varphi}_\ibold = \varphi^c_{\ibold^c} +
\sum_{d' \neq d}
\Bigl[
  \bigl(
    x_{d'}(D^{1, d'}\varphi^c)_{\ibold^c} +
    \half (x_{d'})^2(D^{2, d'}\varphi^c)_{\ibold^c}
  \bigr) + 
  \sum_{d'' \neq d, d'' \neq d'} x_{d'}x_{d''}(D^{d'd''}\varphi^c)_{\ibold^c}
\Bigr]
\eeqa

The second sum has only one term if $\Dim=3$, and no terms if $\Dim=2$.
\item
(ii) Interpolation in the normal direction.  
\begin{equation*}
\tilde{\varphi}_\ibold = I_q^B(\varphi^f,\varphi^{c,valid}) \equiv
4a + 2b + c ~ , ~ 
\tilde{x_d} = x_d - \half (n_{ref} + 3)
\end{equation*}
where $a,b,c$ are computed
to interpolate between the collinear data
\begin{align*}
&((\ibold \pm \half(n^l_{ref} -1)\ebold^d)h,
\hat{\varphi}_\ibold), \\ &((\ibold \mp \ebold^d)h, \varphi^l_{\ibold \mp
\ebold_d}), \\ &((\ibold \mp 2\ebold^d)h, \varphi^l_{\ibold \mp 2\ebold_d)}
\end{align*}
\end{trivlist}
In (i), the quantities $D^{1, d'}\varphi^c, D^{2, d'}\varphi^c$
and $D^{d'd''}\varphi^c$ are difference approximations to $
\frac{\partial}{\partial x_{d'}}, \frac{\partial^2}{\partial
x^2_{d'}}$, and $\frac{\partial^2}{\partial x_{d'}\partial x_{d''}}$,
respectively.  $D^{1, d}\varphi$ must be accurate to $O(h^2)$, while
the other two quantities need only be $O(h)$.  The
strategy for computing these quantities is to use only values in
$\Omega^c_{valid}$ to compute these difference approximations.  For
the case of $D^{1, d'}\varphi, D^{2, d'}\varphi$, we use 3-point
stencils, centered if possible, or shifted as required to consist of
points on $\Omega^c_{valid}$.

\begin{figure}
\epsfxsize=2.0in
\hspace{2.0in} \epsffile{figs/CFInterp-corner.eps}
\caption{Interpolation at a coarse-fine interface.  Left stencil is
the usual stencil. Right stencil is the modified interpolation
stencil; since the upper coarse cell is covered by a fine grid, use
shifted coarse grid stencil (open circles) to get intermediate values
(solid circles), then perform final interpolation as before to get
``ghost cell'' values (circled X's). Note that to perform
interpolation for the horizontal coarse-fine interface, we need to
shift the coarse stencil left.}
\label{fig:CFInterp-corner}
\end{figure}

\beqa
(D^{1, d'}\varphi)_\ibold =
  \begin{cases}
    \frac{1}{2}(\varphi^c_{\ibold + \ebold^{d'}} -
                 \varphi^c_{\ibold - \ebold^{d'}}) &
      \text{if both $\ibold \pm \ebold^{d'} \in \Omega^c_{valid}$}
    \\
    \pm\frac{3}{2}(\varphi^c_{\ibold \pm  \ebold^{d'}} -
                    \varphi^c_\ibold                      )
    \mp\frac{1}{2}(\varphi^c_{\ibold \pm 2\ebold^{d'}} -
                    \varphi^c_{\ibold \pm  \ebold^{d'}}) &
      \text{if $\ibold \pm \ebold^{d'}     \in \Omega^c_{valid}$,
               $\ibold \mp \ebold^{d'} \not\in \Omega^c_{valid}$}
    \\
    0 &
      \text{otherwise}
  \end{cases}
\eeqa
\beqa
(D^{2, d'}\varphi)_\ibold =
  \begin{cases}
    \varphi^c_{\ibold+\ebold^{d'}} -
                 2\varphi^c_\ibold               +
                  \varphi^c_{\ibold-\ebold^{d'}} &
      \text{if both $\ibold \pm \ebold^{d'} \in \Omega^c_{valid}$}
    \\
    \varphi^c_\ibold                     -
                 2\varphi^c_{\ibold \pm   \ebold^{d'}} +
                  \varphi^c_{\ibold \pm 2 \ebold^{d'}} &
      \text{if $\ibold \pm \ebold^{d'}     \in \Omega^c_{valid}$,
               $\ibold \mp \ebold^{d'} \not\in \Omega^c_{valid}$}
    \\
    0 &
      \text{otherwise}
  \end{cases}
\eeqa

\begin{figure}[htp]
\epsfxsize=2.0in
\hspace{1.5in} \epsffile{figs/mixed.eps}
\caption{Mixed-derivative approximation illustration.
        The upper-left corner is covered by a finer level
        so the mixed derivative in the upper left (the uncircled
        x) has a stencil which extends into the finer level.
        We therefore average the mixed derivatives centered
        on the other corners (the filled circles) to approximate
        the mixed derivatives for coarse-fine interpolation in three 
        dimensions.}
\label{MIXEDFIG}
\end{figure}

In the case of $D^{d'd''}\varphi^c$, we use an average of all of
the four-point difference approximations $\frac{\partial^2}{\partial
x_{d'}\partial x_{d''}}$ centered at $d', d''$ corners adjacent to
$\ibold$ such that all four points in the stencil are in $\Omega^c_{valid}$
(Figure \ref{MIXEDFIG})
\beqa
(D^{d'd''}_{corner}\varphi^c)_{\ibold + \half \ebold^{d'} + \half \ebold^{d''}} = 
  \begin{cases}
    \frac{1}{h^2}(\varphi_{\ibold+\ebold^{d'} + \ebold^{d''}} +
                  \varphi_\ibold -
                  \varphi_{\ibold + \ebold^{d'}} -
                  \varphi_{\ibold + \ebold^{d''}}) &
      \text{if $[\ibold, \ibold + \ebold^{d'} + \ebold^{d''}] \subset \Omega^c_{valid}$}
    \\
    0 &
      \text{otherwise}
  \end{cases}
\eeqa
\beqa
(D^{2, d'd''}\varphi^c)_\ibold = 
  \begin{cases}
    \frac{1}{N_{valid}} \sum_{s' = \pm 1} \sum_{ s'' = \pm 1}
      (D^{d'd''}\varphi^c)_{\ibold + \half s'\ebold^{d'} + \half s'' \ebold^{d''}} &
      \text{if $N_{valid} > 0$}
    \\
    0 &
      \text{otherwise}
  \end{cases}
\eeqa
where $N_{valid}$ is the number of nonzero summands.
To compute (ii), we need to compute the interpolation coefficients $a$
$b$, and $c$.

\begin{gather*}
a=\frac{\hat{\varphi} - 
		   (n_{ref}\cdot |x_d| + 2) \varphi_{\ibold \mp \ebold^d} + 
		   (n_{ref}\cdot |x_d| + 1) \varphi_{\ibold \mp 2 \ebold^d}}
		   {(n_{ref}\cdot |x_d| + 2)(n_{ref}\cdot |x_d| + 1)}
% a = \frac{2(\hat{\varphi} - (n_{ref} + 3) \varphi_{\ibold \mp 
% \ebold^d} + (n_{ref} + 1) \varphi_{\ibold \mp 2 \ebold^d})}
% {(n_{ref} + 1)(n_{ref} + 3)}
\\
b = \varphi_{\ibold \mp \ebold^d} - \varphi_{\ibold \mp 2 \ebold^d} - a
\\
c = \varphi_{\ibold \mp 2 \ebold^d}
\end{gather*}

\subsubsection{Level Divergence, Composite Divergence, and Refluxing
\label{sec:fluxreg}}

Let $\vec{F}$ be a level vector field on $\Omega$. We define a
discretized divergence operator as follows.
\begin{equation} \label{eqn:flux1}
(D \vec{F})_\ibold = \frac{1}{h} \sum^{\Dim-1}_{d=0} (F_{d,\ibold
+\half \ebold^d} - F_{d,\ibold - \half \ebold^d}), \ibold \in
\Omega
\end{equation}

Let $\vec{F}^{comp} = \{\vec{F}^f, \vec{F}^{c,valid}\}$
be a two-level composite 
vector field. We want to define a composite
divergence $D^{comp}(\vec{F}^f, \vec{F}^{c,valid})_\ibold$ for 
$\ibold \in \Omega^c_{valid}$. 
To do this, we construct an extension of $\vec{F}^{c, valid}$ to
the edges adjacent to $\Omega^c_{valid}$ that are covered by fine level
faces.
 On the valid coarse-level $d$-faces, $\hat{F}_{d,\ibold + \half
\ebold^d} = F^{c, valid}_{d,\ibold + \half \ebold^d}$.  On the faces
adjacent to cells in $\Omega^c_{valid}$, but not in $\Omega^{l,
\ebold^d}_{valid}$, we set $\hat{F}_d$ to be $<F^f_d>$, the average of
$F^f_d$ onto the next coarser level.
\beqa
<F^f_d>_{\ibold + \half \ebold^d} = \frac{1}{(n_{ref})^{\Dim-1}}
\sum_{\ibold^f + \half \ebold^d \in {\mathcal{F}}^d} F^f_{d,\ibold^f
+ \half \ebold^d}
\eeqa
\beqa
\ibold + \half \ebold^d \in {\zeta}^f_{d,+} \cup {\zeta}^f_{d,-}
\eeqa
Here the sum is over the set of all fine level $d$-faces that are
covered by $[\ibold + \half \ebold^d]$, which is given as a
rectangle in $\Gamma^{f,\ebold^d}$.
\beqa
{\mathcal{F}}^d = [\ibold ~                      n_{ref} + \half \ebold^d,
                  (\ibold + (\ubold - \ebold^d)) n_{ref} + \half \ebold^d]
\eeqa
${\zeta}^f_{d,\pm}$ consists of all the $d$-faces in $\Omega^c$ on
the boundary of $\Omega^{l+1}$, with valid cells on the low $(\pm = -)$
or high $(\pm = +)$ side.
\beqa
{\zeta}^f_{d,\pm} = \{\ibold \pm \half \ebold^d : \ibold \pm 
\ebold^d \in \Omega^c_{valid}, \ibold \in
{\mathcal{C}}_{n_{ref}}(\Omega^f) \}
\eeqa

For both performance reasons and algorithmic reasons, it is useful to
express $D^{comp}$ as a succession of applications of the level
divergence operator $D$ applied to
extensions of $\vec{F}^{l, valid}$ to the entire level, followed by a
step that corrects the cells in $\Omega^c_{valid}$ that are
adjacent to $\Omega^f$.
We define a {\it flux register} $\delta \vec{F}^f$ associated with the fine
level
\beqa
\delta \vec{F}^f = (\delta F^f_0, ..., \delta F^f_{\Dim-1})
\\
\delta F^f_d:{\zeta}^f_{d,+} \cup {\zeta}^f_{d,-}
\rightarrow {\mathbb{R}}^m
\eeqa
Let $\vec{F}^c$ be {\it any}
coarse level vector field that extends $\vec{F}^{c, valid}$, i.e.
\beqa
F^c_d = F^{c, valid}_d \mbox{ on } \Omega^{c, \ebold^d}_{valid}
\eeqa
Then for $\ibold \in \Omega^c_{valid}$,
\begin{equation} \label{eqn:flux2}
D^{comp}(\vec{F}^f,\vec{F}^{c,valid})_\ibold = (D \vec{F}^c)_\ibold +
D_R(\delta \vec{F}^c)_\ibold
\end{equation}
Here $\delta \vec{F}^c$ is a flux register, set to be
\beqa
\delta F^f_d = <F^f_d> - F^c_d \mbox{ on } {\zeta}^c_{d, +}
\cup {\zeta}^c_{d, -}
\eeqa
$D_R$ is the reflux divergence operator, given by the following for
valid coarse level cells adjacent to $\Omega^f$.
\beqa
D_R(\delta \vec{F}^f)_\ibold = \frac{1}{h^c} \sum^{\Dim-1}_{d=0}
\sum_{
\substack
{\pm = +,-: \\ \ibold \pm \half \ebold^d \in
{\zeta}^f_{d, \mp}
}} \pm \delta F^f_{d,\ibold \pm \half \ebold^d}
\eeqa
For the remaining cells in $\Omega^c_{valid}$, $D_R(\delta \vec{F}^f)$
is defined to be identically zero.

Let $\vec{H} = (H_0 \dots H_{\Dim - 1})$,
$H_d :\mathbb{R}^\Dim \rightarrow \mathbb{R}^m$ be a smooth vector field,
and define discrete
level and composite vector fields by evaluating $\vec{H}$ on the grid.
\beqa
\vec{H}^f = (H^f_0 \dots H^f_{\Dim - 1}) \mbox{ , } 
H^f_{d,\ibold + \half \ebold^d} = H_d(\xbold^f_0 + (\ibold + \half
\ebold^d)
h^f) \\
\vec{H}^c = (H^c_0 \dots H^c_{\Dim - 1}) \mbox{ , } 
H^c_{d,\ibold + \half \ebold^d} = H_d(\xbold^c_0 + (\ibold + \half
\ebold^d)
h^c)
\eeqa
We can then compute the truncation error of the composite divergence
using (\ref{eqn:flux2}).
\beqa
D^{comp}(\vec{H}^f,\vec{H}^{c,valid}) &=& D(\vec{H}^c) + D_R(\delta \vec{H})
\\
&=& \nabla \cdot \vec{H} + O(h^2) + D_R(\delta \vec{H})
\eeqa
Here $H^{c,valid}_d$ is given by restricting $H^c_d$ to 
${\Omega^{c,\ebold^d}_{valid}}$, and we make use of
the observation that the centered difference approximation to the
divergence given by (\ref{eqn:flux1}) is second order accurate. Away from
the cells adjacent to $\Omega^f$, the contribution from the reflux
divergence is zero, and the truncation error is $O(h^2)$. To estimate
the truncation error at cells adjacent to the coarse - fine interface,
we note that
\beqa
\delta H^f_d = <H^f_d> - H^c_d = O((h^c)^2)
\eeqa
So that $D_R(\delta \vec{H}) = O(h^c)$, i.e., we lose one order of
accuracy due to the correction to the divergence that maintains
conservation form.

\subsubsection*{Laplacian \label{subsect:compLap}}
Using the operators described above, we can now define a discretization
of the Laplacian on an adaptive mesh hierarchy. Let
$\varphi^{comp}$ a composite array defined on an AMR grid hierarchy
satisfying proper nesting.
The Laplacian is defined as the divergence of the gradient:
\begin{equation}
(L^{comp} \varphi^{comp})_\ibold \equiv
D^{comp}(\vec{G}^{l+1,valid},\vec{G}^{l,valid})_\ibold \mbox{ , } 
\ibold \in \Omega^l_{valid}
\label{eqn:compLap}  
\end{equation}
where $\vec{G}^{l,valid} = \vec{G}(\varphi^{l,valid},\varphi^{l-1,valid})$
is computed using the algorithm in section \ref{sec:quadB}.
It is assumed here that
the discrete gradients can be computed using the boundary conditions 
for the faces that lay on the boundary of the domain. It is not difficult to
check that, if the grids are properly nested, that the stencil of
$(L^{comp} \varphi^{comp})_\ibold$ is contained in the valid regions of the
meshes at levels $l$, $l \pm 1$. Away from boundaries between levels,
this discretization reduces to the standard $2\Dim + 1$ point
discretization of the Laplacian.
%\begin{equation}
%L^{\ell} \varphi^\ell  =  D^\ell G^\ell \varphi^\ell. \label{eqn:levelLap} 
%\end{equation}
%On the interiors of grids, (\ref{eqn:compLap}) and
%(\ref{eqn:levelLap}) reduce to the normal five-point second-order discrete
%Laplacian.  On the fine side of a coarse-fine interface, the
%interpolation operator $I$ fills ghost-cell values which are used in
%the five-point stencil. On the coarse side of a coarse-fine interface,
%(\ref{eqn:compLap}) becomes:
%\begin{equation}
%L^{comp,\ell}\varphi  =  L^{\ell}\varphi^{\ell} + D_R^\ell(\delta
%G\varphi^{\ell+1}) 
%\end{equation}
%\begin{equation}
%\delta G\varphi^{\ell+1}  =  \langle G^{\ell+1} \varphi \rangle -
%G^{\ell}\varphi^{\ell}. \nonumber
%\end{equation}
%
On grid interiors, $L^{comp}$ has a truncation error of $O(h^2)$ due
to cancellation of error terms in the centered-difference stencil.  At
coarse-fine interfaces, this drops to $O(h)$ due to division of the
$O(h^3)$ interpolant by $h^2$ and the loss of centered-difference
cancellations.  However, if the discrete equation $L^{comp}
\varphi = \rho$ is solved using these operators, the resulting solution $\varphi$ is
second-order accurate, because this loss of accuracy occurs on a set
of codimension one \cite{JohansenColella1998}. The dependencies of the Laplacian
operators may again be expressed explicitly: if
$L^{comp,l}(\varphi^{comp})$ is $L^{comp}(\varphi^{comp})$ restricted to 
$\Omega^l_{valid}$, then $L^{comp,l}(\varphi^{comp}) = 
L^{comp,l}(\varphi^{l,valid}, \varphi^{l+1,valid}, \varphi^{l-1,valid})$.

\section{C++ Classes for Two-Level Operators}

In this section, we document the user interfaces for a set of C++
classes that implement the operators described above. Typically, the
interface has two parts. The constructor and define function construct
the persistent data, such as interpolation coefficients and the 
{\tt IntVectSet}s defining the irregular regions where 
the operator must be applied. This can either be done by calling a
defining constructor, or by calling a member function {\tt define} with
the same arguments on an object that has been constructed with a default
constructor. Note that for classes where problem domain information is
required for construction, there are generally two sets of
constructors and define functions -- one with a {\tt Box} to represent
the domain, the second with a {\tt ProblemDomain}; if the functions
with a {\tt Box} are used, a non-periodic domain is assumed.
The second part of the interface consists of the functions
that actually apply the operator to the data. Once the operator has been
defined, the user can apply it multiple times to different data sets. 
The operator must be redefined only when the grids change. 



