\section{Introduction}

The Chombo library is built with the ability to call Fortran routines
from C++.  There are many reasons to want to do this.  For example,
one many want to use the more complex data structures that C++
supports but may not want to forfeit the superior floating-point
performance that Fortran offers.  The details of mixed language
programming, however, can be complex and both compiler and
platform-dependent.   Another complication is that C++ can be
written in a dimension-independent form but the syntax of 
Fortran is intrinsically dimension-dependent.  Array access,
declaration and looping all require knowledge of the dimensionality
of the problem.    Chombo Fortran is designed to create
abstractions which avoid these problems.
Chombo Fortran allows the C++-Fortran programmer many
advantages.
\begin{itemize}
\item The complicated data structures (classes) provided by Chombo in C++ can be 
        passed to and used in Fortran routines.
\item The name-mangling differences between Fortran and C++ are handled
        automatically and cleanly.
\item Type checking of arguments in calls to Fortran from C++ is
        handled automatically by the C++ compiler.  This makes 
       mixed language code far less error-prone.
\item Dimension-independent Fortran code is made possible.  This
       eliminates the maintenance problems associated with having
        to maintain separate Fortran kernels for simulation codes
        which differ only in the number of spatial dimensions.
\item Very long Fortran argument lists and declarations 
      (due to array specification) are greatly reduced by 
      the Chombo Fortran macros.  This makes Chombo Fortran less
      error-prone and easier to read.
\end{itemize}
The basic usage pattern is this.  One uses
Chombo Fortran to declare her subroutine argument lists and local floating
point arguments.  ChF  interprets these macros in the context of 
the input dimensionality and precision and creates a Fortran file. 
ChF also creates a prototype file to be included in the C++ calling file which 
unravels the compiler and platform-dependence of the Fortran name
mangling (so C++ will be able to find the function).  


\section{ChF\ Fortran macros}

There are three classes of Fortran macros in ChF: array declaration, array
access and dimension-handling.  The array declaration macros are used to
specify arguments to Fortran subroutines that will be called from C++.  The
array access macros are used to reference these arguments in the body of
Fortran subroutines.  The dimension-handling macros are used in the body of the 
Fortran subroutines to create dimension-independent code.

\section{dimension-handling macros}

\noindent The dimension-handling macros are:
\begin{itemize}
\item{\tt CHF\_DDECL} and {\tt CHF\_AUTODECL} for declaring variables
  and creating argument lists 
\item{\tt CHF\_DTERM} for choosing multiple expressions or statements based on dimension
\item{\tt CHF\_DINVTERM} for choosing multiple expressions or statements based on dimension and reversing their order
\item{\tt CHF\_DSELECT} for choosing one expression or statement based on dimension
\item{\tt CHF\_AUTOID} for setting multiple variables based on dimension
\item{\tt CHF\_MULTIDO} and {\tt CHF\_AUTOMULTIDO} for handling nested DO loops
\item{\tt CHF\_ENDDO} goes with {\tt CHF\_MULTIDO} and {\tt CHF\_AUTOMULTIDO}
\end{itemize}

{\tt CHF\_DDECL[arg0;arg1;arg2]} translates to {arg0, arg1, arg2} (in three
dimensions).  This is useful when one needs to declare variables that only
exist in a dimension-dependent context. Say, for example, one has SpaceDim
components of velocity called {\tt (u,v,w)} in three dimensions.  Since in
two dimensions, the third component is not used in the code, one could
declare these variables as
\begin{verbatim}
      integer CHF_DDECL[ u; v; w ]
\end{verbatim}
to avoid ``unused variable'' compiler warnings.  This macro will
respect carriage returns and other white space.

This is also used in creating argument lists for calling other routines.
Using the previous example, to call a routine named {\tt FOO} that expects SpaceDim
arguments, one would write the call as
\begin{verbatim}
      call FOO( CHF_DDECL[ u; v; w ] )
\end{verbatim}

The {\tt CHF\_AUTODECL} macro performs a similar function by expanding
a root: {\tt CHF\_AUTODECL[arg]} will expand to {\tt arg0, arg1, arg2}
(in three dimensions). This can result in more compact code,
especially for code intended to support higher dimensionality.

Similarly, {\tt CHF\_DTERM[arg0;arg1;arg2]} translates to
{\tt arg0arg1arg2} in three dimensions and {\tt arg0arg1} in
two dimensions.  This is useful if one has 
code that is dimension-dependent.
For example:
\begin{verbatim}
      integer CHF_DDECL[ii;jj;kk]
      
      CHF_DTERM[
      ii = CHF_ID(0,idir);
      jj = CHF_ID(1,idir);
      kk = CHF_ID(2,idir)]
\end{verbatim}
This macro will respect carriage returns and other white space.

The {\tt CHF\_DINVTERM[arg0;arg1;arg2]} macro is a variation on {\tt
CHF\_DTERM} which reverses the chosen arguments.  It translates to
{\tt arg2arg1arg0} in three dimensions and {\tt arg1arg0} in
two dimensions.  This is useful if one has indexing loops in
code that is dimension-dependent.
For example:
\begin{verbatim}
      integer CHF_DDECL[ii;jj;kk]
      
      CHF_DINVTERM[
      do ii = 0,10;
      do jj = 0,10;
      do kk = 0,10]
\end{verbatim}
Like {\tt CHF\_DTERM}, this macro respects carriage returns and other white space.

The {\tt CHF\_DSELECT} macro is a variation on {\tt CHF\_DTERM}.  Instead
of choosing the arguments from 1 to SpaceDim, it chooses {\em only} the 
SpaceDim'th argument.  This is useful for expressions that are different
for each dimension.  For example:
\begin{verbatim}
      rho = CHF_DSELECT[ cos(x) ; sin(x*y) ; cos(x*z)*sin(y*z) ]
\end{verbatim}
Like {\tt CHF\_DTERM}, this macro respects carriage returns and other white space.

The macro {\tt CHF\_AUTOID[ii;idir]} generates
\begin{verbatim}
      CHF_DTERM[
      ii0 = CHF_ID(0,idir);
      ii1 = CHF_ID(1,idir);
      ii2 = CHF_ID(2,idir)]
\end{verbatim}
and can also be called with an additional optional argument,
where {\tt CHF\_AUTOID[ii;idir;factor]} generates
\begin{verbatim}
      CHF_DTERM[
      ii0 = factor*CHF_ID(0,idir);
      ii1 = factor*CHF_ID(1,idir);
      ii2 = factor*CHF_ID(2,idir)]
\end{verbatim}
 
{\tt CHF\_MULTIDO} and {\tt CHF\_AUTOMULTIDO} are used to iterate over
a box in a dimension-independent  
fashion by setting up nested Fortran {\tt DO} loops. {\tt CHF\_ENDDO}
is used to 
terminate those {\tt DO} loops correctly.  Specifically,
{\tt CHF\_MULTIDO[box;i;j;k]} will generate a {\tt DO} loop for {\tt i}
nested inside a {\tt DO} loop for {\tt j} and, in 3D, this will be nested inside
a {\tt DO} loop for {\tt k}.  The {\tt i} loop will go from the first element of the
low corner of {\tt box} to the first element of the high corner of {\tt box}.
Similarly, the {\tt j} loop will use the second element and, in 3D, the
{\tt k} loop will use the third element.  {\tt CHF\_ENDDO} will end all the
{\tt DO} loops set up by {\tt CHF\_MULTIDO}.

{\tt CHF\_MULTIDO} can also be used to iterate with a stride.  The syntax
for this is {\tt CHF\_MULTIDO[box;i;j;k;2]}, where the ``2'' could be any
integer constant except 0.  A negative stride will make the loop iterate
backward in each dimension (from the high corner to the low corner).
Be warned that using a variable name instead of an integer constant will 
not produce the desired result because ChomboFortran will just think you've
coded a 4-dimensional loop so it will ignore the last variable.

Here is an example using these macros:
\begin{verbatim}
      subroutine LOOP(CHF_FRA1[array],CHF_BOX[box])

      integer CHF_DDECL[i;j;k]
      integer productsum

      productsum = 0
      CHF_MULTIDO[box;i;j;k]
        productsum = productsum + CHF_DTERM[i;*j;*k]
        array(CHF_IX[i;j;k]) = productsum
      CHF_ENDDO

      return
      end
\end{verbatim}
The other sections contain exact definitions of the other macros used in
this example.

The {\tt CHF\_AUTOMULTIDO} macro also sets up nested loops, but
constructs the indices of the loops based on a root. Specifically,
{\tt CHF\_AUTOMULTIDO[box;i]} is the same as {\tt
  CHF\_MULTIDO[box;i0;i1;i2]}. Strides are also supported, so {\tt
  CHF\_AUTOMULTIDO[box;i; 2]} is the same as {\tt
  CHF\_MULTIDO[box;i0;i1;i2;2]} (note that the indices start with 0
instead of 1, which is consistent with the conventions elsewhere in
ChomboFortran). 

Here is the previous example written using the AUTO macros:
\begin{verbatim}
      subroutine LOOP(CHF_FRA1[array],CHF_BOX[box])

      integer CHF_AUTODECL[i]
      integer productsum

      productsum = 0
      CHF_AUTOMULTIDO[box;i]
        productsum = productsum + CHF_DTERM[i0;*i1;*i2]
        array(CHF_AUTOIX[i]) = productsum
      CHF_ENDDO

      return
      end
\end{verbatim}

\section{Declaration macros}

The declaration macros are used inside Fortran {\tt
SUBROUTINE} statements (in the argument list) to specify
the types of the arguments to the subroutine. 

The ChF\ system automatically generates type declaration statements for the
variables named in ChF\ declaration macros so explicit declarations
statements for these variables are unnecessary and will likely cause
compilation errors.

The declaration macros can be used to declare variables of the basic data
types ({\tt INTEGER} and {\tt REAL\_T}) and variables corresponding to
Chombo\ C++ classes ({\tt Box, FArrayBox} and {\tt IntVect, RealVect,
Vector}).  Variables of the basic types can be scalars or 1D arrays ({\tt
CHF*1D} macros).  Variables of {\tt FArrayBox} type can have single or
multiple components ({\tt CHF*F*} macros).

The macros automatically create and declare all the extra arguments related
to array sizes that are needed.  The ChF\ access macros can be used to
access these variables.  For example, the macro {\tt CHF\_LBOUND[ A;1 ]}
would return the lowest index of the array {\tt A} in the second dimension
(dimensions are counted starting at 0).  As a special case, {\tt
CHF\_UBOUND[ V ]} is the same as {\tt CHF\_UBOUND[ V;0 ]} and is used with
{\tt Vector}s and 1D arrays of basic data types.

The ``{\tt \_CONST}'' qualifier in the macro names indicates that the
variable named in the macro is not modified in the Fortran subroutine.  This
form of the macros should be used when the C++ variable is declared 'const'.
This has no direct effect on the Fortran code or its execution, but it does
affect the C++ code that calls the Fortran subroutine and the C++ prototype
that is automatically-generated by ChF.

The following is the complete list of ChF Fortran declaration macros and
their uses.

\begin{itemize}
\item {\tt CHF\_INT[\argv]} Declare a scalar integer argument.
\item {\tt CHF\_CONST\_INT[\argv]} Declare a read-only scalar integer argument.
\item {\tt CHF\_REAL[\argv]} Declare a scalar floating point argument.
\item {\tt CHF\_CONST\_REAL[\argv]} Declare a read-only scalar floating point argument.
\item {\tt CHF\_COMPLEX[\argv]} Declare a scalar complex argument.
\item {\tt CHF\_CONST\_COMPLEX[\argv]} Declare a read-only scalar complex argument.
\item {\tt CHF\_REALVECT[\argv]} Declare a real vector of SpaceDim length argument (indices go from 0 to SpaceDim-1).
\item {\tt CHF\_CONST\_REALVECT[\argv]} Declare a constant real vector of SpaceDim length argument ("). 
\item {\tt CHF\_INTVECT[\argv]} Declare an integer vector of SpaceDim length argument (").
\item {\tt CHF\_CONST\_INTVECT[\argv]} Declare a constant integer vector of SpaceDim length argument (").
\item {\tt CHF\_I1D[\argv]} Declare a C array of integers (indices go from 0 to CHF\_UBOUND[\argv]).
\item {\tt CHF\_CONST\_I1D[\argv]} Declare a read-only C array (").
\item {\tt CHF\_R1D[\argv]} Declare a C array of reals (").
\item {\tt CHF\_CONST\_R1D[\argv]} Declare a read-only C array of reals (").
\item {\tt CHF\_VI[\argv]} Declare a Chombo {\tt Vector$<$int$>$}.
\item {\tt CHF\_CONST\_VI[\argv]} Declare a read-only Chombo {\tt Vector$<$int$>$}.
\item {\tt CHF\_VR[\argv]} Declare a Chombo {\tt Vector$<$Real$>$}.
\item {\tt CHF\_CONST\_VR[\argv]} Declare a read-only Chombo {\tt Vector$<$Real$>$}.
\item {\tt CHF\_VC[\argv]} Declare a Chombo {\tt Vector$<$Complex$>$}.
\item {\tt CHF\_CONST\_VC[\argv]} Declare a read-only Chombo {\tt Vector$<$Complex$>$}.
\item {\tt CHF\_FIA[\argv]} Declare a multi-component integer C++ BaseFab argument.
\item {\tt CHF\_CONST\_FIA[\argv]} Declare a read-only multi-component integer BaseFab argument.
\item {\tt CHF\_FRA[\argv]} Declare a multi-component floating point BaseFab argument.
\item {\tt CHF\_CONST\_FRA[\argv]} Declare a read-only multi-component floating point BaseFab argument.
\item {\tt CHF\_FIA1[\argv]} Declare a single-component integer BaseFab argument.
\item {\tt CHF\_CONST\_FIA1[\argv)} Declare a read-only single-component integer BaseFab argument.
\item {\tt CHF\_FRA1[\argv]} Declare a single-component floating point BaseFab argument.
\item {\tt CHF\_CONST\_FRA1[\argv]} Declare a read-only single-component floating point BaseFab argument.
\item {\tt CHF\_BOX[\argv]} Declare a Box argument. {\tt Box}s are always read-only.  
\end{itemize}
So a typical subroutine declaration would look like this:
\begin{verbatim}
      subroutine TYPICAL(
     &     CHF_FRA[fab],
     &     CHF_CONST_FRA[constfab],
     &     CHF_BOX[region],
     &     CHF_CONST_REAL[dx],
     &     CHF_INT[intflag])
\end{verbatim}
This routine takes two floating point BaseFabs (one constant), a box,
a constant floating point scalar and an integer.  Keep in mind that 
this is still Fortran.  All arguments are still being sent as 
pointers so they can be changed in the Fortran code.  The {\tt CONST}
modifier of the declaration just adds a {\tt const} to the C++
prototype to allow the user to send {\tt const} 
C++ variables without the C++ compiler complaining.

Chombo Fortran preprocessing of arguments can be disabled by adding the comment {\tt !~NO\_CHF} to the end of the line with the subroutine statement.  In a ``.ChF'' file, this should only be used where absolutely necessary (an example would be an internal procedure as shown in section~\ref{subsec:InternalProcExample}).  Ordinary Fortran subroutines should normally be placed in a separate ``.F'' file.

\section{Access macros}

\begin{itemize}
\item {\tt CHF\_LBOUND[\argv;\dimv]} Access the lower bound of
a BaseFab or Box \argv\ in constant
dimension \dimv.  Returns an integer variable.
\item {\tt CHF\_UBOUND[\argv;\dimv]} Access the upper bound of
a BaseFab or Box \argv\ in constant 
dimension \dimv.  Returns an integer variable.
Also used to access the upper bound of a 1D array or Chombo {\tt Vector}, 
in which case \dimv\ need not be specified.\footnote{the upper bound of
a 1D array is always one less than the dimension specified in the C++
call to {\tt CHF\_I1D} or {\tt CHF\_R1D}.}
\item {\tt CHF\_NCOMP[\argv]} Access the number of components in 
the BaseFab \argv.  Returns an integer.  Note that the components in Fortran code are
numbered from 0 to {\tt CHF\_NCOMP(\argv)-1} 
to be consistent with the requirements of C++.
\item {\tt CHF\_IX[$<${\em index0}$>$;$<${\em index1}$>$;$<${\em index2}$>$]}
Access an element of an array declared with one of the {\tt F*A*}
macros.

\item {\tt CHF\_AUTOIX[$<${\em indexRoot}$>$]}
Access an element of an array declared with one of the {\tt F*A*}
macros. Expands the {\tt indexRoot} so that {\tt CHF\_AUTOIX[i]} is
the same as {\tt CHF\_IX[i0;i1;i2]}.

\item {\tt CHF\_OFFSETIX[$<${\em indexRoot}$>$;$<${\em offsetRoot}$>$]}
Access an element of an array declared with one of the {\tt F*A*}
macros. Similar to {\tt CHF\_AUTOIX}, but with an offset.  For
example, {\tt CHF\_OFFSETIX[i;-ii]} expands to be the same as {\tt
  CHF\_IX[i0-ii0; i1-ii1; i2-ii2]}. 

\item {\tt CHF\_ID($<${\em dim1}$>$,$<${\em dim2}$>$)} Return 1 when the
arguments have the same value.  Used with {\tt CHF\_IX} for accessing
``nearby'' array elements.  
 Notice that {\tt CHF\_ID} uses
parentheses instead of square brackets and a comma instead of a
semicolon.  
Simply put, {\tt CHF\_ID}
isn't really a macro---it is a 6x6 (to support up to 6D) identity
matrix which gets declared in every subroutine.  The parentheses are
consistent with array access in Fortran, 
\end{itemize}

\noindent Notes:
\begin{itemize}
\item The \argv\ macro argument must be a variable that was declared with
one of the BaseFab, Box, 1D array or Chombo {\tt Vector} macros.
\item The \dimv\ macro argument must be an integer constant
in the
range 0\ldots{\tt CH\_SPACEDIM-1}.
\item The $<${\em dim1}$>$ and $<${\em dim2}$>$
macro arguments must be integer variables or constants in the
range 0\ldots{\tt CH\_SPACEDIM-1}.
\item Only {\tt SUBROUTINE}s 
can be called from C++.  {FUNCTION}s are not supported.
\item The dimensions values are 0-based as in C++, not 1-based 
as is the default for Fortran.
\end{itemize}

\section{C++ macros}

The ChF\ C++ macros are intended to be used in C++ code that calls Fortran
subroutines that have been declared using the ChF\ Fortran macros.  The
prototype header file that is automatically generated by the ChF\ Fortran
macros must be {\tt \#include}d in any file where the ChF\ C++ macros are 
used to call a Fortran subroutine.  The name of this header file is of the form
``$<${\em fortran\_file\_basename}$>$\_{\tt F.H}'', where $<${\em
fortran\_file\_basename}$>$ is the name of the Fortran source code file without
the extension.  Every Fortran subroutine that is called from C++ must appear in
one and only one included prototype header file.

There are two aspects to using the ChF\ macros to call Fortran subroutines:
specifying the name of the Fortran subroutine and specifying the arguments to
the Fortran subroutine.

Fortran subroutines must be called from C++ by prefixing the name of the
subroutine with {\tt FORT\_} and always using uppercase.  For example, the
Fortran subroutine named ``{\tt FOO}'' must be called from C++ using the
name ``{\tt FORT\_FOO}''.  Attempts to access the Fortran name directly will 
fail on some systems because of compiler-dependent inter-language calling
conventions.  

The C++ prototypes for Fortran subroutines with no arguments will 
be generated with the keyword ``{\tt void}'' in the argument list.

All arguments to a Fortran subroutine called from C++ must be specified in 
ChF declaration macros.  The macro names indicate the data type of the
argument and allow the ChF\ system to generate appropriate
dimension-independent code.   The macros used in C++ application code should match the
macros that appear in the prototypes provided in the {\tt *\_F.H} header
files, except that macros in application code should use the {\tt CHF\_}
prefix where the macros used in the prototypes use the {\tt CHFp\_} 
prefix.\footnote{application code should never use the {\tt CHFp\_} macros directly.}

Most of the declaration macros come in a {\tt CONST} and non-{\tt CONST}
form.  The {\tt CONST} form should be used to declare arguments that are not
modified by the Fortran subroutine.  The Box macro does not have a {\tt CONST}
form because Boxes are assumed to be constant always.

The ChF\ C++ declaration macros are almost identical in syntax and usage to
the Fortran declaration macros.  The differences are:
\begin{itemize}
\item the C++ macros are case-sensitive,
\item the single-component BaseFab macros ({\tt CHF\_*F\rm \{\tt I|R\rm \}\tt 1()})
take 2 arguments ( BaseFab, component\_number ) in C++ and 1 in Fortran,
\item the 1D array macros ({\tt CHF\_*1D}) take 2 arguments (array, length)
in C++ and 1 in Fortran,
\item for each Fortran subroutine $<${\em proc}$>$, a C++ macro {\tt
FORT\_$<${\em proc}$>$} is defined.
\end{itemize}

\section{Declaration macros}

The C++ declaration macros are those that the application programmer
uses to pass variables to Fortran routines from C++.

The following is the complete list of ChF\ C++ declaration macros and their uses.
\begin{itemize}
\item {\tt CHF\_INT(\argv)} Pass a scalar {\tt int} variable.
\item {\tt CHF\_CONST\_INT(\argv)} Pass a {\tt const} scalar {\tt int} variable.
\item {\tt CHF\_REAL(\argv)} Pass a scalar {\tt Real} variable.
\item {\tt CHF\_CONST\_REAL(\argv)} Pass a {\tt const} scalar {\tt Real} variable.
\item {\tt CHF\_COMPLEX(\argv)} Pass a scalar {\tt Complex} variable.
\item {\tt CHF\_CONST\_COMPLEX(\argv)} Pass a {\tt const} scalar {\tt Complex} variable.
\item {\tt CHF\_REALVECT(\argv)} Pass a {\tt RealVect} variable. 
\item {\tt CHF\_CONST\_REALVECT(\argv)} Pass a constant {\tt RealVect} variable. 
\item {\tt CHF\_INTVECT(\argv)} Pass a {\tt IntVect} variable.
\item {\tt CHF\_CONST\_INTVECT(\argv)} Pass a constant {\tt IntVect} variable.
\item {\tt CHF\_I1D(\argv,\lenv)} Pass a 1D array of {\tt int}s of length \lenv.
\item {\tt CHF\_CONST\_I1D(\argv,\lenv)} Pass a constant 1D array of {\tt int}s of length \lenv.
\item {\tt CHF\_R1D(\argv,\lenv)} Pass a 1D array of {\tt Real}s of length \lenv.
\item {\tt CHF\_CONST\_R1D(\argv,\lenv)} Pass a constant 1D array of {\tt Real}s of length \lenv.
\item {\tt CHF\_VI(\argv)} Pass a {\tt Vector$<$int$>$}.
\item {\tt CHF\_CONST\_VI(\argv)} Pass a constant {\tt Vector$<$int$>$}.
\item {\tt CHF\_VR(\argv)} Pass a {\tt Vector$<$Real$>$}.
\item {\tt CHF\_CONST\_VR(\argv)} Pass a constant {\tt Vector$<$Real$>$}.
\item {\tt CHF\_VC(\argv)} Pass a {\tt Vector$<$Complex$>$}.
\item {\tt CHF\_CONST\_VC(\argv)} Pass a constant {\tt Vector$<$Complex$>$}.
\item {\tt CHF\_FIA(\argv)} Pass a {\verb/BaseFab<int>/} .
\item {\tt CHF\_CONST\_FIA(\argv)} Pass a {\verb/const BaseFab<int>/} .
\item {\tt CHF\_FRA(\argv)} Pass a {\verb/BaseFab<Real>/} .
\item {\tt CHF\_CONST\_FRA(\argv)} Pass a {\verb/const BaseFab<Real>/}.
\item {\tt CHF\_FIA1(\argv,\compv)} Pass a single component of a {\verb/BaseFab<int>/}.
\item {\tt CHF\_CONST\_FIA1(\argv,\compv)} Pass a single {\tt const} component of a {\verb/BaseFab<int>/}.
\item {\tt CHF\_FRA1(\argv,\compv)} Pass a single component of a {\verb/BaseFab<Real>/}.
\item {\tt CHF\_CONST\_FRA1(\argv,\compv)} Pass a single {\tt const} component of a {\verb/BaseFab<Real>/}.
\item {\tt CHF\_BOX(\argv)} Pass a Box.  Boxes are always {\tt const}.
\item {\tt FORT\_$<${\em proc}$>$(\ldots)} Call the Fortran subroutine $<${\em proc}$>$ with the arguments specified.
\end{itemize}

The macros {\tt CHF(\_CONST)\_FIA}, {\tt CHF(\_CONST)\_FRA}, {\tt CHF(\_CONST)\_FIA1}, and\linebreak {\tt CHF(\_CONST)\_FRA1} all come in a version with the appendix {\tt \_SHIFT} and take an extra {\tt IntVect} describing the amount to shift the associated box before passing the argument to Fortran.  This can be used to align box data and/or change the reference frame.  No change is required to the corresponding Fortran declaration macro and using the shift macro has no effect on the original data.  In other words, there is no need to un-shift.  A common use is to shift all boxes to the positive quadrant so that coarsening can be achieved by simple integer division.  In the C++ code:

\begin{small}
\begin{verbatim}
{
  const int refRatio = 2;
  Box crBox (-IntVect::Unit, IntVect::Unit);
  Box fnBox = refine(crBox, refRatio);
  FArrayBox crFRA(crBox, 1);
  crFRA.setVal(0.);
  FArrayBox fnFRA(fnBox, 1);
  fnFRA.setVal(1.);

  const IntVect crShiftToZero = crBox.smallEnd();
  const IntVect fnShiftToZero = scale(crShiftToZero, refRatio);
  FORT_SUMFINE(CHF_FRA1_SHIFT(crFRA, 0, crShiftToZero),
               CHF_BOX_SHIFT(fnBox, fnShiftToZero),
               CHF_CONST_FRA1_SHIFT(fnFRA, 0, fnShiftToZero),
               CHF_CONST_INT(refRatio));
}
\end{verbatim}
\end{small}
In the Fortran code:

\begin{small}
\begin{verbatim}
      subroutine SUMFINE(
     &   CHF_FRA1[crFRA],
     &   CHF_BOX[fnBox],
     &   CHF_CONST_FRA1[fnFRA],
     &   CHF_CONST_INT[refRatio])

      integer CHF_AUTODECL[iFn]
      integer CHF_AUTODECL[iCr]

      CHF_AUTOMULTIDO[fnBox;iFn]
c        Coarsen iFn
         CHF_DTERM[
            iCr0 = iFn0/refRatio;
            iCr1 = iFn1/refRatio;
            iCr2 = iFn2/refRatio;
            iCr3 = iFn3/refRatio;
            iCr4 = iFn4/refRatio;
            iCr5 = iFn5/refRatio;]
         crFRA(CHF_AUTOIX[iCr]) = crFRA(CHF_AUTOIX[iCr]) +
     &      fnFRA(CHF_AUTOIX[iFn])
      CHF_ENDDO

      return
      end
\end{verbatim}
\end{small}

\section{Language support}

Chombo\ Fortran supports the Fortran standard language with a few
exceptions.  The exceptions include standard Fortran features that are not
supported and an extension to the standard that is required.  

\noindent Chombo\ Fortran does not support the following features of the
Fortran standard:
\begin{itemize}
\item {\tt REAL, DOUBLE PRECISION, COMPLEX} datatypes.  The only floating point
data\-type that is supported is {\tt REAL\_T}\@.  {\tt REAL\_T} is a Chombo\
Fortran extension to the Fortran standard.
\item  Appending ``*{\em $<$length$>$}'' to a datatype
is not supported.  This is not standard Fortran, but is a common
extension.
\item Non-void functions are not supported by Chombo Fortran.  Only
        subroutine statements are supported and those are only allowed
        with Chombo Fortran macros as arguments.
\end{itemize}

\noindent The code generated by  the Chombo\ Fortran preprocessor conforms
to the Fortran standard (ISO/IEC 1539:1991, ANSI X3.198-1992) with the
following exceptions:
\begin{itemize}
\item The code produced by ChF\ may violate the Fortran standard maximum
number of continuation lines in a statement (19).  If this occurs, it will
be necessary to provide a compiler option to increase the limit or change the
original Fortran code so that it produces fewer continuation lines, usually 
by breaking a single statement into several separate statements.

\item 
Chombo\ Fortran does not support input and output to the standard units
(i.e., 5,6,``*'') on all combinations of C++ and Fortran compilers.  Input
and output to files should work correctly in all systems.  This
problem is a fundamental one of mixed-language programming and cannot
be solved in any kind of a general way.
A special subroutine is provided which allows the 
Fortran code to print a special
message and terminate execution of the program.  This subroutine interfaces
with the {\tt MayDay} class in the Chombo\ C++ library.  The subroutine
has two versions, named {\tt MAYDAY\_ERROR} and {\tt MAYDAY\_ABORT}.  

\item
The code generated for any ChomboFortran subroutine
will contain an {\tt IMPLICIT NONE} statement so this
statement should not be used in the source code.  As a result, all variables
used in the subroutine must be explicitly declared else the code will not
compile successfully.

\end{itemize}

\section{Examples}

\subsection{Dot Product Example}
This routine multiplies each point of one BaseFab with the
corresponding point the other BaseFab over the input Box
and puts the result in the input Real.
\begin{small}
\begin{verbatim}
      subroutine DOTPRODUCT(
     &     CHF_REAL[dotprodout],
     &     CHF_CONST_FRA[afab],
     &     CHF_CONST_FRA[bfab],
     &     CHF_BOX[region])

      integer CHF_DDECL[i;j;k]
      integer nv,ncomp
      
      ncomp = CHF_NCOMP[afab]
      if(ncomp .ne. CHF_NCOMP[bfab]) then
          call MAYDAY_ERROR()
      endif

      dotprodout = zero
      do nv = 0, ncomp-1
         CHF_MULTIDO[region; i; j; k]
         
         dotprodout = dotprodout +
     &        afab(CHF_IX[i;j;k],nv)*
     &        bfab(CHF_IX[i;j;k],nv)
         
         CHF_ENDDO
      enddo

      return 
      end
\end{verbatim}
\end{small}
\subsection{RealVect and IntVect Example}

\begin{small}
\begin{verbatim}
       subroutine realVectTest(CHF_REALVECT[foo])
       
       CHF_DTERM[
       foo(0) = 1.0;
       foo(1) = 2.0;
       foo(2) = 3.0]
       
       return
       end     
       subroutine intVectTest(CHF_INTVECT[foo])
       
       CHF_DTERM[
       foo(0) = 1;
       foo(1) = 2;
       foo(2) = 3]
       
       return
       end     
\end{verbatim}
\end{small}

\subsection{Laplacian Example}
\label{subsec:LaplacianExample}
This subroutine produces a standard (3 point in one dimension,
5 point in two dimensions, and 7 point in three dimensions) discrete
Laplacian of the input BaseFab over the input box.
\begin{small}
\begin{verbatim}
      subroutine OPERATORLAP(
     &     CHF_FRA[lofphi],
     &     CHF_CONST_FRA[phi],
     &     CHF_BOX[region],
     &     CHF_CONST_REAL[dx])


      REAL_T dxinv,lphi
      integer n,ncomp,idir
      
      integer CHF_DDECL[ii,i;jj,j;kk,k]

      ncomp = CHF_NCOMP[phi]
      if(ncomp .ne. CHF_NCOMP[lofphi]) then
         call MAYDAY_ERROR()
      endif

      dxinv = one/(dx*dx)
      do n = 0, ncomp-1
         CHF_MULTIDO[region; i; j; k]

         lphi = zero
         do idir = 0, CH_SPACEDIM-1
         CHF_DTERM[
            ii = CHF_ID(idir, 0);
            jj = CHF_ID(idir, 1);
            kk = CHF_ID(idir, 2)]

            lphi = lphi +
     &           ( (phi(CHF_IX[i+ii;j+jj;k+kk],n)
     &           -  phi(CHF_IX[i   ;j   ;k   ],n))
     &           - (phi(CHF_IX[i   ;j   ;k   ],n)
     &           -  phi(CHF_IX[i-ii;j-jj;k-kk],n))
     &           )*(dxinv)
         enddo

         lofphi(CHF_IX[i;j;k],n) =  lphi

         CHF_ENDDO
      enddo

      return
      end
\end{verbatim}
\end{small}

This can also be expressed in a simpler way making full use of the
AUTO and OFFSET macros:
\begin{small}
\begin{verbatim}
      subroutine OPERATORLAP(
     &     CHF_FRA[lofphi],
     &     CHF_CONST_FRA[phi],
     &     CHF_BOX[region],
     &     CHF_CONST_REAL[dx])


      REAL_T dxinv,lphi
      integer n,ncomp,idir
      
      integer CHF_AUTODECL[i], CHF_AUTODECL[ii]

      ncomp = CHF_NCOMP[phi]
      if(ncomp .ne. CHF_NCOMP[lofphi]) then
         call MAYDAY_ERROR()
      endif

      dxinv = one/(dx*dx)
      do n = 0, ncomp-1
         CHF_AUTOMULTIDO[region; i]

         lphi = zero
         do idir = 0, CH_SPACEDIM-1
            CHF_AUTOID[ii;idir]

            lphi = lphi +
     &           ( (phi(CHF_OFFSETIX[i;+ii],n)
     &           -  phi(CHF_AUTOIX[i],n))
     &           - (phi(CHF_AUTOIX[i],n)
     &           -  phi(CHF_OFFSETIX[i;-ii],n))
     &           )*(dxinv)
         enddo

         lofphi(CHF_AUTOIX[i],n) =  lphi

         CHF_ENDDO
      enddo

      return
      end
\end{verbatim}
\end{small}

\subsection{Internal Procedure Example}
\label{subsec:InternalProcExample}
This example demonstrates use of an internal procedure in a ``*.ChF'' file.  Ordinary Fortran arguments could have also been declared for the internal procedure.
\begin{small}
\begin{verbatim}
      subroutine testNoChFCall(
     &   CHF_BOX[box])

      integer CHF_AUTODECL[i]
      integer c

      c = 0
      call countCells
      write(*,*) c
      call countCells
      write(*,*) c

      return

      contains

      subroutine countCells ! NO_CHF
         CHF_AUTOMULTIDO[box;i]
            c = c+1
         CHF_ENDDO
      end subroutine

      end
\end{verbatim}
\end{small}

\section{Landmines}

This section is intended to point out some known uses of Chombo
Fortran that will result in errors.

\begin{itemize}
\item Be aware that using C++ and Fortran together defeats most bounds
checkers.  If you step out of bounds in a Fortran, as a rule, your
bounds checker will not save you.  This holds for both Fortran and
Chombo Fortran.

\item Combining Fortran and Chombo Fortran in the same file is a bad
idea, with the exception of internal procedures as shown in the example
of section~\ref{subsec:InternalProcExample}. The Chombo Fortran parser
keys on the word ``subroutine,'' and
dissects the argument list as described above.  If ordinary Fortran
subroutines are  put into a Chombo Fortran file, the parser will fail
to produce correct code.  To use both Fortran and Chombo Fortran in
the same application, put them into separate files.  The Chombo
makefile system recognizes files with ``.F'' extensions as Fortran
and files with ``.ChF'' extensions as Chombo Fortran files.

\item Send constants to Chombo Fortran (or plain Fortran, for that
matter) using temporary variables.  The C++ macros in Chombo Fortran
are precisely that--macros.  If you insert an explicit constant in a
Chombo Fortran call, the macro will simply try to take the address of
the explicit constant, resulting in undefined behavior.  Say you want
to send the number four to a Chombo Fortran routine.  Here are both
the incorrect and correct ways to do so.
\begin{small}
\begin{verbatim}
//error! gets tranlated to senseless:
//myfunc_(&4);
FORT_MYFUNC(CHF_CONST_INT(4));

//correct.  address sent to Fortran is legal.  This gets translated to 
//myfunc_(&ivar);
int ivar = 4;
FORT_MYFUNC(CHF_CONST_INT(ivar));
\end{verbatim}
\end{small}
The exception to this is the 2nd argument to the {\tt CHF\_*1D} macros, 
which can be a literal constant.

\item The arguments of a ChF Fortran macro must be enclosed
in square brackets and separated by semicolons.
Commas between the brackets
will pass through to Fortran, as in the example in section
\ref{subsec:LaplacianExample} where
{\tt CHF\_DDECL[ii,i;jj,j;kk,k]} translates to
{\tt ii,i,jj,j,kk,k} or {\tt ii,i,jj,j}.
The one apparent exception is
{\tt CHF\_ID($<${\em dim1}$>$,$<${\em dim2}$>$)}, but as noted above,
{\tt CHF\_ID} is a matrix, not a macro.

\item Chombo {\tt Vector}s and 1D arrays {\em always} start at index 0.  You
{\em cannot} call {\tt CHF\_LBOUND} on a {\tt Vector} or 1D array.  The value
returned from {\tt CHF\_UBOUND} on a 1D array will always be one less than the
length value passed as the 2nd argument in the C++ call to {\tt CHF\_*1D}.

\end{itemize}
%\end{document}

