
\section{Class ProblemDomain}

{\tt ProblemDomain} is a class to handle interaction with boundary
conditions at the edge of the computational domain, either physical
boundary conditions or periodic ones. This class contains much of the
functionality of the Box class, since logically the computational
domain is generally a Box.   
  
Intersection with a ProblemDomain object will result in only removing
regions which are outside the physical domain in non-periodic 
directions. Regions outside the logical computational domain in
periodic directions will be treated as ghost cells which can be filled
with an exchange() function or through suitable interpolation from a
coarser domain. 

Since ProblemDomain will contain a Box, it is a dimension-dependent
class, so SpaceDim must be defined appropriately when compiling.   

Note that this implementation of ProblemDomain is inherently
cell-centered. 

The user interface for {\tt ProblemDomain} is as follows:

\begin{itemize}
\item
\begin{verbatim}
ProblemDomain()
\end{verbatim}
Default constructor -- the object is defined in an unusable state
until the user calls the {\tt define} function.

\item
\begin{verbatim}
ProblemDomain(const Box& domBox, const bool* isPeriodic)
\end{verbatim}
Full constructor.  Places the {\tt BRMeshRefine} object in a usable
state. \\
\\
{\bf Arguments:} 
\begin{itemize} 
\item  \verb/domBox/ Computational domain.
\item  \verb/isPeriodic/ SpaceDim array of bools which defines
whether BC's are physical or periodic in each coordinate direction.   

\end{itemize}


\item
\begin{verbatim}
ProblemDomain(const Box& domBox)
\end{verbatim}
Partial constructor, creates non-periodic (in any coordinate
direction) ProblemDomain. \\
\\
{\bf Arguments:}
\begin{itemize}
\item \verb/domBox/ Computational domain.
\end{itemize}

\item
\begin{verbatim}
ProblemDomain(const IntVect& small, const IntVect& big, const
              bool* isPeriodic)
\end{verbatim}
Full constructor, creates ProblemDomain. \\
\\
{\bf Arguments:}
\begin{itemize}
\item \verb/small/ Location of lower-left corner of domain box
\item \verb/big/  Location of upper-right corner of domain box
\item  \verb/isPeriodic/ SpaceDim array of bools which defines
whether BC's are physical or periodic in each coordinate direction.   
\end{itemize}

\item
\begin{verbatim}
ProblemDomain(const IntVect& small, const IntVect& big)
\end{verbatim}
Partial constructor, creates non-periodic (in any coordinate
direction) ProblemDomain. \\
\\
{\bf Arguments:}
\begin{itemize}
\item \verb/small/ Location of lower-left corner of domain box
\item \verb/big/  Location of upper-right corner of domain box
\end{itemize}

\item
\begin{verbatim}
ProblemDomain(const IntVect& small, const int* vec_len,
              const bool* isPeriodic)
\end{verbatim}
Full constructor, creates ProblemDomain. \\
\\
{\bf Arguments:}
\begin{itemize}
\item \verb/small/ Location of lower-left corner of domain box
\item \verb/vec_len/ Size of domain in each direction.
\item  \verb/isPeriodic/ SpaceDim array of bools which defines
whether BC's are physical or periodic in each coordinate direction.   
\end{itemize}

\item
\begin{verbatim}
ProblemDomain(const IntVect& small, const int* vec_len)
\end{verbatim}
Partial constructor, creates ProblemDomain with non-periodic boundary
conditions by default. \\
\\
{\bf Arguments:}
\begin{itemize}
\item \verb/small/ Location of lower-left corner of domain box
\item \verb/vec_len/ Size of domain in each direction.
\end{itemize}


\item
\begin{verbatim}
ProblemDomain(const ProblemDomain& src)
\end{verbatim}
Copy  constructor  


\item 
\begin{verbatim}
const Box& domainBox() const 
\end{verbatim}
Returns logical computational domain. 

\item
\begin{verbatim}
bool isPeriodic(int dir) const
\end{verbatim}
Returns {\tt true} if boundary condition is periodic in direction {\tt
dir}.


\item
\begin{verbatim}
bool isPeriodic() const
\end{verbatim}
Returns {\tt true} if boundary condition is periodic in any direction.

\item
\begin{verbatim}
ShiftIterator shiftIterator() const
\end{verbatim}
Returns ShiftIterator for this problem domain.  ShiftIterator is a
utility class to aid with periodic boundary conditions whose use is
mostly internal to BoxLayout, Copier, etc which allows looping over
IntVects which used to shift the domain for enforcing periodic
boundary conditions.

\item
\begin{verbatim}
bool isEmpty() const
\end{verbatim}
Returns {\tt true} if this ProblemDomain has an empty domainBox.


\item
\begin{verbatim}
bool contains(const IntVect& p) const
\end{verbatim}
Returns true if argument is contained within this ProblemDomain.  An
empty ProblemDomain does not contain and is not contained by any
ProblemDomain.  In a periodic direction, all locations are contained,
since a periodic domain is an infinite domain. If periodic in all
directions, this will always return {\tt true}.


\item
\begin{verbatim}
bool contains(const Box& b) const
\end{verbatim}
Returns true if argument is contained within this ProblemDomain.  An
empty ProblemDomain does not contain and is not contained by any
ProblemDomain.  In a periodic direction, all locations are contained,
since a periodic domain is an infinite domain. If periodic in all
directions, this will always return {\tt true}.


\item
\begin{verbatim}
bool intersects( const Box& a_box) const
\end{verbatim}
Returns true if this ProblemDomain and the argument have non-null
intersections.  It is an error if a\_box is not cell-centered.  An
empty ProblemDomain does not intersect any Box.  Boxes always
intersect in periodic dimensions, since a periodic domain is an
infinite domain.  If periodic in all directions, this will always
return {\tt true}.


\item
\begin{verbatim}
bool intersectsNotEmpty (const Box& a_box) const
\end{verbatim}
Returns true if this ProblemDomain and the argument have non-null
intersections.  It is an error if a\_box is not cell-centered.
This routine does not perform the check to see if *this or b are
empty boxes.  It is the callers responsibility to ensure that
this never happens.  If you are unsure, the use the .intersects(..) 
routine.  In periodic directions, will always return true.

\item
\begin{verbatim}
bool intersects(const Box& box1, const Box& box2) const
\end{verbatim}
Returns true of box1 and box2 and any of their periodic images
intersect. (This is useful for checking disjointness).

\item
\begin{verbatim}
ProblemDomain& operator= (const ProblemDomain& b)
\end{verbatim}
Assignment operator.

\item
\begin{verbatim}
void setPeriodic(int a_dir, bool a_isPeriodic)
\end{verbatim}
Sets whether boundary condition is periodic in direction a\_dir ({\tt
true} is periodic).


\item
\begin{verbatim}
friend
Box bdryLo(const ProblemDomain& a_pd, int a_dir, int a_len=1)
\end{verbatim}
Returns the edge-centered box (in direction a\_dir) defining the low
side of the domainBox in the argument ProblemDomain.  The neighbor of
an empty ProblemDomain is an empty Box of the appropriate type.  If
dir is a periodic direction, will return an empty box. \\ 
\\
{\bf Arguments:}
\begin{itemize}
\item \verb/a_pd/ input ProblemDomain
\item \verb/a_dir/ normal direction of edge to return.  Directions are
zero-based and must be $0 \leq${\tt a\_dir} $<${\tt SpaceDim}.
\item \verb/a_len/ Width of returned box in normal direction {\tt
a\_dir}.
\end{itemize}


\item
\begin{verbatim}
friend
Box bdryHi(const ProblemDomain& a_pd, int a_dir, int a_len=1)
\end{verbatim}
Returns the edge-centered box (in direction a\_dir) defining the high
side of the domainBox in the argument ProblemDomain.  The neighbor of
an empty ProblemDomain is an empty Box of the appropriate type.  If
dir is a periodic direction, will return an empty box. \\ 
\\
{\bf Arguments:}
\begin{itemize}
\item \verb/a_pd/ input ProblemDomain
\item \verb/a_dir/ normal direction of edge to return.  Directions are
zero-based and must be $0 \leq${\tt a\_dir} $<${\tt SpaceDim}.
\item \verb/a_len/ Width of returned box in normal direction {\tt
a\_dir}.
\end{itemize}


\item
\begin{verbatim}
friend
Box adjCellLo(const ProblemDomain& a_pd, int a_dir, int a_len=1)
\end{verbatim}
Returns the cell-centered box (in direction a\_dir) adjacent to the low
side of the domainBox in the argument ProblemDomain.  The neighbor of
an empty ProblemDomain is an empty Box of cell-centered type.  If
dir is a periodic direction, will return an empty box. \\ 
\\
{\bf Arguments:}
\begin{itemize}
\item \verb/a_pd/ input ProblemDomain
\item \verb/a_dir/ normal direction of side to return.  Directions are
zero-based and must be $0 \leq${\tt a\_dir} $<${\tt SpaceDim}.
\item \verb/a_len/ Width of returned box in normal direction {\tt
a\_dir}.
\end{itemize}

\item
\begin{verbatim}
friend
Box adjCellLo(const ProblemDomain& a_pd, int a_dir, int a_len=1)
\end{verbatim}
Returns the cell-centered box (in direction a\_dir) adjacent to the low
side of the domainBox in the argument ProblemDomain.  The neighbor of
an empty ProblemDomain is an empty Box of cell-centered type.  If
dir is a periodic direction, will return an empty box. \\ 
\\
{\bf Arguments:}
\begin{itemize}
\item \verb/a_pd/ input ProblemDomain
\item \verb/a_dir/ normal direction of side to return.  Directions are
zero-based and must be $0 \leq${\tt a\_dir} $<${\tt SpaceDim}.
\item \verb/a_len/ Width of returned box in normal direction {\tt
a\_dir}.
\end{itemize}

\item
\begin{verbatim}
Box operator& (const Box& b) const
\end{verbatim}
Returns the Box that is the intersection of the input box {\tt b} and
the ProblemDomain.  The Box {\tt b} must be cell-centered.  The
intersection of an empty ProblemDomain and any box is the Empty Box.
This operator does nothing in periodic directions (since a periodic
domain is an infinite domain).

\item
\begin{verbatim}
ProblemDomain& refine(int a_refinement_ratio)
\end{verbatim}
Modifies this ProblemDomain by refining it by (the positive) {\tt
a\_refinement\_ratio}. The empty ProblemDomain is not modified by this
function. 

\item
\begin{verbatim}
friend
ProblemDomain refine(const ProblemDomain& a_probdomain,
                     int a_refinement_ratio)
\end{verbatim}
Returns a ProblemDomain that is the argument ProblemDomain refined by
(the positive) {\tt a\_refinement\_ratio}.  If {\tt a\_probdomain} is
an Empty ProblemDomain, then an empty problem domain is produced.

\begin{verbatim}
ProblemDomain& refine(const IntVect& a_refinement_ratio)
\end{verbatim} 
Modifies this ProblemDomain by refining it by the given refinement
ratio in each direction.  The empty ProblemDomain is not modified by
this function.

\begin{verbatim}
friend 
ProblemDomain refine (const ProblemDomain&     a_probdomain,
                     const IntVect& a_refinement_ratio)
\end{verbatim}
Returns a ProblemDomain that is the argument ProblemDomain refined by
(the positive) {\tt a\_refinement\_ratio}.  If {\tt a\_probdomain} is
an Empty ProblemDomain, then an empty problem domain is produced.


\item
\begin{verbatim}
ProblemDomain& coarsen(int a_refinement_ratio)
\end{verbatim}
Modifies this ProblemDomain by coarsening it by (the positive) {\tt
a\_refinement\_ratio}. The empty ProblemDomain is not modified by this
function. 

\item
\begin{verbatim}
friend
ProblemDomain coarsen(const ProblemDomain& a_probdomain,
                     int a_refinement_ratio)
\end{verbatim}
Returns a ProblemDomain that is the argument ProblemDomain coarsened by
(the positive) {\tt a\_refinement\_ratio}.  If {\tt a\_probdomain} is
an Empty ProblemDomain, then an empty problem domain is produced.

\begin{verbatim}
ProblemDomain& coarsen(const IntVect& a_refinement_ratio)
\end{verbatim} 
Modifies this ProblemDomain by coarsening it by the given refinement
ratio in each direction.  The empty ProblemDomain is not modified by
this function.

\begin{verbatim}
friend 
ProblemDomain refine (const ProblemDomain&     a_probdomain,
                     const IntVect& a_refinement_ratio)
\end{verbatim}
Returns a ProblemDomain that is the argument ProblemDomain coarsened by
(the positive) {\tt a\_refinement\_ratio}.  If {\tt a\_probdomain} is
an Empty ProblemDomain, then an empty problem domain is produced.



\begin{verbatim}
friend
std::ostream& operator<< (std::ostream& os, const ProblemDomain& b)
\end{verbatim}
Writes and ASCII representation to the ostream.

\begin{verbatim}
friend
std::istream& operator<< (std::istream& is, ProblemDomain& b)
\end{verbatim}
read from istream.



\end{itemize}


