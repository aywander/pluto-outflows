\section{Elliptic Examples}

We provide several examples of elliptic operators that conform to the
{\tt AMRLevelOp} interface
\begin{itemize}
\item {\tt AMRPoissonOp} is used to solve Poisson's equation with
  constant coefficients.
\item {\tt ResistivityOp} is used to solve the variable-coefficient
  resistivity equations that arise from MHD.
\item {\tt ViscousTensorOp} is used to solve the variable-coefficient
  elliptic equations that arise when solving the compressible Navier-Stokes
  equations with variable viscosity.
\end{itemize}
All of these examples use the boundary condition interface described 
in section \ref{sec::AMRPBC}.   


\subsection{AMRPoisson}

We provide an {\tt AMRPoisson}, an example of an {\tt AMRLevelOp}
class.   This class is designed to solve 
\begin{equation}
(\alpha I + \beta \Delta)\phi = \rho
\label{AMRPoissonEQ}
\end{equation}
where $\alpha$ and $\beta$ are constants.     The discretization of
the Laplacian is the standard centered-difference approximation
$$
(\Delta^h \phi)_\ibold = \frac{1}{h^2}\sum\limits^{\Dim-1}_{d=0} 
(\phi_{\ibold+\ebold^d} + \phi_{\ibold-\ebold^d} - 2 \phi_\ibold)
$$
Domain boundary conditions are enforced by setting ghost cell values 
as described in section \ref{sec::AMRPBC}.
Fluxes through coarse-fine interfaces are needed for the
Martin-Cartwright algorithm.  The flux $F$ through a face at $\ibold +
\half \ebold^d$ is given by
$$
F_{\ibold + \half \ebold^d} = \frac{\beta}{h}(\phi_{\ibold+\ebold^d} - \phi_{\ibold})
$$


\subsubsection{AMRPoisson Factory Interface}


An AMRPoissonOpFactory needs to be defined using the following 
function.
%\begin{scriptsize}
\begin{verbatim}
  void define(const ProblemDomain& coarseDomain,
              const Vector<DisjointBoxLayout>& grids,
              const Vector<int>& refRatios,
              const Real&        coarsedx,
              BCFunc bc,
              Real   alpha = 0.,
              Real   beta  = 1.);

\end{verbatim}
\begin{itemize}
  \item  {\tt  coarseDomain} is the domain at the coarsest level.
  \item  {\tt  grids} is the AMR  hierarchy.
  \item  {\tt  refRatios} are the refinement ratios between levels.
    The ratio lives with the coarser level    so {\tt refRatios[4]} is
    the ratio between  AMR levels 4 and 5.
  \item  {\tt  coarseDx} is the grid spacing at the coarsest level.
  \item {\tt BCFunc} holds the boundary conditions.
  \item {\tt alpha} is the coefficient of the identity.
  \item {\tt beta} is the coefficient of the Laplacian.
\end{itemize}
%\end{scriptsize}
  
\subsubsection{Code fragment}

For those of us who find code easier to read than documents, we
provide a simplified  example of  how to use {\tt  AMRMultiGrid} 
and {\tt AMRPoissonOp}. In the example below, we start with a known
AMR hierarchy and a right-hand side and we solve (\ref{AMRPoissonEQ}).
For a description of the boundary condition routine along with its
code fragment, see subsection \ref{sec::AMRPBC}.    Complete
examples (along with convergence tests) can be found in 
{\tt  Chombo/example/AMRPoisson}. 
\begin{scriptsize}
\begin{verbatim}

/**
   solveElliptic solves (alpha I + beta Laplacian) phi = rhs
   using AMRMultiGrid and AMRPoissonOp
   
   Inputs:  
   rhs:  Right-hand side of the solve over the level.
   grids: AMRHierarchy of grids
   refRatio:  refinement ratios
   level0Domain:  domain at the coarsest AMR level 
   coarsestDx: grid spacing at the coarsest level
   alpha: identity  coefficient
   beta:  Laplacian coefficient

   Outputs:
   phi = (alpha I + beta Lapl)^{-1}(rhs)
 */
void solveElliptic( Vector<LevelData<FArrayBox>* >&      phi,
                    const Vector<LevelData<FArrayBox>* > rhs,
                    const Vector<DisjointBoxLayout>&     grids,
                    const Vector<int>&                   refRatios,
                    const ProblemDomain&                 level0Domain,
                    Real alpha, Real beta, Real coarsestDx)

{
  int numlevels = rhs.size();
  
  //define the operator factory
  AMRPoissonOpFactory opFactory;
  opFactory.define(level0Domain,
                   grids, refRatios, coarsestDx, 
                   &ParseBC, alpha, beta);

  //this is the solver we shall use
  AMRMultiGrid<LevelData<FArrayBox> > solver;


  //this is the solver for the bottom of the muligrid v-cycle
  BiCGStabSolver<LevelData<FArrayBox> >   bottomSolver;
  //bicgstab can be noisy
  bottomSolver.m_verbosity = 0;

  //define the solver
  solver.define(level0Domain, opFactory, &bottomSolver, numlevels);

  //we want to solve over the whole hierarchy
  int lbase = 0;
  //so solve already.
  solver.solve(phi, rhs, numlevels-1, lbase);
}

\end{verbatim}
\end{scriptsize}



\newcommand{\vb}{\vec B}
\newcommand{\vx}{\vec x}

\subsection{ResistivityOp}


{\tt ResistivityOp} is the {\tt AMRLevelOp}-derived class which solves
the variable-coefficient equation
For notation's sake, what we are solving is
$$
L \vb = \alpha \vb + \beta \nabla \cdot F = \rho.
$$
$\alpha$ and $\beta$ are constants and $F$ is given by 
$$
F = \eta(\nabla \vb - \nabla \vb^T +  I \nabla \cdot \vb)
$$
where $I$ is the identity matrix and $\eta = \eta(\vx) \ge 0$.
The discretization of the flux divergence is as follows.
$$
(\nabla \cdot F)^h_\ibold = \frac{1}{h}\sum\limits^D_{d=1} 
(F_{\ibold+\half \ebold^d}  F\phi_{\ibold-\half \ebold^d})
$$
We discretize normal components of the face-centered gradient  using 
an average of cell-centered gradients for tangential components and a
centered-difference approximation to the normal gradient.
$$
(\nabla \vb)^{d'}_{\ibold + \half \ebold^d} = \left( \begin{array}{c}
                      \frac{1}{h}(\vb_{\ibold+ \ebold^d}  -\vb_{\ibold})  \mbox{ if } d = d'\\
                      \frac{1}{2}((\nabla \vb)^{d'}_{\ibold+\ebold^d}  +
                      (\nabla \vb)^{d'}_{\ibold})  \mbox{ if } d \neq d'
                      \end{array}
                      \right)
$$
where 
$$
(\nabla \vb)^{d}_{\ibold} = \frac{1}{2 h}(\vb_{\ibold + \ebold^d} - \vb_{\ibold - \ebold^d}).
$$

\subsubsection{ResistivityOp Factory Interface}


An {\tt ResistivityOpFactory} needs to be defined using the following 
constructor.
%\begin{scriptsize}
\begin{verbatim}
  ResistivityOpFactory(const Vector<DisjointBoxLayout>&                     grids,
                       const Vector<RefCountedPtr<LevelData<FluxBox> > >&   eta,
                       Real                                                 alpha, 
                       Real                                                 beta,
                       const Vector<int>&                                   refRatios,
                       const ProblemDomain&                                 domainCoar,
                       const Real&                                          dxCoar,
                       BCFunc                                               bc);
\end{verbatim}
\begin{itemize}
  \item  {\tt  domainCoar} is the domain at the coarsest level.
  \item  {\tt  grids} is the AMR  hierarchy.
  \item  {\tt  refRatios} are the refinement ratios between levels.
    The ratio lives with the coarser level    so {\tt refRatios[4]} is
    the ratio between  AMR levels 4 and 5.
  \item  {\tt  dxCoar} is the grid spacing at the coarsest level.
  \item {\tt bc} holds the boundary conditions.
  \item {\tt alpha} is the coefficient of the identity.
  \item {\tt beta} is the coefficient of the divergence of the flux.
  \item {\tt eta} is the variable coeff function.  This ought to to be
  positive if you expect multigrid to converge.
\end{itemize}
%\end{scriptsize}

\subsection{ViscousTensorOp}


{\tt ViscousTensorOp} is the {\tt AMRLevelOp}-derived class which solves
the variable-coefficient equation
For notation's sake, what we are solving is
$$
L \vb = \alpha \vb + \beta \nabla \cdot F = \rho.
$$
$\alpha = \alpha(\vx)$ and $\beta = \beta(\vx)$. $F$ is given by 
$$
F = \eta(\nabla \vb + \nabla \vb^T) +  \lambda (I \nabla \cdot \vb)
$$
where $I$ is the identity matrix $\eta = \eta(\vx)$, and $\lambda = \lambda(\vx)$.
The discretization of the flux divergence is as follows.
$$
(\nabla \cdot F)^h_\ibold = \frac{1}{h}\sum\limits^D_{d=1} 
(F_{\ibold+\half \ebold^d} - F_{\ibold-\half \ebold^d})
$$
We discretize normal components of the face-centered gradient  using 
an average of cell-centered gradients for tangential components and a
centered-difference approximation to the normal gradient.
$$
(\nabla \vb)^{d'}_{\ibold + \half \ebold^d} = \left( \begin{array}{c}
                      \frac{1}{h}(\vb_{\ibold+ \ebold^d}  -\vb_{\ibold})  \mbox{ if } d = d'\\
                      \frac{1}{2}((\nabla \vb)^{d'}_{\ibold+\ebold^d}  +
                      (\nabla \vb)^{d'}_{\ibold})  \mbox{ if } d \neq d'
                      \end{array}
                      \right)
$$
where 
$$
(\nabla \vb)^{d}_{\ibold} = \frac{1}{2 h}(\vb_{\ibold + \ebold^d} - \vb_{\ibold - \ebold^d}).
$$

\subsubsection{ViscousTensorOp Factory Interface}


An {\tt ViscousTensorOpFactory} needs to be defined using the following 
constructor.
%\begin{scriptsize}
\begin{verbatim}
  ViscousTensorOpFactory(const Vector<DisjointBoxLayout>&                     a_grids,
                         const Vector<RefCountedPtr<LevelData<FluxBox> > >&   a_eta,
                         const Vector<RefCountedPtr<LevelData<FluxBox> > >&   a_lambda,
                         Real                                                 a_alpha, 
                         const Vector<RefCountedPtr<LevelData<FArrayBox> > >& a_beta,
                         const Vector<int>&                                   a_refRatios,
                         const ProblemDomain&                                 a_domainCoar,
                         const Real&                                          a_dxCoar,
                         BCFunc                                               a_bc);
\end{verbatim}
\begin{itemize}
  \item  {\tt  domainCoar} is the domain at the coarsest level.
  \item  {\tt  grids} is the AMR  hierarchy.
  \item  {\tt  refRatios} are the refinement ratios between levels.
    The ratio lives with the coarser level    so {\tt refRatios[4]} is
    the ratio between  AMR levels 4 and 5.
  \item  {\tt  dxCoar} is the grid spacing at the coarsest level.
  \item {\tt bc} holds the boundary conditions.
  \item {\tt alpha} is the coefficient of the identity.
  \item {\tt beta} is the coefficient of the divergence of the flux.
  \item {\tt eta} is the variable coeff function that multiplies the gradient. 
  \item {\tt lambda} is the variable coeff function that multiplies the divergence. 
\end{itemize}
%\end{scriptsize}
  

\subsection{Boundary Condition Interface}
\label{sec::AMRPBC}
The value of the boundary
condition is described by the {\tt BCValueFunc} function interface.
\begin{scriptsize}
\begin{verbatim}
/* Given
   pos [x,y,z] position on center of cell edge
   int dir direction, x being 0
   int side -1 for low, +1 = high,
   fill in the values array */
typedef void(*BCValueFunc)(Real*           pos,
                           int*            dir,
                           Side::LoHiSide* side,
                           Real*           value);
\end{verbatim}
\end{scriptsize}
The boundary condition function must conform to the {\tt BCFunc}
specification. We provide both Dirichlet and Neumann boundary
condition examples.
\begin{scriptsize}
\begin{verbatim}
   /* Function interface for ghost cell boundary conditions
   of EBAMRPoissonOp.   If you are using Neumann or Dirichlet 
   boundary conditions,   it is easiest to use the functions 
   provided. */
typedef void(*BCFunc)(FArrayBox&           state,
                      const Box& valid,
                      const ProblemDomain& domain,
                      Real                 dx,
                      bool                 homogeneous);
\end{verbatim}
\end{scriptsize}
A simple example of a custom boundary condition is given below.
\begin{scriptsize}
\begin{verbatim}
/* this is the bc value func */
void ParseValue(Real* pos,
                int* dir,
                Side::LoHiSide* side,
                Real* values)
{
  ParmParse pp;
  Real bcVal;
  pp.get("bc_value",bcVal);
  values[0]=bcVal;
}


/*Use ParmParse to select boundary condtions */ 
/* this is the bcfunc */
void ParseBC(FArrayBox& state,
             const Box& valid,
             const ProblemDomain& domain,
             Real dx,
             bool homogeneous)
{
  if(!domain.domainBox().contains(state.box()))
    {
      std::vector<int>  bcLo, bcHi;
      ParmParse pp;
      pp.getarr("bc_lo", bcLo, 0, SpaceDim);
      pp.getarr("bc_hi", bcHi, 0, SpaceDim);

      Box valid = valid;
      for(int i=0; i<CH_SPACEDIM; ++i)
        {
          Box ghostBoxLo = adjCellBox(valid, i, Side::Lo, 1);
          Box ghostBoxHi = adjCellBox(valid, i, Side::Hi, 1);
          if(!domain.domainBox().contains(ghostBoxLo))
            {
              if(bcLo[i] == 1)
                {
                  pout() << "const neum bcs lo for direction " << i << endl;
                  NeumBC(state,
                         valid,
                         dx,
                         homogeneous,
                         ParseValue,
                         i,
                         Side::Lo);
                }
              else if(bcLo[i] == 0)
                {
                  pout() << "const diri bcs lo for direction " << i << endl;
                  DiriBC(state,
                         valid,
                         dx,
                         homogeneous,
                         ParseValue,
                         i,
                         Side::Lo);
                  
                }
              else
                {
                  MayDay::Error("bogus bc flag lo");
                }
            }

          if(!domain.domainBox().contains(ghostBoxHi))
            {
              if(bcHi[i] == 1)
                {
                  pout() << "const neum bcs hi for direction " << i << endl;
                  NeumBC(state,
                         valid,
                         dx,
                         homogeneous,
                         ParseValue,
                         i,
                         Side::Hi);
                }
              else if(bcHi[i] == 0)
                {
                  pout() << "const diri bcs hi for direction " << i << endl;
                  DiriBC(state,
                         valid,
                         dx,
                         homogeneous,
                         ParseValue,
                         i,
                         Side::Hi);
                }
              else
                {
                  MayDay::Error("bogus bc flag hi");
                }
            }
        }
    }
}
\end{verbatim}
\end{scriptsize}
