\subsection{Class {\tt LevelFluxRegister}}

{\tt LevelFluxRegister} manages the manipulations at coarse-fine boundaries
associated with maintaining conservation form of cell-centered
discretizations of the divergence operator, using the algorithm
described in section \ref{sec:fluxreg}. Unlike the previous operators, 
{\tt LevelFluxRegister} holds data, corresponding to the flux register 
$\delta \vec{F^f}$ defined in section \ref{sec:fluxreg}. 
The class also manages the manipulation of that data. 

The user interface for {\tt LevelFluxRegister} is as follows.

\begin{itemize}

\item
\begin{verbatim}
void define(const DisjointBoxLayout& a_dbl,
            const DisjointBoxLayout& a_dblCoarse,
            const ProblemDomain& a_dProblem,
            int a_nRefine,
            int a_nComp);

void define(const DisjointBoxLayout& a_dbl,
            const DisjointBoxLayout& a_dblCoarse,
            const Box& a_dProblem,
            int a_nRefine,
            int a_nComp);
\end{verbatim}
Defines the internal state of the flux register, allocating space for
the register itself, as well as the indexing information required to
perform the other operations. 
\\ {\bf Arguments:} 
  \begin{itemize}
  \item
  \verb/a_dbl/ (not modified):
  The grids at the current level.        
  \item
  \verb/a_dblCoarse/ (not modified):
  The grids at the next coarser level in the AMR hierarchy.
  \item
  \verb/a_dProblem/ (not modified):
  The problem domain at the current level.
  \item
  \verb/a_nRefine/ (not modified):
  The refinement ratio between this level and the next coarser level.
  \item
  \verb/a_nComp/ (not modified):
  The number of variables used in the computation.
  \end{itemize}

\item
{\tt void setToZero()} Initializes the register to all zeros.

\item
\begin{verbatim}
void incrementCoarse(FArrayBox& a_coarseFlux,
                     Real a_scale,
                     const DataIndex& a_coarseDataIndex,
                     const Interval& a_srcInterval,
                     const Interval& a_dstInterval,
                     int a_dir);
\end{verbatim}
Increments the register with data from \verb/a_coarseFlux/, multiplied by 
\verb/a_scale/ ($\alpha$):
$\delta F^f_d := \delta F^f_d + \alpha F^c_d$, for all of the d-faces where
the input flux (defined on a single rectangle) coincides with the d-faces on
which the flux register is defined. \verb/a_coarseFlux/ contains fluxes in
the \verb/a_dir/ direction for the grid \verb/a_dblCoarse[a_coarsePatchIndex]/.
Only the registers corresponding to the low faces of
\verb/a_dblCoarse[a_coarseDataIndex]/ in the \verb/a_dir/ direction
are incremented (this avoids double-counting at coarse-coarse interfaces.
\verb/a_srcInterval/ gives the \verb/Interval/ of components of
\verb/a_coarseFlux/ that correspond to \verb/a_dstInterval/ of components
of the flux register.
\\ {\bf Arguments:} 
  \begin{itemize}
  \item
  \verb/a_coarseFlux/ (not modified):
  Flux to put into the flux register.
  This is not \verb/const/ because its box is shifted back and forth -
  no net change occurs.
  \item
  \verb/a_scale/ (not modified):
  Factor by which to multiply \verb/a_coarseFlux/ in flux register.
  \item
  \verb/a_coarseDataIndex/ (not modified):
  DataIndex which corresponds to which box in the coarse-level
  \verb/DisjointBoxLayout/ (\verb/a_dblCoarse/ in the \verb/define/
  function) over which \verb/a_coarseFlux/ was calculated.
  \item
  \verb/a_srcInterval/ (not modified):
  The \verb/Interval/ of components to put into the flux register.
  \item
  \verb/a_dstInterval/ (not modified):
  The \verb/Interval/ of components of the flux register which are
  incremented by the flux data. Should have the same size as
  \verb/a_srcInterval/. 
  \item
  \verb/a_dir/ (not modified):
  Direction of faces upon which fluxes live.
  \end{itemize}

\item
\begin{verbatim}
void incrementFine(FArrayBox& a_fineFlux,
                   Real a_scale,
                   const DataIndex& a_finePatchIndex,
                   const Interval& a_srcInterval,
                   const Interval& a_dstInterval,
                   int a_dir,
                   Side::LoHiSide a_sd);
\end{verbatim}
Increments the register with the average over each face of data from 
\verb/a_fineFlux/, scaled by \verb/a_scale/ ($\alpha$):
$\delta F^f_d = \delta F^f_d + \alpha <F^f_d>$, for all of the d-faces
where the input flux (defined on a single rectangle) covers the
d-faces on which the flux register is defined.
\verb/a_fineFlux/ contains fluxes in the \verb/a_dir/
direction for the grid \verb/a_dbl[a_finePatchIndex]/. Only the register
corresponding to the direction \verb/a_dir/ and the side \verb/a_sd/ is 
initialized. \verb/a_srcInterval/ and \verb/a_dstInterval/ are as
above.
\\ {\bf Arguments:} 
  \begin{itemize}
  \item
  \verb/a_fineFlux/ (not modified):
  Flux to put into the flux register.
  This is not \verb/const/ because its box is shifted back and forth -
  no net change occurs.
  \item
  \verb/a_scale/ (not modified):
  Factor by which to multiply \verb/a_fineFlux/ in flux register.
  \item
  \verb/a_finePatchIndex/ (not modified):
  Index which corresponds to which box in the fine-level
  \verb/DisjointBoxLayout/ (\verb/a_dbl/ in the \verb/define/
  function) over which \verb/a_fineFlux/ was calculated.
  \item
  \verb/a_srcInterval/ (not modified):
  The \verb/Interval/ of components to put into the flux register.
  \item
  \verb/a_dstInterval/ (not modified):
  The \verb/Interval/ of components of the flux register which are
  incremented by the flux data.
  \item
  \verb/a_dir/ (not modified):
  Direction of faces upon which fluxes live.
  \item
  \verb/a_sd/ (not modified):
  Side of the fine face where coarse-fine interface lies.
  \end{itemize}

\item
\begin{verbatim}
void reflux(LevelData<FArrayBox>& a_uCoarse,
            const Interval& a_coarse_interval,
            const Interval& a_flux_interval,
            Real a_scale);
\end{verbatim}
Increments \verb/a_uCoarse/ with the reflux divergence of the contents of
the flux register, scaled by \verb/a_scale/ ($\alpha$):
$U^c := U^c + \alpha D_R(\delta \vec{F})$. 
\verb/a_flux_interval/ gives the \verb/Interval/ of components of
the flux register that correspond to \verb/a_coarse_interval/ of components
of \verb/a_uCoarse/.
\\ {\bf Arguments:} 
  \begin{itemize}
  \item
  \verb/a_uCoarse/ (modified):
  \verb/LevelData<FArrayBox>/ which is modified by the refluxing operation.
  \item
  \verb/a_coarse_interval/ (not modified):
  The \verb/Interval/ of components to put into \verb/a_uCoarse/.
  \item
  \verb/a_flux_interval/ (not modified):
  The \verb/Interval/ of components to use from the flux register.
  \item
  \verb/a_scale/ (not modified):
  Factor by which to scale the flux register.
  \end{itemize}

\end{itemize}


\subsection{Class {\tt LevelFluxRegisterEdge}}
In Magnetohydrodynamics (among other fields), there is a class of
numerical schemes commonly known as ``constrained transport'' schemes
in which the solenoidal property of a field is maintained by making use of the
identity $div(curl(u) ) == 0$. In MHD, for example, one can write the
evolution of the magnetic field $\vec{B}$ in terms of the curl of the
electric field $\vec{E}$, which ensures that the $div(\vec{B}) = 0$
down to roundoff. This is generally accomplished on a staggered mesh,
in which $\vec{B}$ is face-centered and $\vec{E}$ is edge-centered.

At coarse-fine interfaces, one must perform a correction analogous to
the reflux-divergence operation in order to maintain a divergence-free
magnetic field. This correction is described, for example, in
\cite{Balsara:2001}. The {\tt LevelFluxRegisterEdge} is a class
designed to manage and apply this correction, and is analogous to the
{\tt LevelFluxRegister}. While the {\tt LevelFluxRegister} corrects a
cell-centered field with a ``reflux-divergence'' of face-centered
fluxes, the {\tt LevelFluxRegisterEdge} corrects a face-centered field
with a ``reflux curl'' of edge-centered ``fluxes''. 

\begin{itemize}
\item
\begin{verbatim}
void define(const DisjointBoxLayout& a_dbl,
            const DisjointBoxLayout& a_dblCoarse,
            const ProblemDomain& a_dProblem,
            int a_nRefine,
            int a_nComp)
\end{verbatim}
Defines this object


\item
\begin{verbatim}
  void setToZero()
\end{verbatim}
Sets all registers to zero.

\item
\begin{verbatim}
void incrementCoarse(FArrayBox& a_coarseFlux,
                     Real a_scale,
                     const DataIndex& a_coarseDataIndex,
                     const Interval& a_srcInterval,
                     const Interval& a_dstInterval)
\end{verbatim}
increments the register with data from coarseFlux, multiplied by
\verb/a_scale/. \verb/a_coarseFlux/ must contain the edge-centered (in 3d, node
centered in 2d) coarse fluxes in the dir direction for the grid        
\verb/m_coarseLayout[coarseDataIndex]/.
By convention, only the low side flux is used to avoid double-counting
at coarse-fine interfaces. 



\item
\begin{verbatim}
void incrementFine(FArrayBox& a_fineFlux,
                   Real a_scale,
                   const DataIndex& a_fineDataIndex,
                   const Interval& a_srcInterval,
                   const Interval& a_dstInterval)
\end{verbatim}
increments the register with data from fineFlux (which is edge-centered
in 3d, node-centered in 2d), multiplied by \verb/a_scale/, for all coarse-fine
face directions associated with the grid box \verb/m_fineLayout[fineDataIndex]/


\item
\begin{verbatim}
void incrementFine(FArrayBox& a_fineFlux,
                   Real a_scale,
                   const DataIndex& a_fineDataIndex,
                   const Interval& a_srcInterval,
                   const Interval& a_dstInterval,
                   int a_dir,
                   Side::LoHiSide a_sd)
\end{verbatim}
increments the register with data from fineFlux (which is edge-centered
in 3d, node-centered in 2d), multiplied by \verb/a_scale/.
\verb/a_dir/ is the normal of the coarse-fine interface, and
\verb/a_sd/ determines whether we're looking at the high-side or the
low-side for the grid box \verb/m_fineLayout[fineDataIndex]/


\item
\begin{verbatim}
void refluxCurl(LevelData<FluxBox>& a_uCoarse,
                Real a_scale)
\end{verbatim}
increments uCoarse with the reflux "CURL" of the
contents of the flux register. This is done for all components so
\verb/a_uCoarse/ has to have the same number of components as input
\verb/a_nComp/. This operation is global and blocking.

\end{itemize}
