\newcommand{\sfrac}[2]{\mbox{$\frac{#1}{#2}$}}

\subsection{Class {\tt FineInterp}}

This class fills the valid region of a level of data by piecewise
linear interpolation from data on a coarser level of refinement, using
the piecewise linear interpolation operator described in section
\ref{sec:ilt}.

\begin{itemize}
\item
\begin{verbatim}
void 
define(const DisjointBoxLayout& a_fine_domain,
       int a_numcomps,                        
       int a_ref_ratio,                       
       const ProblemDomain& a_fine_problem_domain);      

void
define(const DisjointBoxLayout& a_fine_domain,
       int a_numcomps, 
       int a_ref_ratio,
       const Box& a_fine_problem_domain)
\end{verbatim}
Arguments:
\begin{itemize}
\item
\verb|a_fine_domain| (not modified): grids (valid region) on the fine level.
\item
\verb|a_numcomps| (not modified): number of components of the coarse
and fine data.
\item
\verb|a_ref_ratio| (not modified): the refinement ratio~$N_r = \Delta
x^c / \Delta x^f$.
\item
\verb|a_fine_problem_domain| (not modified): the problem domain in the
fine level index space.
\end{itemize}

\item
\begin{verbatim}
void 
interpToFine(LevelData<FArrayBox>& a_fine_data,
             const LevelData<FArrayBox>& a_coarse_data,
             bool a_averageFromDest=false););
\end{verbatim}
Replaces fine data by interpolation from coarse data.
Arguments:
\begin{itemize}
\item
\verb|a_fine_data| (modified): the fine data set, destination of
interpolation. 
\item
\verb|a_coarse_data| (not modified): the coarse data set, source of
interpolation. 
\item
\verb|a_averageFromDest|: if true, first fill a projection of the fine grid with averaged values from {\tt a\_fine\_data} before filling with coarse data and performing interpolation. This is often useful in operations like flattening an AMR hierarchy to a single resolution, in which the fine data may not be properly nested within the coarse data grids. Default is {\tt false}.
\end{itemize}

\item
\begin{verbatim}
void 
pwcinterpToFine(LevelData<FArrayBox>& a_fine_data,
                const LevelData<FArrayBox>& a_coarse_data,
                bool a_averageFromDest=false););
\end{verbatim}
Replaces fine data by piecewise-constant interpolation from coarse data.
Arguments:
\begin{itemize}
\item
\verb|a_fine_data| (modified): the fine data set, destination of
interpolation. 
\item
\verb|a_coarse_data| (not modified): the coarse data set, source of
interpolation. 
\item
\verb|a_averageFromDest|: if true, first fill a projection of the fine grid with averaged values from {\tt a\_fine\_data} before filling with coarse data and performing interpolation. This is often useful in operations like flattening an AMR hierarchy to a single resolution, in which the fine data may not be properly nested within the coarse data grids. Default is {\tt false}.
\end{itemize}
\end{itemize}


\subsection{Class {\tt FineInterpFace}}

This class fills face-centered data in the valid region of a level of
data by piecewise linear interpolation from face-centered data on a
coarser level of refinement. This
  interpolation is performed in two steps:
\begin{enumerate}

\item Data on fine-level faces which overlie coarse-level faces is
  interpolated using only the underlying co-planar coarse faces.
\item Data on fine-level faces which do not overlie coarse-level faces
  is computed using linear interpolation in the normal direction
  between the two nearest fine-level faces which overlie coarse-level
  faces (and were filled in the previous step)
\end{enumerate}


\begin{itemize}
\item
\begin{verbatim}
void  define(const DisjointBoxLayout& a_fine_domain, 
             const int& a_numcomps,                   
             const int& a_ref_ratio,                  
             const ProblemDomain& a_fine_problem_domain)
\end{verbatim}
{\bf Arguments: }
\begin{itemize}
\item
\verb/a_fine_domain/: the fine level domain
\item
\verb/a_numcomps/: the number of components
\item
\verb/a_ref_ratio/: the refinement ratio
\item
\verb/a_fine_problem_domain/: problem domain
\end{itemize}
             
\item
\begin{verbatim}
void define(const DisjointBoxLayout& a_fine_domain, // the fine level domain
         const int& a_numcomps,                   // the number of components
         const int& a_ref_ratio,                  // the refinement ratio
         const ProblemDomain& a_fine_problem_domain);      // problem
domain
\end{verbatim}
{\bf Arguments:}
\begin{itemize}
\item
\verb/a_fine_domain/: the fine level domain
\item
\verb/a_numcomps/: the number of components
\item
\verb/a_ref_ratio/: the refinement ratio
\item
\verb/a_fine_problem_domain/: problem domain on finest level
\end{itemize}

\item
\begin{verbatim}
void interpToFine(LevelData<FluxBox>& a_fine_data,
               const LevelData<FluxBox>& a_coarse_data);
               bool a_averageFromDest=false);
\end{verbatim}

\end{itemize}
