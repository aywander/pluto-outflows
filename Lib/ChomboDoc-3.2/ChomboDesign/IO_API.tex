
\section{HDF5 I/O}

We have developed a user interface for file I/O based on version 5 of
the Hierarchical Data Format library (HDF5) developed at 
The National Center for Supercomputing Applications (NCSA).  HDF5
provides efficient and flexible mechanisms for handling I/O of large
scientific datasets, and is becoming a standard in the
scientific community for binary portable data files. We exploit a number of features provided by 
HDF5, including the portability of data across platforms and the
ability to read and write files on distributed memory parallel systems.
HDF5 also has a number of useful utilities, such as {\tt h5dump}, which
produces a human-readable formatted ASCII output of an HDF5 file. 

HDF5 has three main user abstractions: {\it group, dataset, attribute}.
Group abstracts the notion of the location in a file, while dataset
and attribute are different types of data that can be stored in an HDF5
file. We provide an API for creating HDF5 files and for reading from and
writing into such files. These are implemented using two classes, plus a
collection of stand-alone functions. 

\subsection{Class {\tt HDF5Handle}}

HDF5Handle is a class that manages the accessing of and navigation
within an HDF5 file.

\begin{trivlist} 
\item $\bullet$
\underline{Constructors.}
\begin{verbatim}
HDF5Handle();
HDF5Handle(const std:string& a_filename, mode);
int open(const std:string& a_filename, mode);
bool isOpen();
void close();
enum mode {CREATE,OPEN_RDONLY,OPEN_RDWR}
\end{verbatim}

A {\tt HDF5Handle} requires a {\tt a\_filename} supplied either 
at construction using the second constructor, or by a call to {\tt open}. 
{\tt filename} follows the semantics of \verb+fopen+ from the 
\verb+<stdio.h>+ of libc.
It is an error if a file has already been
opened by the {\tt HDF5Handle}.
It is also illegal to open a single file using two different 
{\tt HDF5Handle}s. The enumeration class {\tt mode} specifies the access
permissions. If {\tt mode} = {\tt CREATE}, the a file is created,
deleting the previously existing copy of that file if necessary. 
If {\tt mode} = {\tt OPEN\_RDONLY}, an existing file is opened with read-only
access. If {\tt mode} = {\tt OPEN\_RDWR} an existing file is opened with
read-write access. In the latter two cases, if the file doesn't exist,
then the open operation fails: 
there is no file bound to the {\tt HDF5Handle}, and a call to {\tt
isOpen} would return false. 
HDF5Handle objects must be explicitly closed by the user, just
like file pointers in standard C.  This is done with the \verb/close/
function.  You can inquire whether a handle is open or closed
with the \verb/isOpen()/ function.
Once {\tt close} has been called, it is possible open a new file with
the same {\tt HDF5Handle} using {\tt open}.

\item $\bullet$
\underline{File Navigation.}
\begin{verbatim}
int HDF5Handle::setGroup(const std::string& a_groupAbsPath);
const std::string& HDF5Handle::getGroup() const;
\end{verbatim}
The function \verb/setGroup/ sets the group (i.e., the location in the
file) to be that labeled with the string {\tt a\_groupAbsPath}. If such a
group does not yet exist, {\tt setGroup} creates such a group.
The function \verb/getGroup/ returns the string corresponding to the
group to which the {\tt HDF5Handle} is currently set.
The input and output strings to which the groups are set are assumed to
be of the form of an Unix absolute directory path, e.g., 
{\tt "/foo"}, {\tt "/level\_1/info"}, etc. There is a distinguished root
group {\tt "/"} to which the {\tt HDF5Handle} is initialized when a file is
opened.
\verb/setGroup/ can be thought
of as analogous to a Unix ``cd'' command.  \verb/getGroup/ can be
thought of as analogous to the Unix ``pwd'' command.
We should emphasize that the setting of the group in an HDF5Handle is
usually unrelated to the actual physical file layout. 
It just represents an evocative and
convenient notation for navigating within an HDF5 file.
%\centerline{\epsffile{../BoxTools/handle.eps}}
\verb/setGroup/ returns 0 on success, a negative number if HDF5
had an error.  If the group doesn't already exist, then it is
created if the file is write-enabled ({\tt CREATE} or {\tt OPEN\_RDWR}).  
In the event of error, file remains open and setGroup can be
called again, but HDF5Handle object is not capable of processing
reads or writes until a successful \verb/setGroup/ has been performed.
(except immediately after file opening when the root group is
valid for writing).

\end{trivlist}

\subsection{Class {\tt HDF5HeaderData}}

The class 
{\tt HDF5HeaderData} provides an interface for writing collections
of reals, integers, strings, {\tt Box}es and {\tt IntVectSet}s. 
In this interface, one must associate a name (in the form of a
character string) for each object. The internal treatment of this data
assumes that these are small collections of 
``metadata'', where the efficiency of storage is not a serious concern.
\begin{verbatim}
class HDF5HeaderData
{
public:
  int writeToFile(HDF5Handle& a_handle) const;
  int readFromFile(HDF5Handle& a_handle);
  void clear();

  map<std::string, Real>        m_real;
  map<std::string, int>         m_int;
  map<std::string, std::string> m_string;
  map<std::string, IntVect>     m_intvect;
  map<std::string, RealVect>    m_realvect;
  map<std::string, Box>         m_box;

};

\end{verbatim}
Once an {\tt HDF5HeaderData} object is created, the user adds objects to
to be stored by adding values to the STL {\tt map}s that are contained as
member data. For example,
\begin{verbatim}
HDF5HeaderData metaData;
metaData.m_real["mesh spacing"] = dx;
\end{verbatim}
If there is already a value in the map corresponding to the string 
"{\tt mesh spacing}", the value is overwritten. 
One queries to see if an if an attribute is entered in one of the maps
as follows.
\begin{verbatim}
bool ghost_exists = 
  (metaData.m_intvect.find("ghost") != metaData.m_intvect.end());
\end{verbatim}
Finally, one deletes an attribute from a {\tt map} as follows:
\begin{verbatim}
metaData.m_real.erase("mesh spacing");
\end{verbatim}

Once the user finishes filling in an {\tt HDF5HeaderData} object, the 
member functions are 
\verb/writeToFile/ and \verb/readFromFile/ write and read group attributes
from the group currently pointed to by {\tt a\_handle}.


\subsection{HDF5 I/O for {\tt BoxLayoutData}}

We provide a set of function interfaces for writing out 
data defined on unions of rectangles. There are two sets of functions: 
one for reading and writing the unions of rectangles, the second 
for reading and writing {\tt BoxLayoutData} objects. 

\begin{trivlist}
\item $\bullet$
\underline{BoxLayout I/O.}
\begin{verbatim}
int write(HDF5Handle& a_handle, const BoxLayout& a_layout);
int read(HDF5Handle& a_handle, Vector<Box>& boxes);
\end{verbatim}
The {\tt write} function writes out the union of boxes corresponding to
all of the {\tt BoxLayoutData} objects to be written to that
group. Consequently, one can only write one {\tt BoxLayout} object for
that group. 
The read function is not symmetric with the write function. The reason
for this is that processor assignment is not written out to the file
with the BoxLayout.  The file is considered {\it parallel neutral}. Since
a BoxLayout is a combination of Boxes {\it and} processor assignment, the
read function does not have enough information to build a BoxLayout.
It is the users responsibility to invoke the appropriate load balancing
function after the boxes have been read in, and build a BoxLayout object
\item $\bullet$
\underline{BoxLayoutData I/O.}

\begin{verbatim}
template <class T>
int write(HDF5Handle& a_handle, 
          const LevelData<T>& a_data, 
          const std::string& a_name);

template <class T>
int read(HDF5Handle& a_handle, 
         LevelData<T>& a_data, 
         const std::string& a_name, 
         const DisjointBoxLayout& a_layout, 
         bool redefineData = true);
\end{verbatim}
{\tt write} writes the collection of {\tt T} objects in {\tt a\_data}
into an HDF5 dataset, linearizing each object into a 
into a set of 1D Arrays. The default implementation is to use the 
linearization function {\tt T::linearOut} that is required to define the
{\tt LevelData<T>} class, but that will output data in terms of bytes,
and in general will not be portable across platforms. The user may
provide a more detailed linearization interface, 
in which case the HDF5 files can be made portable across platforms. 
Such an interface has been provided in Chombo for the case of 
{\tt T} = {\tt FArrayBox}. {\tt read} reads a {\tt LevelData<T>} object
that had been previously written by the {\tt write} function. 
On input, {\tt a\_layout} is a null-constructed {\tt LevelData}, which
is then defined inside {\tt read}. If {\tt redfineData == false}, then
the user takes responsibility for calling {\tt define} is the correct
manner for the \verb+LevelData<T> a_data+ argument.
The {\tt a\_layout} argument must consist of the same collection of 
{\tt Box}es as that used to define {\tt a\_data}, but may have a
different mapping of boxes to processors than that of the data as it was
written.

There are also versions of these functions for the case of 
{\tt BoxLayoutData<T>}.
\end{trivlist}

\subsection{HDF5 Out-Of-Core readers}

Frequently, a user will generate a data file from a simulation
run (particularly in parallel) that will exceed a single computers
available RAM.  This can make auxiliary post-processing programs
impossible to run unless they too are programmed in parallel. The
nature of many post-processing operation, however, are more
like data-mining than a full simulations.  For those cases a
simpler approach can be used where only a subset of the 
data file is read in and operated on.

\begin{verbatim}
template <class T>
int readLevel(
         HDF5Handle& a_handle,  
         const int& a_level, 
         LevelData<T>& a_data, 
         Real& a_dx, 
         Real& a_dt, 
         Real& a_time, 
         Box& a_domain,
         int& a_refRatio, 
         const Interval& a_comps = Interval(), 
         const IntVect& ghost = IntVect::Zero, 
         bool setGhost = false);

int readBoxes(
         HDF5Handle& a_handle,
	     Vector<Vector<Box> >& boxes);
 


int readFArrayBox(
         HDF5Handle& a_handle,
	     FArrayBox&  a_fab,
	     int a_level,
	     int a_boxNumber,
	     const Interval& a_components,
	     const std::string& a_dataName = "data" );

\end{verbatim}

The first function defaults to reading the entire range of components for a given
level of data.  If the user specifies {\tt Interval a\_comps} (currently defaulting
to the entire data range) then a user can select out just a particular range of
components.

{\tt readFArrayBox} is even more specific, in that it reads individual FArrayBox data's
from the file by \[level,index\] reference.
