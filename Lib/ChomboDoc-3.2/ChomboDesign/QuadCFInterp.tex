\subsection{Class {\tt QuadCFInterp}}

The class {\tt QuadCFInterp} interpolates data onto
the ghost cells along the coarse-fine interface of a
\verb/LevelData<FArrayBox>/, using the algorithm described in section
\ref{sec:quadB}.  It uses one-sided  differencing in places where the
stencil to do full centered differencing is partially covered by
finer grids. The user interface of {\tt QuadCFInterp} is given as
follows. 

\begin{itemize}

\item
\begin{verbatim} 
void define(const DisjointBoxLayout& a_fineBoxes,
            const DisjointBoxLayout* a_coarBoxes,
            Real a_dx, 
            int a_refRatio,
            int a_nComp,
            const ProblemDomain& a_domf);

void define(const DisjointBoxLayout& a_fineBoxes,
            const DisjointBoxLayout* a_coarBoxes,
            Real a_dx, 
            int a_refRatio,
            int a_nComp,
            const Box& a_domf);
\end{verbatim}
Full define function. This makes all coarse-fine information and sets
internal variables.
\\ {\bf Arguments:} 
  \begin{itemize}
  \item
  \verb/a_fineBoxes/ (not modified):
  The grids at the current level.
  \item
  \verb/a_coarBoxes/ (not modified):
  The grids for the next coarser level in the AMR hierarchy.
  \item
  \verb/a_dx/ (not modified):
  The grid spacing at the current level.
  \item
  \verb/a_refRatio/ (not modified):
  The refinement ratio between this level and the next coarser level in the
  AMR hierarchy.
  \item
  \verb/a_nComp/ (not modified):
  The number of components in the data to be interpolated.
  \item \verb/a_domf/ (not modified):
  The problem domain at the fine level.
  \end{itemize}

\item 
\begin{verbatim} 
void coarseFineInterp(LevelData<FArrayBox>& a_phif,
                      const LevelData<FArrayBox>& a_phic) const;
\end{verbatim}
Coarse-fine interpolation operator.  Fills all the ghost cells on 
all the faces of the \verb/LevelData<FArrayBox> a_phif/
with values interpolated with \verb/a_phic/.
\\ {\bf Arguments:} 
  \begin{itemize}
  \item
  \verb/a_phif/ (modified):
  The solution at the current level.
  \item
  \verb/a_phic/ (not modified):
  The solution at the next coarser level in the AMR hierarchy.
  \end{itemize}

\end{itemize}
