\section{Data on Unions of Rectangles}
\label{MultiBoxSection}
This section describes our tools for doing calculations over
unions of rectangles.  These tools may be used either
in serial or in parallel though the design reflects
our parallel programming model.  
In this section, we explain the tools 
we use to describe unions of rectangles and the data which lives
over these regions.

\subsection{Introduction}

We wish to represent data defined on unions of rectangles.
Such data can be mapped naturally onto distributed memory
by assigning boxes to processors, with data defined on
those boxes stored on the processor to which the box is
assigned.  This approach has been used quite successfully.
Berger and Bokhari \cite{BergerBokhari1986},
Kohn and Baden 
\cite{KohnBaden1996}, 
Rendleman, et. al. \cite{CCSE1999}, 
and others have used this technique.
Our API is derived from joint work with Baden to develop an abstract
version of KeLP \cite{fink96:kelp}.
It is implemented using the following three sets
of classes:
\begin{trivlist}
\item $\bullet$ 
{\tt BoxLayout, DisjointBoxLayout}---classes that 
        represent unions of rectangles and the mapping of those 
        rectangles to processors.
\item $\bullet$ 
{\tt LayoutData, BoxLayoutData, LevelData}---
        templated classes for distributing data over
        processors.
\item $\bullet$ 
{\tt LayoutIterator/LayoutIndex, DataIterator/DataIndex}---
        classes for iterating over and indexing into the classes
        above.
\end{trivlist}

\subsection{Layouts}
\label{LayoutSection}
The classes {\tt BoxLayout} and {\tt DisjointBoxLayout} represent
unions of rectangles and the mapping of the rectangles
onto processors.  {\tt BoxLayout}  represents an arbitrary
union of valid boxes. {\tt DisjointBoxLayout} is a
{\tt BoxLayout} and has the additional 
property that none of the boxes intersect.  
Both types of layout have two states:  open and
closed.  During construction, a layout is open.  In its
open state, a user can add boxes and modify the mapping 
of boxes to processors.  When a user is finished changing
a {\tt BoxLayout} to her satisfaction, she invokes 
the {\tt close()} function.
After closing, the {\tt BoxLayout} cannot be accessed
in a non-const manner.  There is no way to 
reopen a closed {\tt BoxLayout}.
The closed property propagates through assignment and copy construction.
Only closed layouts may be used to build the distributed
data classes.

\subsubsection{Class BoxLayout}
\label{BoxLayoutSection}
\begin{figure}[htp]
\centerline{
\epsfxsize=2.0in
\epsffile{figs/boxlayout2.eps}
\hspace{1.in} 
\epsfxsize=2.0in
\epsffile{figs/boxlayout1.eps}}
\caption{Left:  Example of a {\tt{BoxLayout}}.  The first integer in the pair
identifies the {\tt{Box}}, and the second integer the processor ID.
In this case we have the following assignments.  Processor 0:  $B_1,
B_5$.  Processor 1:  $B_0, B_2$.  Processor 2:  $B_3, B_4$.  Note that
 $B_0$ and $B_1$ have a non-empty intersection.  Right:  Example of a {\tt{disjoint BoxLayout}}.  The first integer in the pair
identifies the {\tt{Box}}, and the second integer the processor ID.
In this case we have the following assignments.  Processor 0:  $B_0,
B_2, B_4, B_5$.  Processor 1:  $B_1$.  Processor 2:  $B_3$. Note that a disjoint {\tt{BoxLayout}} has empty intersections.}  
\label{fig:BoxLayouts}
\end{figure}

A {\tt BoxLayout} is a collection of boxes. On parallel platforms,
{\tt BoxLayout} includes a mapping to processors. In both cases,
the data holders {\tt LayoutData}, {\tt BoxLayoutData}, and {\tt
LevelData} define mappings from the {\tt Box}es in the {\tt BoxLayout} 
to objects of the template type {\tt T}. 
The important functions of {\tt BoxLayout} are as follows:
\begin{trivlist}
\item $\bullet$ \underline{Construction}
\begin{verbatim} 
BoxLayout(const Vector<Box>& boxes, const Vector<int>& procIDs)
void define(const Vector<Box>& boxes, const Vector<int>& procIDs)
virtual void deepCopy(const BoxLayout& source)
DataIndex addBox(const Box& box, int iProc)
virtual void close()
\end{verbatim}
The constructor and define functions 
construct a {\tt BoxLayout} from a vector of Boxes and a vector of
processor assignments. The input {\tt procIDs}
must all be in the range \verb/[0...numProcs()-1]/ 
where the function {\tt numProcs()}, located in 
{\tt SPMD.H}, returns the number of processors being used in
the calculation.
\verb/procIDs[i]/ is the processor number of 
the processor on which the 
data that maps to the box \verb/boxes[i]/ is stored.
The processor assignment Vector
must be the same length
as the {\tt Vector<Box>} argument.
On exit, the {\tt BoxLayout} will be closed.
One can either null construct the {\tt BoxLayout} and call
the define function or construct and define at once.
If the user is not using MPI, the {\tt procIDs} argument
is ignored.
The new object created with {\tt deepCopy} 
disassociates itself with original implementation
safely.  This object now is considered 'open' and can be non-const
modified.  There is no assurance that the order in which this 
{\tt BoxLayout}
is indexed corresponds to the indexing of {\tt source}.
{\tt addBox} incrementally adds a box and its processor assignment to
an open layout (if the layout has been closed, calling
this function generates a run-time error) and returns a {\tt DataIndex}
object.
The {\tt DataIndex} object is valid both before and after
{\tt close} is called.
It can
be used later to access this box again, or access the 
data object ({\tt T})
in a \verb/BoxLayoutData/ that is built from this 
{\tt BoxLayout} object. 
{\tt close} marks this {\tt BoxLayout} as complete and unchangeable.   
It is here that the layout
gets sorted.  This must be called before any data containers
are constructed using the layout.

\item $\bullet$ \underline{Boolean functions.}
\begin{verbatim} 
bool operator==(const BoxLayout& rhs) const 
bool check(const DataIndex& index) const
bool isClosed()
\end{verbatim}
Equality for {\tt BoxLayout} is a reference-counted  pointer
check.  This returns true if these two objects
share the same implementation. 
Important Warning: 
Two layouts can have the same boxes and same
processor mapping and still return false if they were built
separately.  To force equality of two layouts, use 
the copy constructor.
{\tt check} returns true if the input {\tt DataIndex} matches
the layout. 
{\tt isClosed} returns true if close() has been called.

\item
$\bullet$ \underline{Accessors.}
\begin{verbatim} 
Box& operator[](const DataIndex& it)
DataIterator dataIterator() const
LayoutIterator layoutIterator() const
\end{verbatim}
This allows access to an individual
box through the iterator.  
One must be iterating through the 
correct layout ({\tt check} must return true) in order 
for the accessor operator to work correctly.
The member functions {\tt dataIterator}, {\tt layoutIterator}
return the iterators associated with this layout.

\item $\bullet$ \underline{Coarsening and Refinement Operations}.
\begin{verbatim} 
friend void coarsen(BoxLayout& output, const BoxLayout& input,
                    int refinement)
friend void refine(BoxLayout& output, const BoxLayout& input,
                   int refinement)
\end{verbatim}
The functions {\tt coarsen}, {\tt refine} coarsens or refines each 
box in the layout by the input refinement ratio.
Iterator objects that worked for the input will work for
the output.  
\end{trivlist}
\subsubsection{Class DisjointBoxLayout}
\label{DisjointBoxLayoutSection}


{\tt DisjointBoxLayout} is-a {\tt BoxLayout}. The difference between
them is that, for {\tt DisjointBoxLayout}, closed also implies that
the boxes in a {\tt DisjointBoxLayout} have no non-trivial
intersection with one another in index space.  Any attempt to close a
{\tt DisjointBoxLayout} object with boxes which have non-trivial
intersection will result in a run-time error. Coarsening may not
preserve disjointedness, and applying the {\tt coarsen} operator to a
{\tt DisjointBoxLayout} will generate also a run-time error if the new
coarsened boxes aren't disjoint.  Otherwise, all of the functions of
{\tt BoxLayout} carry over to {\tt DisjointBoxLayout}. 

If the problem domain is periodic, disjointness is tied to the
periodicity of the domain -- a box in the BoxLayout may intersect the
periodic image of another box.  To account for this, {\tt
DisjointBoxLayout} can also be defined with a {\tt ProblemDomain}.  In
the default case, the domain is defined to be non-periodic in all
directions.  If the domain is periodic, then the periodicity of the
domain is taken into account when checking for disjointness of the
boxes in the BoxLayout. 

The important extra functions of DisjointBoxLayout are as follows:
\begin{trivlist}
\item $\bullet$ \underline{Constructors}
\begin{verbatim}
DisjointBoxLayout(const Vector<Box>& boxes, 
                  const Vector<int>& procIDs,
                  const ProblemDomain& probDomain) 
void define(const Vector<Box>& boxes, const Vector<int>& procIDs,
            const ProblemDomain& probDomain)
void define(BoxLayout& a_layout, const ProblemDomain& a_physDomain);
virtual void deepCopy(const BoxLayout& a_source, 
                      const ProblemDomain& a_physDomain)
\end{verbatim}
These functions are the same as the corresponding functions in {\tt
BoxLayout}, but with the addition of a {\tt ProblemDomain} argument.
\item $\bullet$ \underline{Checking functions}
\begin{verbatim}
bool checkPeriodic(const ProblemDomain& probDomain)
\end{verbatim}
The {\tt checkPeriodic} function returns true if the argument
{\tt ProblemDomain} is consistent with the {\tt ProblemDomain} used to
define the {\tt DisjointBoxLayout}.  Two {\tt ProblemDomain}s are
consistent if they are periodic in the same directions, and they have
same domain size in any periodic directions.  In non-periodic
directions, no consistency is required.
\end{trivlist}
