\section{Introduction and Motivation}

Chombo applications typically accept user input via files containing key-value 
pairs. The mechanism for parsing these pairs into a C++ table is the 
\verb ParmParse , a singleton which retrieves basic C++ primitives upon 
request. \verb ParmParse  has long provided ANAG developers with a simple method of 
parsing input and is used by nearly all of the examples in the Chombo source 
tree.

However, \verb ParmParse  is limited by the fact that it only understands 
primitive C++ variables. As Chombo moves into the realm of applications, and 
as we improve our testing utilities, it makes sense for us to consider adding 
some richness to our input language. Embedded boundaries present an interesting 
example of a problem not currently addressed by \verb ParmParse : can we 
construct implicit functions during the input stage instead of writing a 
separate driver for every problem or related family of problems with 
embedded boundaries?

We are currently investigating ways of more easily developing and manipulating 
implicit functions using a set of primitives maintained in Chombo's C++ 
code base. One of the more promising explorations has been \verb PyWorkShop : 
a tool that allows us to create and manipulate C++ \verb BaseIF  objects by 
performing set operations in Python. This can be used to create an intermediate 
representation of Chombo's C++ implicit functions, expressed in Python, that 
can be easily translated into different formats for use in external tools. 
In addition, the presence of a Python interpreter makes it possible to run
programs interactively in an manner similar to Matlab.

The ease of embedding the Python parser into C++ applications such as 
\verb PyWorkShop  and \verb Scylla  has led us to consider developing a Python drop-in replacement
for \verb ParmParse  that we will call \verb PyParse . The Python parser is 
capable of a level of syntactic richness that far exceeds that of 
\verb ParmParse  while allowing us to retain the relatively simple structure of key-value 
inputs. For example, consider the following possibilities:

\begin{itemize}
\item Specifying an arbitrarily complicated implicit function in an input file 
  with no additional parsing code in the application
\item Specifying arbitrary initial conditions and/or source functions 
  with no additional parsing code in the application
\item Performing post-processing or convergence studies within input files 
  using control structures (loops, tests)
\item Interactively creating, visualizing, and manipulating implicit functions
  composed of a well-defined set of primitives.
\end{itemize}

\noindent
Each of these goals gets us closer to the goal of ``making the code look more 
like the math." Capabilities of this sort are already available in the 
\verb Scylla  capillary modeling code, a Chombo application with a Python 
front end.
  
The syntax of Python is already very similar to that of \verb ParmParse , 
so very few changes would have to be made to ``port" existing input files to 
\verb PyParse , and in any case, the similarity in the usage would allow 
developers and users to move to \verb PyParse  at their convenience. Moreover, 
if we encapsulate the interpreter in a class like \verb ParmParse , no one 
needs to learn to use the Python C API.

This does not require a great deal of effort. The purpose of this document is 
to make the group aware of the existence of a functional prototype, to discuss 
its features, gather comments about possible modifications, and to advocate 
its adoption.

\section{Supported types}

The most interesting feature of \verb PyParse  is that it can be made to 
understand richer data structures than can allow input scripts to be much 
more expressive. We need to decide what data types will be supported ``out of 
the box" by \verb PyParse . Currently, the following seem to make sense:

\begin{itemize}
\item C++ numeric primitives: \verb Real , \verb int .
\item Some basic C++ container classes: \verb string , \verb vector<int> , \verb vector<Real> .
\item Some Chombo ``primitives": \verb RealVect , \verb IntVect .
\item The \verb Box  class.
\item \verb ScalarFunction , \verb VectorFunction , \verb TensorFunction  types for 
      analytic functions (e.g. $f(\vec{x}), g(\vec{x}, t)$).
\item Implicit functions: \verb BaseIF .
\item \verb vector s of all the above types (as well as \verb vector s  of \verb vector s and so on).
\end{itemize}

\section{Software Interface}

As a drop-in replacement for \verb ParmParse , \verb PyParse  has an interface 
that is very similar--it is a singleton instance that needs to be initialized 
by a file and/or \verb argc , \verb argv . It must support methods for 
querying whether a given named parameter was parsed by \verb PyParse , and it 
must be able to retrieve the parameter itself. The proposed interface 
appears in the Doxygen-generated reference manual for the \verb PyParse  class,
which should be distributed with this document.

The deliverable code will consist of a header file, \verb PyParse.H , and 
an associated library or set of object files that implement the embedding of 
the Python interpreter and the representation of Chombo data types in Python.
The code associated with the \verb PyWorkShop  effort may be incorporated or 
included separately depending on how we see fit to use it.

\section{Usage}

Here is an example of an input file that initializes a linear advection 
problem in a domain cut by two planes $p_1$ and $p_2$. Notice the similarity 
in spirit to the traditional \verb ParmParse  syntax.

\begin{verbatim}
# Initial condition: Gaussian pulse centered at x0 
# Note the use of math functions!
x0 = (0.5, 0.5)
def F(x):
    return 1.0/sqrt(pi) * exp(-(x[0]-x0[0])**2 + (x[1]-x0[1])**2/2
phi0 = F

# Advection velocity--uniform, but can also be set to a function.
v = (1,-1)

# Initialize the embedded boundary.
p1 = Plane(normal = (1, 1), point = (0.5, 0.5))
p2 = Plane(normal = (1, -1), point = (0.5, 0.5))
boundary = p1 | p2 # Union of p1 and p2
\end{verbatim}

\noindent
The relevant parameters for the simulation are \verb phi0 , \verb v , and 
\verb boundary , which are all parsed and retrieved with the following C++ 
code:

\begin{verbatim}

#include "PyParse.H"
using namespace std;

int main(int argc, char* argv)
{
  ...

  // Parse the parameters given a file in command line arguments.
  PyParse params(argc, argv);

  // Get the initial condition.
  RefCountedPtr<ScalarFunction> phi0;
  params.get("phi0", phi0);

  // Get the advection velocity.
  RealVect v;
  params.get("velocity", v);

  // Get the boundary.
  RefCountedPtr<BaseIF> boundary;
  params.get("boundary", boundary);

  ...
}
\end{verbatim}

