\section{AMRTimeDependent  Example: Advection-Diffusion}

We provide an example which uses the AMRTimeDependent infrastructure
to solve the advection-diffusion equation.   Given an analytic
velocity field $\vec v$ and a diffusion coefficient $\nu$, we evolve
an advected and diffused scalar $\phi$ using the advection-diffusion
equation:  
$$
\frac{\partial \phi}{\partial t} + \nabla \cdot(\vec v \phi) = \nu
\Delta \phi .
$$ 
The algorithm is mostly as described in previous sections.  The
solution to the Riemann problem is given by simple upwinding.  $F_d$,
the advective flux in the $d$ direction, is given by $F_d = v^d \phi$.
The characteristic values are the velocity components.  

The main difference between the algorithm used in this example and
that in Figure \ref{fig:HSCLcode} is in the approach taken to
flux correction during the synchronization step. The total flux
is the advective flux minus the diffusive flux: $F_d^{tot} = v^d
\phi - \nu \nabla \phi$. Because the fluxes used to update the
solution include both advective and diffusive fluxes, simply
performing the explicit refluxing correction step:
$$\phi^l(t^l + \Delta t^l) := \phi^l(t^l + \Delta t^l) - \Delta t^l D_R
(\delta F^{l+1})$$ 
will be unstable because it represents a forward-Euler update for the
heat equation (which requires that $\Delta t \approx O(h^2)$ for
stability). To preserve stability, we instead apply the flux
correction implicitly, as described in
\cite{martinColellaGraves:2008}:  

We first solve a Helmholtz equation for a correction:
\begin{equation}
(I -  \Delta t^{\ell_{base}} \nu \Delta^{comp}) \delta \phi = \Delta
t^{\ell} D_R(\delta \vec{F}^{tot,\ell+1}) \ \ \ {\rm for } \ \ \ell \geq
\ell_{base} 
\end{equation}
Then, the correction is added to the solution: 
\begin{equation}
\phi^\ell := \phi^\ell + \delta \phi^\ell \ \ {\rm for } \ \ \ell \geq
\ell_{base}. 
\end{equation}

\subsection{AdvectionDiffusion-Specific Classes}

The AMR advection-diffusion application in
{\tt Chombo/example/AMRGodunov}
uses the AMRTimeDependent infrastructure
extensively.  These are the classes that are specific to this application.
\begin{itemize}
\item \verb|AdvectPhysics| is the \verb|GodunovPhysics|-derived class.   It provides
  \verb|PatchGodunov| the physics-specific information needed to advance the
  solution. 
\item 
\verb|LevelAdvect| is used to advance an advection-diffusion solution one
step:
$$
\phi^{new} = \phi^{old} - \dt (\nabla \cdot (\vec v \phi) )^{n+\half} .
$$
   In the case of non-zero diffusion, we use this advance as an
input to a semi-implicit advance using the \verb|LevelTGA| class
described in section \ref{section:TGA}.   In the pure advection
case, we can just use this update.  
\item \verb|AMRLevelAdvectDiffuse| is the \verb|AMRLevel|-derived class that drives
  the calculation.
\item  \verb|AMRLevelAdvectDiffuseFactory| is the factory class that the user
  sends to \verb|AMR|.  This class generates \verb|AMRLevelAdvectDiffuse| objects
  for \verb|AMR|.
\end{itemize}

\subsection{Usage Pattern}

In broad strokes, the driver for the \verb|AMRLevelAdvectDiffuse| example
looks something like this:
\begin{small}
\begin{verbatim}
  // read inputs
  ParmParse ppgodunov;

  //define the domain of computation
  ProblemDomain prob_domain;
  getProblemDomain(prob_domain);

  //define the initial and boundary conditions
  RefCountedPtr<AdvectTestIBC> ibc;
  getAdvectTestIBC(ibc);

  //define the AdvectPhysics object
  AdvectPhysics advPhys;
  advPhys.setPhysIBC(&(*ibc));

  //pick an advection velocity function
  AdvectionVelocityFunction velFunc;
  getAdvectionVelocityFunction(velFunc);

  //define the factory object
  RefCountedPtr<AMRLevelAdvectDiffuseFactory>  amrg_fact;
  getAMRLADFactory(amrg_fact, velFunc, advPhys);

  //define the AMR object
  AMR amr;
  defineAMR(amr, amrg_fact, prob_domain, a_refRat);
 
  //sets all the amr parameters (as opposed to a fixed grid run)
  setupAMRForAMRRun(amr);

  // run
  amr.run(a_stopTime,a_nstop);

  // output last pltfile and statistics
  amr.conclude();
\end{verbatim}
\end{small}
To use these functions  directly (all of these functions exist in
\verb|AdvectDiffuseUtils|), one must have an input file that looks something
like this:
\begin{small}
\begin{verbatim}
##cfl number and its initial time counterpart
cfl = 0.9
initial_cfl = 0.1
##nu
diffusion_coef = 0.001
##how much output one wants from the application
verbosity = 1
##max number of time steps
max_step = 100
##final solution time
max_time = 100.0
##number of buffer cells
tag_buffer_size = 3
##tagging threshold for undivided difference of solution
refine_thresh = 0.05
##how often we regrid
regrid_interval = 2 2 2 2 2 2
##how fast time step can grow and how far out of bounds it can get
max_dt_growth = 1.1
dt_tolerance_factor = 1.1
## size of the domain
domain_length = 1.0 
## distance (in level l cells) between levels l+1 and l-1
grid_buffer_size = 1
##do periodic boundary conditions or not
periodic_bc = 1 1 1 
##fixed time step size.  negative if not fixing time step
fixed_dt   = -1.0
##maximum AMR level number
max_level = 2
##coarsest AMRLevel grid size
n_cell =  64 64 64
##refinement ratios between levels
ref_ratio = 2 2 2 2 
##blocking factor
block_factor =  4
##maximum size of grids
max_grid_size = 64
##fill ratio for Berger-Rigoutsos
fill_ratio = 0.75
## how often to checkpoint.  negative if no checkpoints
checkpoint_interval = -1
## how often to dump plotfile.  negative if no plotfiles.
plot_interval = 10
##prefix of plotfiles
plot_prefix  = plt
##prefix of checkpoint files
chk_prefix = chk
blob_center = 0.75 0.5 0.5
blob_radius = 0.1
#0 = constant
#1 = rotating
advection_vel_type = 0
use_limiting = true
amrmultigrid.num_smooth  = 8
amrmultigrid.tolerance   = 1.0e-10
amrmultigrid.num_mg      = 1
amrmultigrid.norm_thresh = 1.0e-10
amrmultigrid.hang_eps    = 1.0e-10
amrmultigrid.max_iter    = 100
amrmultigrid.verbosity   = 1
##1 = neumann, 0 = dirichlet
bc_lo   = 1 1 1
bc_hi   = 1 1 1
bc_value = 0.
\end{verbatim}
\end{small}

\subsection{{\tt AMRLevelAdvectDiffuseFactory} Interface}

We use the following interface to create an \verb|AMRLevelAdvectDiffuseFactory|.
\begin{small}
\begin{verbatim}
  AMRLevelAdvectDiffuseFactory(const AdvectPhysics&        gphys,  
                               AdvectionVelocityFunction   advFunc,
                               BCHolder                    bcFunc, 
                               const Real&                 cfl,
                               const Real&                 domainLength,
                               const Real&                 refineThresh,
                               const int&                  tagBufferSize,
                               const Real&                 initialDtMultiplier,
                               const bool&                 useLimiting,
                               const Real&                 nu);

\end{verbatim}
\end{small}
\begin{itemize}
\item \verb|gphys| -- The \verb|AdvectPhysics| used to create the advection term.  
\item \verb|advFunc| -- The function that provides the advection velocity.
  This function must be of the form
\begin{small}
\begin{verbatim}
typedef Real (*AdvectionVelocityFunction)(const RealVect& point,
                                          const int&      velComp);
\end{verbatim}
\end{small}
\item \verb|bcFunc| -- The boundary condition class sent to solve the
  diffusion equation.   See section \ref{sec::AMRPBC} for details.
\item \verb|cfl| -- The CFL number.
\item \verb|domainLength| -- The physical length of the domain.
\item \verb|refineThresh| -- Undivided gradient size over which a cell will be
  tagged for refinement.
\item \verb|tagBufferSize| -- Number of buffer cells around each tagged point
  that will also be tagged.
\item \verb|initialDtMultiplier| -- CFL number at the beginning of the calculation.
\item \verb|useLimiting| -- Whether to use vanLeer limiting.  Turn this to
  \verb|true| unless you are doing a convergence test.
\item \verb|nu| -- Diffusion coefficient.
\end{itemize}

\subsection{{\tt LevelAdvect} Interface}
\verb|LevelAdvect| is used to advance an advection-diffusion solution one
step.   In the case of non-zero diffusion, we use this advance as an
input to a semi-implicit advance using \verb|LevelTGA|.   In the pure
advection case, we can just use this update.  If a user is simply
using the \verb|AMRLevelAdvectDiffuse| application, this class is entirely
internal to \verb|AMRLevelAdvectDiffuse|.   This discussion is provided for
the benefit for those who want to use this object for other
applications. 

We define a \verb|LevelAdvect| with the following interface
\begin{small}
\begin{verbatim}
  void define(const AdvectPhysics&        gphys,
              const DisjointBoxLayout&    thisDisjointBoxLayout,
              const DisjointBoxLayout&    coarserDisjointBoxLayout,
              const ProblemDomain&        domain,
              const int&                  refineCoarse,
              const bool&                 useLimiting,
              const Real&                 dx,
              const bool&                 hasCoarser,
              const bool&                 hasFiner);

\end{verbatim}
\end{small}
\begin{itemize}
\item \verb|gphys| -- The \verb|AdvectPhysics| used to create the advection term.  
\item \verb|domain| -- The domain of the computation.
\item \verb|refineCoarse| -- The refinement ratio to the next coarser level.
\item \verb|useLimiting| -- Whether to use vanLeer limiting.  Turn this to
  \verb|true| unless you are doing a convergence test.
\item \verb|dx| -- The grid spacing.
\item \verb|hasCoarser| -- Whether there is a coarser AMR level.
\item \verb|hasFiner| -- Whether there is a finer AMR level.
\end{itemize}

We present the interface function that advances the solution.
$$
\phi^{new} = \phi^{old} - \dt (\nabla \cdot (\vec v \phi) )^{n+\half}
$$
\begin{small}
\begin{verbatim}
  Real step(LevelData<FArrayBox>&         U,
            LevelFluxRegister&            finerFluxRegister,
            LevelFluxRegister&            coarserFluxRegister,
            LevelData<FluxBox>&           advectionVelocity,
            const LevelData<FArrayBox>&   S,
            const LevelData<FArrayBox>&   UCoarseOld,
            const Real&                   TCoarseOld,
            const LevelData<FArrayBox>&   UCoarseNew,
            const Real&                   TCoarseNew,
            const Real&                   time,
            const Real&                   dt);
\end{verbatim}
\end{small}
\begin{itemize}
\item \verb|U| -- The solution.   Input is the old solution $\phi^{old}$.  Input is the new solution $\phi^{new}$.
\item \verb|finerFluxRegister| -- The flux register with the next finer level.
\item \verb|coarserFluxRegister| -- The flux register with the next coarser level.
\item \verb|advectionVelocity| -- The holder for the advection velocity.
  This is held in a \verb|LevelData| to make this object of somewhat more 
  general utility.
\item \verb|S| -- Source term.  In the case of advection-diffusion, $S= \nu \Delta \phi$.
\item \verb|UCoarseOld| -- Solution at next coarser level at the old time.   
\item \verb|TCoarseOld| -- Time at which \verb|UCoarseOld| exists.
\item \verb|UCoarseNew| -- Solution at next coarser level at the new time.   
\item \verb|TCoarseNew| -- Time at which \verb|UCoarseNew| exists.
\item \verb|time| -- Time at which U exists before advance happens ($t^{old}$).
\item \verb|dt|  -- Time step.
\end{itemize}
