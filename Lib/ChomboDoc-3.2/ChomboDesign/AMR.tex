\section{Classes {\tt AMR} and {\tt AMRLevel}}

The class \verb|AMR| implements a framework for the Berger--Oliger
adaptive mesh refinement (AMR) algorithm for time-dependent
simulations~\cite{bergerOliger:1984}.  The data is organized on a
hierarchy of levels of refinement, stored as a collection of
\verb|AMRLevel|s.  The class \verb|AMRLevel| is an abstract base class
from which must be derived a concrete class which defines and contains
the data representation for one level and implements the algorithms
for advancing one level in time.  The class \verb|AMR| implements
refinement of the time step~$\Delta t$ based on the refinement of the
grid.

\subsection{Class structure}

The class \verb|AMR| manages the entire hierarchy of levels.  The
levels are represented in the class \verb|AMR| as a collection of
\verb|AMRLevel|s.  The class \verb|AMRLevel| is an abstract base class
from which the applications implementer must derive a concrete {\em
physics class\/} which defines the form of the solution data and,
for a single level, implements algorithms for advancement by~$\Delta
t$, calculation of a stable~$\Delta t$, initialization, input and
output, etc.  The class \verb|AMR| has no knowledge of the form of the
solution data.  The applications implementer must instantiate an
object of type \verb|AMR| and an object of the physics class type.
The physics class object is required input to the definition of an
\verb|AMR|.  See figure~\ref{fig:classes}.

\begin{figure}
\vspace{2.5in}
\special{psfile=figs/classes.eps hscale=50 vscale=50}
\caption{Class structure.  {\tt AMR} contains some
{\tt AMRLevel}s.  Physics class is derived from
{\tt AMRLevel} and contains the data representation.}
\label{fig:classes}
\end{figure}

Applications code constructs and invokes public member functions of
the class \verb|AMR|.  The class \verb|AMR| controls the entire
hierarchy of levels.  It constructs and invokes member functions of
the \verb|AMRLevel|s.  It requires a physics class derived from
\verb|AMRLevelFactory| as input.

\subsection{Class {\tt AMR}}

   The {\tt AMR} class is a framework for Berger--Oliger timestepping for
   adaptive mesh refinement of time-dependent problems.  It is
   applicable to both hyperbolic and parabolic problems.  It
   represents a hierarchy of levels of refinement as a collection of
   {\tt AMRLevel}s.  
   The usage pattern of this class is as follows: 
\begin{itemize}
\item Call \verb/define/ to define the parameters that do not change
      throughout the run (max level, refinement ratios, domain, and operator).
\item Modify any parameters you like (blocking factor and so forth)
      using access functions.
\item Call any one of the three setup functions so \verb|AMR| can set up all its
      internal data structures.
\item Call \verb/run/ to run the calculation.
\item Call \verb/conclude/ to produce statistical output, e.g., how many
      cells were updated. 
\end{itemize}
The important functions of the public interface for the \verb|AMR| class are:

\begin{itemize}

\item
\begin{verbatim}
void define(int a_max_level, 
            const Vector<int>& a_ref_ratios,
            const ProblemDomain& a_prob_domain,
            const AMRLevelFactory* const a_amrLevelFact);

void define(int a_max_level, 
            const Vector<int>& a_ref_ratios,
            const Box& a_prob_domain,
            const AMRLevelFactory* const a_amrLevelFact);
\end{verbatim}
Defines this object.  User must call a setup function before running.
\\ {\bf Arguments:}
  \begin{itemize}
  \item
  \verb/a_max_level/ (not modified):
  The maximum level number allowed, where the base level is zero.  There
  may be a total of \verb/a_max_level/+1 levels, since level zero and level
  \verb/a_max_level/ can both exist.  Note that it while this is the maximum
  possible level, it is possible that fewer levels are actually defined,
  depending on the problem and the method and tolerances used to tag cells
  for refinement.
  \item
  \verb/ref_ratios/ (not modified):
  Refinement ratios.  There must be at least \verb/a_max_level/+1 elements
  or an error will result.  Element  zero is the base level.
  \item
  \verb/a_prob_domain/ (not modified):
  Problem domain on the base level.
  \item
  \verb/a_amrLevelFact/ (not modified):
  Pointer to a physics class factory object. The object it points to is
  used to construct the collection of \verb|AMRLevel|s in this \verb|AMR| as objects of
  the physics class type.  
  \end{itemize}

\item
\begin{verbatim}
void setupForRestart(HDF5Handle& a_handle);
\end{verbatim}
Sets up this object from checkpointed data.  User must have
previously called {\tt define}. A user needs to call either this
function or 
\verb|setupForNewAMRRun| or \verb|setupforFixedHierarchyRun|
before she calls \verb|run|.

\item
\begin{verbatim}
void setupForNewAMRRun();
\end{verbatim}
Sets up this object for cold start.  User must have previously called
{\tt define}.  Need to call this function or \verb|setupForRestart| or
\verb|setupforFixedHierarchyRun| before you run.

\item
\begin{verbatim}
void setupForFixedHierarchyRun(const Vector<Vector<Box>>& a_amr_grids,
                               int a_proper_nest = 1);
\end{verbatim}
This sets the grid hierarchy and sets \verb/regrid_intervals/ to -1
(turns off regridding).  If you want to keep regridding
on, reset regridIntervals after this call.

\item
\begin{verbatim}
void run(Real a_max_time, int a_max_step);
\end{verbatim}
Run the calculation.  User must have previously called both the \verb|define|
function and a setup function in order to call this.
\\ {\bf Arguments:} 
  \begin{itemize}
  \item
  \verb/a_max_time/ :
  Time to stop the calculation.
  \item
  \verb/a_max_step/ :
  Maximum number of iterations.
  \end{itemize}

\item
\verb/void conclude() const/:
The user should call this last.  It writes the last checkpoint file
and performs other housekeeping functions.

\end{itemize}

There are also functions in the {\tt AMR} class which allow the user to reset
various parameters of the run (blocking factor, regridding 
intervals, checkpointing intervals, etc.  See the reference
manual for details.  Examples of applications that
use the \verb|AMR| class to implement an adaptive Godunov method
are given in {\tt Chombo/example/AMRGodunov}.

\subsection{Class {\tt AMRLevel}}
{\tt AMRLevel} is an abstract base class for data at the same level of
refinement within a hierarchy of levels.  The concrete class
derived from {\tt AMRLevel} is called a {\em physics class}.  The domain
of a level is a disjoint union of rectangles in a logically
rectangular index space.  Data is defined within this domain.
There is also a problem domain, which may be larger, within which
data can, in theory, be interpolated from some coarser level.

{\tt AMRLevel} is the interface that the class {\tt AMR}
uses to call the physics class.   The important parts
of the public interface to {\tt AMRLevel} are:
\begin{itemize}

\item
\begin{verbatim}
virtual void define (AMRLevel* a_coarser_level_ptr,
                     const ProblemDomain& a_problem_domain,
                     int a_level,
                     int a_ref_ratio);

virtual void define (AMRLevel* a_coarser_level_ptr,
                     const Box& a_problem_domain,
                     int a_level,
                     int a_ref_ratio);
\end{verbatim}
Defines this \verb|AMRLevel|.
\\ {\bf Arguments:}
  \begin{itemize}
  \item
  \verb/coarser_level_ptr/ (not modified):
  Pointer to next coarser level object.
  \item
  \verb/problem_domain/ (not modified):
  Problem domain of this level.
  \item
  \verb/level/ (not modified):
  Index of this level.  The base level is zero.
  \item
  \verb/ref_ratio/ (not modified):
  The refinement ratio between this level and the next finer level.
  \end{itemize}

\item
\begin{verbatim}
virtual Real advance() = 0;
\end{verbatim}
Advance this level by one time step.
Return an estimate of the new time step at this level.

\item
\begin{verbatim}
virtual void postTimeStep() = 0;
\end{verbatim}
Do all operations that are required after a timestep is completed.
Refluxing happens here.

\item
\begin{verbatim}
virtual void tagCells(IntVectSet& a_tags) const = 0;
\end{verbatim}
Create tagged cells for dynamic mesh refinement.

\item
\begin{verbatim}
virtual void tagCellsInit(IntVectSet& a_tags) const = 0;
\end{verbatim}
Create tagged cells for mesh refinement at initialization.

\item
\begin{verbatim}
virtual void preRegrid(int a_base_level,
                       const Vector<Vector<Box> >& a_new_grids);
\end{verbatim}
Perform any pre-regridding operations which are necessary.
This is not a pure virtual function, to preserve compatibility
with earlier versions of \verb|AMRLevel|.  The \verb|AMRLevel::preRegrid()|
instantiation does nothing.

\item
\begin{verbatim}
virtual void regrid(const Vector<Box>& a_new_grids) = 0;
\end{verbatim}
Redefines this level to have the specified grids as its defined union of
rectangles.

\item
\begin{verbatim}
virtual void postRegrid(int a_base_level);
\end{verbatim}
Perform any post-regridding operations which are necessary.
This is not a pure virtual function to preserve compatibility
with earlier versions of \verb|AMRLevel|.  The \verb|AMRLevel::postRegrid()|
instantiation does nothing.

\item
\begin{verbatim}
virtual void initialGrid(const Vector<Box>& a_new_grids) = 0;
\end{verbatim}
Initialize this level to have the specified domain \verb/a_new_grids/.

\item
\begin{verbatim}
virtual void initialData() = 0;
\end{verbatim}
Initialize the data.

\item
\begin{verbatim}
virtual void postInitialize() = 0;
\end{verbatim}
Do any operations that are required just after initialization.

\item
\begin{verbatim}
virtual Real computeDt() = 0;
\end{verbatim}
Returns maximum stable time step for this level.

\item
\begin{verbatim}
virtual Real computeInitialDt() = 0;
\end{verbatim}
Returns maximum stable time step for this level with initial data.

\item
\begin{verbatim}
virtual void writeCheckpointHeader(HDF5Handle& a_handle) const = 0;
\end{verbatim}
Write the header to the checkpoint file handle.

\item
\begin{verbatim}
virtual void writeCheckpointLevel(HDF5Handle& a_handle) const = 0;
\end{verbatim}
Write checkpoint data for this level.
 
\item
\begin{verbatim}
virtual void readCheckpointHeader(HDF5Handle& a_handle) = 0;
\end{verbatim}
Read checkpoint header.

\item
\begin{verbatim}
virtual void readCheckpointLevel(HDF5Handle& a_handle) = 0;
\end{verbatim}
Read checkpoint data for this level.
 
\item
\begin{verbatim}
virtual void writePlotHeader(HDF5Handle& a_handle) const = 0;
\end{verbatim}
Write plot file header for this level.

\item
\begin{verbatim}
virtual void writePlotLevel(HDF5Handle& a_handle) const = 0;
\end{verbatim}
Write plot file data for this level.

\end{itemize}

These are all the pure virtual functions of the {\tt AMRLevel}
interface and therefore all the functions that the user {\bf must}
define for her application.  There are other ancillary functions in
the interface that have reasonable defaults.  Most of these involve
data member access and modification.

\subsection{Class {\tt AMRLevelFactory}}

The class {\tt AMRLevelFactory} is a pure virtual base class, with only
one member function:

\begin{itemize}

\item
\begin{verbatim}
virtual AMRLevel* new_amrlevel() const = 0;
\end{verbatim}
This is the only member function of {\tt AMRLevelFactory}, and it must
be defined by the user in a derived class. The derived function will
return a pointer to a physics-specific class derived from {\tt
AMRLevel}.

\end{itemize}
A pointer to an object of this class is passed to the 
define function of an {\tt AMR} object, which uses it to construct the 
various {\tt AMRLevel} objects that it requires.
